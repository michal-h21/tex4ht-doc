\section{Sectioning commands}
\DocConfigure{unit-name}
{start}
{end}
{before title}
{after title}\EndDoc

This command determines the content to be inserted at the mentioned locations of the specified units. 

Example:

\begin{texsource}
\Configure{chapter}
{\HCode{<section class="chapter">}}  {\HCode{</section>}}
{\HCode{<h2 class="chapterHead">}\chaptername ~\TitleMark
\HCode{<br />}}
{\HCode{</h2>}}
\end{texsource}



\DocCommand{ConfigureMark}\marg{unit-name}\marg{print section number}


Defines a macro \cmd{<<unit-name>>HMark} to hold the given argument. Upon
entering the unit, \cmd{TitleMark} command gets the content of this macro.

Some built-in configurations of \texfourht{} require an argument for the
\cmd{<<unit-name>>HMark} commands. For safety, these commands should
always be followed by a, possiblely empty, argument.  The argument
should be a separator between the title mark and its content.

Example:

\begin{texsource}
\Configure{section}
   {}{}
   {\HCode{<h3>}\TitleMark\space}    {\HCode{</h3>}}
\ConfigureMark{section}{\thesection}
\end{texsource}


\DocCommand{NewSection}\cmd{unit} \marg{mark-for-toc}

This directive introduces a sectioning command \cmd{unit}, which submits
\textit{mark-for-toc} to the tables of contents.

\begin{texsource}
\newcounter{c}
\NewSection\X {\thec}
\Configure{X}
{}{} 
{\refstepcounter{c}\ifvmode\IgnorePar\fi\EndP\HCode{<h2>}\TitleMark\space}
{\HCode{</h2>}}
\ConfigureMark{X}{\arabic{c}}
\end{texsource}

\subsection{Links to and from TOC}
\DocConfigure{toTocLink} {start link} {end link}\EndDoc

Each unit title contains a \cmd{Link}\marg{\ldots}\marg{\ldots}\ldots\cmd{EndLink} command.
The first argument of \cmd{Link} points to the first table of contents
referencing the title. The second argument provides an anchor
for references to the title (mainly from tables of contents).

The package option \option{section+} requests the inclusion of the
title within the anchor.  Without this option, the link command
resides between the title mark and its content.

The \cmd{Configure}\marg{toTocLink} command is provided for configuring
the \cmd{Link} and \cmd{EndLink} instructions.  In the default setting,
when the \option{sections+} option is not activated, the \cmd{Link}
command is altered to replace its first argument with an empty
argument.

Example:

\begin{texsource}
\Configure{toTocLink}
     {\Link}
     {\ifx \TitleMark\sectionHMark
        \Picture[\up]{haut.jpg align="right"}%
        \EndLink
        \TitleMark\space
      \else \EndLink \fi
     }
  \def\up{[up]}
\end{texsource}

\DocConfigure{toToc} 
{unit type}
{desired contents type  (if empty, `unit type' is assumed)}\EndDoc

You can also pass special values to \textit{unit type}:

\begin{description}
  \item[empty] stop adding entries of `unit type' to toc
  \item[@] add entries of `unit type' to toc
   \item[?] resume mode in effect before the last stop
\end{description}

Example: 
\begin{texsource}
\Configure{toToc}{chapter}{likechapter}
\end{texsource}

Introduces chapter as likechapter into TOC.

Example that temporarily suppresses inclusion of chapter to TOC:

\begin{texsource}
\Configure{toToc}{}{chapter}
\chapter{...}
\Configure{toToc}{@}{chapter}
\end{texsource}

\DocCommand{NoLink}\marg{section name}

Ignore option \option{section+} for sections of type \#1


\DocConfigure{NoSection} {before} {after}\EndDoc

Insertions around the parameters of sectioning commands, applied when
the parameters are not used to create titles for the divisions.
