\section{Tables of Contents}

\texfourht{} supports tables of contents created using the \cmd{tableofcontents}
command in the converted document. In addition, it can add local tables of contents
to the 

Created from the entries collected in the previous compilation within
a \file{\cmd{jobname}.4tc} file.


\DocConfigure{tableofcontents}
 {before-toc} {end-of-toc} {after-toc} {before-nonindented-par} {before-indented-par}\EndDoc

The \textit{end-of-toc} is inserted at the end of the internal environment of the tables. 
The \textit{after-toc} is included after leaving the internal environment. 

\DocCommand{tableofcontents}\oarg{unit-1,unit-2,\ldots}
\DocCommand{TableOfContents}\oarg{unit-1,unit-2,\ldots}

The \cmd{tableofcontents} command may be followed by a comma separated
list of sectioning unit names to be included in the table of
contents.  The list should be enclosed within square brackets.
Alternatively, a command of the form \cmd{TableOfContents}\oarg{\ldots} might
be used.

\DocConfigure{writetoc} {definitions-for-the-writing-environment}\EndDoc

\texfourht{} expands and then writes the sectioning titles into an auxiliary
file, and it might encounter there problems from macros that are not fit for
such conditions or for inclusion in the table of contents. The current
configuration instruction allows to locally modify the behavior of macros for
the writing phase.


New configurations are added to those request earlier by the command.  An empty
argument cancels the earlier contributions.

For instance, the instruction \texcommand{\section{Foo \\ bar}} suggests a configuration
similar to \texcommand{\Configure{writetoc}{\let\\\space}}. 

\subsection{Local tables of contents}

The following commands and configurations can produce tables of contents
for chapters or sections.

\DocCommand{ConfigureToc}\marg{unit-name}%
\marg{before unit number}%
\marg{before content}%
\marg{before page number}%
\marg{at end}%

\begin{itemize}
  \item Empty arguments request the omission of the corresponding field.

  \item \cmd{TocCount}  Specifies the entry count withing the \file{\cmd{jobname}.4tc} file.

  \item \cmd{TitleCount} Count of entries submitted to the toc file
\end{itemize}

Example:

\begin{texsource}
\ConfigureToc{section}
   {}
   {\Picture[*]{pic.jpg width="13"  height="13"}~}
   {}
   {\HCode{<br />}}
\end{texsource}

As an alternative to \cmd{ConfigureToc}, you can also use the following alternative:

\begin{texsource}
\def\toc<unit-name>#1#2#3{<before unit number>#1<before content>#2%
    <before page number>#3<at end>}
\end{texsource}




\DocConfigure{TocLink} {\cmd{Link} like command}\EndDoc

Configures the link offered in the third arguments of \cmd{ConfigureToc}

Example:   

\begin{texsource}
\Configure{TocLink}{\Link{#2}{#3}#4\EndLink}
\end{texsource}

\DocCommand{TocAt}\marg{\#1,\#2,\#3,\ldots}

\begin{description}
  \item[\#1]           section type for which local tables of contents
                 \cmd{Toc\#1} are requested
  \item[\#2,\#3,\ldots]    sectioning types to be included in the tables of contents
\end{description}

The non-leading arguments may be preceded by slashes '/', in
which cases the arguments specify end points for the tables.

The default setting requests automatic insertion of the local
tables immediately after the sectioning heads.

A star `*' character may be introduced, between the  \cmd{TocAt} and
the left brace, to request the appearances of the tables of
contents at the end of the units' prefaces.

A hyphen `-' character, on the other hand, disables the automatic
insertions of the local tables.

In case of a single argument, the command removes the
existing definition of \cmd{Toc\#1}.

    Example:

\begin{texsource}
\TocAt{mychapter,mysection,mysubsection,/myappendix,/mypart}
\TocAt-{mysection,mysubsection,/mylikesection}
\section{...}...\Tocmysection
\end{texsource}

The definition  of the local table of contents can be redefined
within \texcommand{\csname Toc#1\endcsname}.

Example:

\begin{texsource}
\TocAt{section}
\def\Tocsection{\TableOfContents[section]}

\Css{div.sectionTOCS {
                    width : 30\%;
                    float : right;
               text-align : left;
           vertical-align : top;
              margin-left : 1em;
                font-size : 85\%;
         background-color : \#DDDDDD;
    }}
\end{texsource}

Example: Table of content before the section title.

\begin{texsource}
\Configure{section}{}{}
   {\Tocsection \let\saveTocsection=\Tocsection
    \def\Tocsection{\let\Tocsection=\saveTocsection}%
    \ifvmode \IgnorePar\fi \EndP\IgnorePar
    \HCode{<h3 class="sectionHead">}\TitleMark\space\HtmlParOff}
   {\HCode{</h3>}\HtmlParOn\ShowPar \IgnoreIndent \par}
 \end{texsource}

\DocConfigure{TocAt} {before the tables of contents} {after the tables of contents}\EndDoc
\DocConfigure{TocAt*} {before the tables of contents} {after the tables of contents}\EndDoc

These configurations can be used to execute code before and after \cmd{TocAt}, respective
\cmd{TocAt*}.

\DocConfigure{tableofcontents*} {section types}\EndDoc

A non-empty  parameter asks to implicitly introduce
a \cmd{tableofcontents} upon reaching the first sectioning
command, if no \cmd{tableofcontents} command is encountered
earlier. The parameter assumes a colon-separated list
of sectioning types to be included in the output
of \cmd{tableofcontents}.  Starred variants of sectioning
types should be referenced with the `like' prefix.

An empty parameter cancels earlier requests for implicit
calls to \cmd{tableofcontents} (e.g., embedded within requests
to package options 1, 2, 3, 4)

Example:

\begin{texsource}
\Configure{tableofcontents*}{part,likepart,%
           section,likesection,subsection,likesubsection}
\end{texsource}

\DocCommand{contentsname} A \LaTeX{} macro stating the title for a table of contents division.



   


