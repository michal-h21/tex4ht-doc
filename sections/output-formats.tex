\texfourht\ supports following output formats:

\begin{description}
  \item[HTML 5] used by default, when you don't provide the \shellcmd{-f} option.
  \item[XHTML]  alternative if you don't want to use new HTML elements.
  \item[ODT]    suitable for LibreOffice or MS Word.
\end{description}

There are also historicaly other formats with more limited support. Basic structures,
like sections or lists should work, but specific formatting for most of \LaTeX\ packages 
can be missing.

\begin{description}
  \item[DocBook]  
  \item[TEI] 
  \item[JATS] support for this format is limited, even basic document structure is missing.
\end{description}

Additionally, you can create e-book formats such as EPUB or MOBI. For historical reasons, these formats
are created using separate tool, \texttt{tex4ebook}. It supports the folowing formats:

\begin{description}
  \item[AZW]
  \item[AZW 3]
  \item[EPUB]
  \item[EPUB 3]
  \item[MOBI]
\end{description}

To convert your file to any of these formats, use it as a lowercase name, without spaces, in the
\texttt{-f} option of \makefourht:

\begin{shellcommand}
$ make4ht -f odt filename.tex
$ tex4ebook -f epub3 filename.tex
\end{shellcommand}

