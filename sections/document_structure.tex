\section{Basic document structure and metadata}
\label{sec:document_structure}

\subsection{File extension}

\DocConfigure {ext} {default extension name for target files  (recorded in \cmd{:html})}\EndDoc

It can also be requested through a command line option \shellcmd{ext=...}.

\subsection{HTML header}
\DocConfigure {PROLOG} {Comma separated list of hooks to appear before HTML}\EndDoc

Each hook E is declared to be configurable by an instruction of the form \texcommand{\NewConfigure{E}{1}}


Example:

\begin{texsource}
\Configure{PROLOG}{VERSION,DOCTYPE,*XML-STYLESHEET}
\Configure{VERSION}{\HCode{<?xml version="1.0"?>}}
\end{texsource}

A star '*' prefix calls for accumulative configurations

\subsection{Document structure}

The following configurations are used to declare tags for basic structure of generated documents.


\DocConfigure{DOCTYPE} {doctype definition}\EndDoc
\DocConfigure{HTML} {HTML start} {HTML end}\EndDoc
\DocConfigure{HEAD} {HEAD start} {HEAD end}\EndDoc
\DocConfigure{@HEAD} {content added to HEAD}\EndDoc
\DocConfigure{BODY} {BODY start} {BODY end}\EndDoc
\DocConfigure{TITLE} {TITLE start} {TITLE end}\EndDoc
\DocConfigure{Preamble} {code executed at the very beging} {code executed at \cmd{EndPreamble}}\EndDoc




The \texcommand{\Configure{@HEAD}{...}} command is additive, concatenating the
content of all of its appearances.  An empty parameter requests
the cancellation of the earlier contributions.

For instance,

\begin{texsource}
\Configure{@HEAD}{A}
\Configure{@HEAD}{}
\Configure{@HEAD}{B}
\Configure{@HEAD}{C}
\end{texsource}

contributes `BC'.


\DocConfigure{@BODY} {additive insertions at start of BODY}\EndDoc
\DocConfigure{@/BODY} {additive insertions at end of BODY}\EndDoc

   Variants of \texcommand{\Configure{@HEAD}} which contribute their content,
   respectively, after \htmlcommand{<body>} and before \htmlcommand{</body>}.


The document structure hooks are placed in the following sequence:


\begin{texsource}
<DOCTYPE>
<HTML 1>
  <HEAD 1>
     <TITLE 1>
        <@TITLE>
        <TITLE+>
     <TITLE 2>
     <@HEAD>
  <HEAD 2>
  <BODY 1>
  <@BODY>
  ......
  <@/BODY>
  <BODY 2>
<HTML 2>
\end{texsource}


\subsection{Titles}

\DocConfigure{TITLE+} {Title contents}\EndDoc

Set contents of the \htmlcommand{<title>} element.

There are also similar configurations for pages split on sectioning commands. For example, for document that
is split on chapters, use:

\begin{texsource}
\Configure{TITLE+}{My web site |\@title}
\Configure{chapterTITLE+}{My web site | \thechapter\space#1}
\Configure{likechapterTITLE+}{My web site | #1}
\end{texsource}

\texttt{likechapter} prefix is used for \texcommand{\chapter*} command. The text of sectioning command
can be accessed using \texcommand{#1}.

\DocConfigure{@TITLE} {redefinitions of commands in the title group}\EndDoc

Acts as a hook for introducing localized configurations. As is the case for
\texcommand{\Configure{@HEAD}}, the contribution of \texcommand{\Configure{@TITLE}} is also
additive.

\DocConfigure{CutAtTITLE+} {an insertion just before the content of TITLE}\EndDoc
\DocConfigure{HPageTITLE+} {an insertion just before the content of TITLE}\EndDoc

If \texcommand{#1} is a one parametric macro, it gets the title content for an
argument.

