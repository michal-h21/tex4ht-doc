
The translation of a LaTeX source file into HTML involves of loading tex4ht.sty
and *.4ht style files, choosing the desirable options for the translation,
compiling the source into dvi code with the native LaTeX engine, and
postprocessing the outcome with the tex4ht and t4ht programs (see \nameref{sec:overview}). 


\section{Overview of the Translation Process}\label{sec:overview}

The system can be activated with a sequence of commands of the following form, typically embedded within a script.

\begin{shellcommand}
latex      x            (or ‘tex x’) 
latex      x 
latex      x 
tex4ht     x 
t4ht       x 
\end{shellcommand}

The three compilations with La(TeX) are needed to ensure proper links. The approach is illustrated in the following picture. 

\begin{description}
  \item[x.tex]

This is a source TeX/LaTeX/OtherTeX file that imports the style files tex4ht.sty and *.4ht. The style files define the features for the output.

\item[tex4ht]

The output of \TeX{} is a standard dvi file interleaved with special
instructions for the postprocessor \shellcmd{tex4ht} to use. The special
instructions come from implicit and explicit requests made in the source file
through commands of \texfourht.

The utility tex4ht translates the dvi code into standard text, while obeying
the requests it gets from the special instructions. The special instructions
may request the creation of files, insertion of html code, filtering of
pictures, and so forth.

In the extreme case that the source code contains no commands of TeX4ht, tex4ht
gets pure dvi code and it outputs (almost) plain text with no hypertext
elements in it.

The special (\texcommand{\special}) instructions seeded in the dvi code are not understood
by dvi processors other than those of TeX4ht.

\item[x.idv]

This is a dvi file extracted from x.dvi, and it contains the pictures needed in
the html files.

\item[x.lg]

This is a log file listing the pictures of x.idv, the png files that should be
created, CSS information, and user directives introduced through the
‘\texcommand{\Needs{...}}’ command.

\item[t4ht]
This is an interpreter for executing the requests made in the x.lg script.

\end{description}


