\chapter{\texfourht{} Options}
\label{chap:options}

% the list of options has been copied from the CVR's blog
% http://cvr.cc/?p=504


Following is an incomplete list of options that can be passed to the \fourhtsty\ package.
These options can be used to modify the compilation process, for example to
select a \term{SVG} format for generated images, to request math environments
to be convert as images or to split sections as separate HTML pages. 

The options may be defined in the \fourhtfile\ files and may depend on the
output format, so it is not feasible to provide their full list. Most of the
following options work only in the \HTML\ output.

There are several ways how to pass the options to \texfourht. The
non-recommended way is to pass them as options to \fourhtsty\ using
\texcommand{\usepackage} command in the \TeX\ file. 

\begin{texsource}
...
\usepackage[xhtml,option1,option2]{tex4ht}
...
\end{texsource}

Recommended solutions don't require modifications of the \TeX\ files.

One way is to pass the options to \texttt{make4ht} as an command line
argument, next to the filename:
% it is run from the command line. These can also be provided as options when \texfourht
% package is loaded in a \LaTeX\ document with the default usepackage command or to
% the \verb|\Preamble| command in the custom config file

\begin{shellcommand}
make4ht filename.tex "fn-in"
\end{shellcommand}

For more information on calling scripts see the section \ref{sec:calling-commands}.

The second way is to pass options in the \texcommand{\Preamble} command in a \cfgfile. Note that first option in the 
\texcommand{\Preamble} command must be \option{xhtml}.

\begin{texsource}
\Preamble{xhtml,fn-in}
...
\begin{document}
\EndPreamble
\end{texsource}

See the section \namerefpage{sec:private-configuration} for more details on
configuration files.


\section{List of options}
\label{sec:texfouhtoptions}

% \begin{tabular}{>{\ttfamily}p{8em} l} 
%   -css & to ignore CSS code, use command line option -css. \\
%   -xtpipes & to avoid xtpipes post-processing the output. This might be useful for docbook XML output.\\
%   0 & pagination shall be obtained through the option 0 or 1, at locations marked with PageBreak.\\
%   1, 2, 3, 4, 5, 6, 7& for automatic sectioning pagination (to break at various section levels), use the appropriate command line option 1, 2, 3, 4, 5, 6, 7.
\begingroup
\catcode`\#=11 \catcode`\^=11 \catcode`\_=11


\begin{description}

\item[-css] to ignore \css\ code, use command line option \verb=-css=.

\item[-xtpipes] to avoid \verb=xtpipes= post-processing the
  output. This might be useful for Docbook \xml\ or ODT output.

  % \item[/bib]
  % \item[/obeylines]
  % \item[0.0]

\item[0] pagination shall be obtained through the option \verb=0= or
  \verb=1=, at locations marked with \verb=\PageBreak=.

\item[1, 2, 3, 4, 5, 6, 7] for automatic sectioning pagination (to
  break at various section levels), use the appropriate command line
  option \verb=1, 2, 3, 4,= \verb=5, 6, 7=. Option 1 break pages
  at parts, 2 at chapters, 3 at sections, 4 at subsections,
  5 at subsubsections and 6 at paragraphs. 


\item[DOCTYPE] to request a \verb=DOCTYPE= declaration, use the
  command line option \verb=DOCTYPE=.

\item[Gin-dim] for key dimensions of the graphic, try this option.

\item[Gin-dim+] for key dimensions when the bounding box is not
  available.

\item[Gin-percent] calculate size of the graphics on the relation to the 
  page width.

\item[NoFonts] don't use original font style information.
\item[NoSections] don't use frames for text alignment environments in 
  the ODT output.

\item[PMath] Option to choose positioned math. Example: 
  \verb=\def\({\PMath$}=;\allowbreak \verb=\def\){$\EndPMath}=;
  \verb=\def\[{\PMath$$}=; \verb=\def\]{$$\EndPMath}=.

\item[RL2LR] to reverse the direction of RL sentences.

%\item[ShowFont]

\item[TocLink] option to request links from the tables of contents.
  
\item[\textasciicircum 13] option for active superscript character.

\item[\_13] option for active subscript character.

\item[accent-] This option is available only together with
  \option{new-accents}. It produces pictures for some math accents.

%\item[base]

\item[bib-] for degraded bibliography friendlier for conversion to
  \verb=.doc=.

\item[bibtex2] Option \verb=bibtex2= requires compilation of
  \verb=\jobname j.aux= with bibtex.

%\item[broken-index]

\item[charset] for alternate character set, use the command line
  option \verb+charset=...+ (e.g., \verb+charset=utf-8+).

%\item[core]

\item[css-in] the inline \css\ code will be extracted from the input of
  the previous compilation, so an extra compilaion might be needed for
  this option to make it effective.

\item[css2] for \css2 code.

% \item[css]
% \item[debug-]
% \item[debug]
% \item[draw]
% \item[dtd]

\item[cut-fullname] by default filenames for files created by section breaking
  options base their names on truncated section type names. Use this option to
  disable truncating.

\item[early\textasciicircum] for default catcode of superscript in the
  \verb=\Preamble=.

\item[early\_] for default catcode of subscript in the
  \verb=\Preamble=.

%\item[edit]

\item[endnotes] for end notes instead of footnotes, use this option.

%\item[enum]

\item[enumerate+] for enumerated list elements that keep the list couter value. This
  will use the description list like \verb=<dt>...</dt>= for the list
  counter.

\item[enumerate-] for enumerated list element's \verb=<li>='s with
  value attributes, use this command line option. This will be an
  ordered list with the value of list counter provided as an attribute
  namely, \verb=value= of the \verb=<li>= element.

%\item[family]
\item[fancylogo] try to visually emulate \verb|\TeX| and \verb|\LaTeX| logos.

\item[fn-in] for inline footnotes use this option.

\item[fn-out] for offline footnotes.

\item[fonts] use HTML elements and CSS for \latex\ font commands, such as
  \verb|\textit|.

\item[fonts+] for marking of the base font, use this option.

\item[font] for adjusted font size, use the command line option
  \verb+font=...+ (e.g., font=-2).

\item[frames-] for frames support. \verb=frames= is also valid option
  for frames support.

\item[frames-fn] for content, \chfont{TOC}\ and footnotes in
  three frames.

\item[frames] for \chfont{TOC}\ and content in two frames.

%\item[fussy]

\item[gif] for bitmaps of pictures in \verb=.gif= format, use this
  option.

\item[graphics-] if the included graphics are of degraded quality, try
  the command line options \verb=graphics-num= or \verb=graphics-=.
  The \verb=num= should provide the density of pixels in the bitmaps
  (e.g., 110).

%\item[graphics-dim]

\item[hidden-ref] option to hide clickable index and bibliography
  references.

% \item[hooks++]
% \item[hooks+]
% \item[hooks]
% \item[hshow]
% \item[htm3]
% \item[htm4]
% \item[htm5]
% \item[htm]

\item[html+] for stricter \HTML\ code.

%\item[html]

\item[imgdir] for addressing images in a subdirectory, use the option
  \verb=\imgdir:.../=.

\item[image-maps] for \verb=image-maps= support.

\item[index] for \emph{n}-column index, use the command line option,
  \verb+index=n+ (e.g., index=2).

\item[info-oo] for extra tracing information while generating open
  office output.

\item[info] for extra information in the \verb=\jobname.log= file.

\item[itemize+] for original characters used as bullets in the \verb|itemize| list.

\item[java] for \verb=java=support.

\item[javahelp] for \verb=JavaHelp= output format, use this command
  line option.

\item[javascript] for \verb=javascript= support.

\item[jh-] for sources failing to produce \xml\ versions of \HTML, try
  this command line option.

%\item[jh1.0]

\item[jpg] for bitmaps of pictures in \verb=.jpg= format, use this
  option.

\item[li-] for enumerated list elements li's with value attributes.

\item[math-] option to use when sources fail to produce clean math
  code.

\item[mathjax] use \term{MathJax} for the math rendering.
%\item[mathaccent-]

\item[mathltx-] option to use when sources fail to produce clean
  \verb=mathltx= code.

\item[mathml-] option to use when sources fail to produce clean
  \mathml code.

\item[mathplayer] for \mathml\ on Internet Explorer + MathPlayer.

\item[minitoc\textless] for mini tocs immediately after the header use the
  command line option, \verb=minitoc<=.

\item[mouseover] for pop ups on mouse over.

\item[new-accents] alternative configurations for accented characters. 

\item[next] for linear cross-links of pages, use this option.

\item[nikud] for Hebrew vowels, use the command line option,
  \verb=nikud=.

\item[no-DOCTYPE] to remove \texttt{DOCTYPE}\space declaration from
  the output.

\item[no-VERSION] to remove \verb+<?xml version="..."?>+ processing
  instruction from the output.

\item[NoFonts] disable ht-fonts processing in the document.

% \item[no-align]
% \item[no-array]
% \item[no-bib]
% \item[no-cases]
\item[no-halign] suppress \texcommand{\halign} redefinition. It doesn't work with the \texcommand{tabular} environment.
% \item[no-matrix]
% \item[no-pmatrix]

\item[no\textasciicircum] for non-active \verb=^= (superscript), use the option
  \verb=no^=.

\item[no\_] for non-active \verb=_= (subscript command), use the
  command line option, \verb=no_=.

\item[no\_\textasciicircum] for both non-active superscript and subscript, use the
  option \verb=no_^=.

\item[nolayers] to remove overlays of slides, use this option.

\item[nominitoc] this will eliminate mini tables of contents from the
  output.

\item[nostyle] to prevent default CSS style for section titles and page dimensions.

\item[notoc*] for tocs without \verb=*= entries, use this option. The
  \verb=notoc*= option is applicable only to pages that are
  automatically decomposed into separate web pages along section
  divides. It shall be used when \verb=\addcontentsline= instructions
  are present in the sources.

\item[obj-toc] for frames-like object based table of contents, use the
  command line option \verb=obj-toc=.

%\item[old-longtable]

\item[p-width] for width specifications of tabular \verb=p= entries,
  use this option.

\item[p-indent] for indented paragraphs, without blank spaces.

\item[pic-RL] for pictorial RL.

\item[pic-align] for pictorial align environment.

\item[pic-array] for pictorial array.

\item[pic-cases] for pictorial cases environment.

\item[pic-eqalign] for pictorial equalign environment.

\item[pic-eqnarray] for pictorial eqnarray.

\item[pic-equation] for pictorial equations.

\item[pic-fbox] for pictorial or bitmapped fbox'es.

\item[pic-framebox] for bitmap fameboxes.

\item[pic-longtable] for bitmapped longtable.

\item[pic-m+] for pictorial \verb=$...$= and \verb=$$...$$=
  environments with \latex\ alt, use the command line option
  \verb=pic-m+= (not safe).

\item[pic-m] for pictorial \verb=$...$= environments, use the command
  line option \verb=pic-m= (not recommended).

\item[pic-matrix] for pictorial matrix.

% \item[pic-tabbing']

% \item[pic-tabbing]

% \item[pic-table]

\item[pic-tabular] use this option for pictorial tabular.

\item[plain-] for scaled down implimentation.

% \item[pmathml-css]

% \item[pmathml]

% \item[postscript]

\item[prog-ref] for pointers to code files from root fragments, use
  the command line option \verb=prof-ref=. This is for debugging.

\item[refcaption] for links into captions, instead of flat heads, use
  this option.

\item[rl2lr] to reverse the direction of Hebrew words, use this
  option.

\item[sec-filename] for file names derived from section titles, use
  the command line option \verb=sec-filename=.

\item[sections+] for back links to table of contents, use this option.

% \item[sections-]
% \item[settabs-]
% \item[stackrel-]

\item[svg-] for external \svg\ files, try this option.

\item[svg-obj] same as above.

\item[svg] for dvi pictures in \verb=svg= format.

\item[svg-inline] same as the \option{svg}, but the \svg\ files are included in the document body.

\item[tab-eq] for tab-based layout of equation environment, use this
  option.

%\item[th4]

\item[trace-onmo] for mouseover tracing of compilation, use the
  command line option, \verb=trace-onmo=.

% \item[uni-emacspeak]
% \item[uni-html4]
% \item[uniaccents]
% \item[unicode]

% \item[url-]

\item[url-enc] for \chfont{URL}\space encoding within href, use this
  option.  \verb=\Configure{url-encoder}= can be used to fine tune
  encoding.

\item[url-il2-pl] for il2-pl \chfont{URL} encoding.

\item[ver] for vertically stacked frames. Effective when \verb=frames=
  option is requested.

% \item[verify+]
% \item[verify]

\item[xht] for file name extension, \verb=.xht=, use this command line
  option.

\item[xhtml] for \xml\ code, use the command line option, \verb=xml= or
  \verb=xhtml=.

\item[xml] See previous entry.

% \item[xmldtd]

\end{description}

\section{Options for the ODT output}

\begin{description}
  \item[bib-] produces degraded bibliography that should be friendlier for conversion to Word.
  \item[description-inline] use run-in style for description lists.
  \item[endnotes] convert footnotes to endnotes.
  \item[hidden-ref] hide clickable index and bibliography references.
  \item[NoSections] don't use sections for text alignment environments.
  \item[tab-eq] tab based layout of equations.
  \item[timestamp] save creation date in the document metadata.
  \item[TocLink] request links from table of contents.
\end{description}
\endgroup
