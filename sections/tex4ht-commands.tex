I've created some alternative commands to \texcommand{\HCode} or \texcommand{\Tg}. The idea is to define
semantic names for logical elements of the document, such as titles, authors,
sections etc. It is possible to assign HTML elements and attributes to these
logical elements. There are commands for inline and block level elements,
which should eliminate the need for constructs like \texcommand{\ifvmode\IgnorePar\fi\EndP}
etc.

I think it will be best to show some concrete examples:


\begin{texsource}
\NewLogicalBlock{textit}
\SetBlockProperty{textit}{class}{textit}

\NewLogicalBlock{maketitle}
\SetTag{maketitle}{header}

\NewLogicalBlock{titlehead}
\SetTag{titlehead}{h1}
\SetBlockProperty{titlehead}{class}{titleHead}

\Configure{textit}
{\NoFonts\InlineElementStart{textit}{}}
{\InlineElementEnd{textit}\EndNoFonts}

\Configure{maketitle}{%
{\Configure{maketitle}{}{}{}{}%


\Tag{TITLE+}{\@title}}
\BlockElementStart{maketitle}{}}
{\BlockElementEnd{maketitle}}
{\NoFonts\BlockElementStart{titlehead}{}}
{\BlockElementEnd{titlehead}\EndNoFonts}
\end{texsource}



The \texcommand{\NewLogicalBlock} creates a new logical element. The used tag is configured
using \texcommand{\SetTag}, the attributes are set using \texcommand{\SetBlockProperty}. The blocks are
inserted to the document using \texcommand{\InlineElementStart} ...  \texcommand{\InlineElementEnd} or
\texcommand{\BlockElementStart} ... \texcommand{\BlockElementEnd} pairs. The start commands take two
parameters, first is the logical block name, the second can be local
parameters which shouldn't be used for the given logical block globally.

The main idea behind this mechanism is to allow easy work with new HTML5
elements and attributes for WAI-ARIA or Schema.org properties. I hope that
this should allow us to make somehow more clear configurations for basic
document building blocks.
