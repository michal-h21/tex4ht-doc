\subsection{Low-level \texfourht\ Commands}

\cmd{Configure}\marg{name}\marg{arg 1}\ldots\marg{arg n}

The \cmd{Configure} command declares code that is inserted into hooks related to the \textit{name} configuration.
It is possible to define up to nine hooks, so number of arguments is variable.


\cmd{ConfigureEnv}\marg{\textless environment name\textgreater}\marg{before env}\marg{after env}
\marg{before-list}\marg{after-list}


\cmd{HCode}\marg{output format markup}

This is a basic command for insertion of the output format markup, as it's content is not escaped for the \textless and \textgreater.

This command allows only for the expansion of macros, before sending its content to the output. 
The instruction \texcommand{\Hnewline} may be introduced there for requesting line breaks, and the command \texcommand{\#} may be used for the sharp symbol ‘\#’.

\begin{texsource}
 First line\HCode{<br />}
 second line 

 You don't want to include tags directly into the document '<br>'. 
\end{texsource}

\cmd{Tg}\texcommand{<markup>}

This is a variation of the \cmd{HCode} command that don't require braces, and it does some additional processing.

\cmd{ifOption}\marg{name of the option to be checked}\marg{true part}\marg{false part}

This command can be used in the private configuration files to test if a custom option was used

\subsection{Hyperlinks}

\texcommand{\Link[target-file arguments]{target-loc}{cur-loc}anchor\EndLink}

This command requests an anchor that links to \verb|‘target-file#target-loc’|, and marks the current location with the name \texttt{‘cur-loc’}.

The component \texttt{‘[...]’} is optional when it is empty, and the target file need not be mentioned if it is created from the current source file.


\texcommand{\LinkCommand}


\subsection{Paragraph Handling}
\label{sec:paragraph_handling}

% https://tex.stackexchange.com/a/66172/2891

\texcommand{\Configure{HtmlPar}}
\texcommand{\IgnorePar}
\texcommand{\EndP}
etc.

\subsection{Logical Document Structure Commands}
I've created some alternative commands to \texcommand{\HCode} or \texcommand{\Tg}. The idea is to define
semantic names for logical elements of the document, such as titles, authors,
sections etc. It is possible to assign HTML elements and attributes to these
logical elements. There are commands for inline and block level elements,
which should eliminate the need for constructs like \texcommand{\ifvmode\IgnorePar\fi\EndP}
etc.

I think it will be best to show some concrete examples:


\begin{texsource}
\NewLogicalBlock{textit}
\SetBlockProperty{textit}{class}{textit}

\NewLogicalBlock{maketitle}
\SetTag{maketitle}{header}

\NewLogicalBlock{titlehead}
\SetTag{titlehead}{h1}
\SetBlockProperty{titlehead}{class}{titleHead}

\Configure{textit}
{\NoFonts\InlineElementStart{textit}{}}
{\InlineElementEnd{textit}\EndNoFonts}

\Configure{maketitle}{%
{\Configure{maketitle}{}{}{}{}%
\Tag{TITLE+}{\@title}}
\BlockElementStart{maketitle}{}}
{\BlockElementEnd{maketitle}}
{\NoFonts\BlockElementStart{titlehead}{}}
{\BlockElementEnd{titlehead}\EndNoFonts}
\end{texsource}



The \texcommand{\NewLogicalBlock} creates a new logical element. The used tag is configured
using \texcommand{\SetTag}, the attributes are set using \texcommand{\SetBlockProperty}. The blocks are
inserted to the document using 

\begin{texsource}
\InlineElementStart ... \InlineElementEnd
\end{texsource}

\noindent or

\begin{texsource}
\BlockElementStart ... \BlockElementEnd 
\end{texsource}

\noindent pairs. The start commands take two
parameters, first is the logical block name, the second can be local
parameters which shouldn't be used for the given logical block globally.

The main idea behind this mechanism is to allow easy work with new HTML5
elements and attributes for WAI-ARIA or Schema.org properties. I hope that
this should allow us to make somehow more clear configurations for basic
document building blocks.
