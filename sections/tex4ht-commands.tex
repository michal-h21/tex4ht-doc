\section{tex4ht Commands}
\label{sec:tex4ht-commands}

\subsection{Low-level \texfourht\ Commands}

\DocCommand{Configure}\marg{name}\marg{arg 1}\ldots\marg{arg n}

The \cmd{Configure} command declares code that is inserted into hooks related to the \textit{name} configuration.
It is possible to define up to nine hooks, so number of arguments is variable.


\DocCommand{ConfigureEnv}\marg{\textless environment name\textgreater}\marg{before env}\marg{after env}
\marg{before-list}\marg{after-list}


\DocCommand{HCode}\marg{output format markup} is a basic command for insertion
of the output format markup, as it's content is not escaped for the \textless{}
and \textgreater.

This command allows only for the expansion of macros, before sending its content to the output. 
The instruction \texcommand{\Hnewline} may be introduced there for requesting line breaks, and the command \texcommand{\#} may be used for the sharp symbol ‘\#’.

\begin{texsource}
First line\HCode{<br />}
second line 

You don't want to include tags directly into the document '<br>'. 
\end{texsource}

\DocCommand{Tg}\verb|<markup>| is a variation of the \cmd{HCode} command that don't require braces, and it does some additional processing.

\DocCommand{ifOption}\marg{name of the option to be checked}\marg{true part}\marg{false part}

This command can be used in the private configuration files to test if a custom option was used

\subsection{Hyperlinks}

\DocCommand{Link}\texttt{[target-file arguments]}\marg{target-loc}\marg{cur-loc} text inside link\cmd{EndLink}

This command requests an anchor that links to \verb|target-file#target-loc|, and marks the current location with the name \verb|‘cur-loc’|.

The component in square brackets \texttt{‘[...]’} is optional when it is empty, 
and the target file need not be mentioned if it is created from the current source file.


\DocCommand{LinkCommand} creates a \cmd{Link}\textit{-like} command. It has variable number for parameters:

\begin{enumerate}
  \item tag name
  \item href-like attribute
  \item name-like attribute
  \item insertion
  \item /, if empty element
  \item replacement for \texcommand{#} character  (ignored if absent)
\end{enumerate}

Example:

\begin{texsource}
\LinkCommand\JSLink{a,\noexpand\jsref,name}
\def\jsref="#1"{href="javascript:window.open('#1')"}

% example of use of the defined command:
\JSLink{a}{}xx\EndJSLink % link to a destination
\Link{}{a}\EndLink       % set link destination (or by \JSLink{}{a}\EndJSLink)
\end{texsource}

\DocCommand{Tag}\marg{label}\marg{content}
\DocCommand{Ref}\marg{label}
\DocCommand{LikeRef}\marg{label}

\cmd{Tag} and \cmd{Ref} are  commands introduced cross-referencing
content through \file{.xref} auxiliary files.

\cmd{LikeRef} is a variant of \cmd{Ref} which doesn't verify whether the
labels exit.  It is mainly used in \cmd{Link} and \cmd{edef} environments.

\DocCommand{ifTag}\marg{label}\marg{true part}\marg{false part}

Test existence of a tag, and execute code depending on that.


\subsection{Cross references}

For correct support of cross-references, you need to place link destinations for labels.
\texfourht{} handles labels defined by the \cmd{refstepcounter} command, 
but links for labels that use a direct setting of the \cmd{@currentlabel} command needs to be 
inserted manually.

\DocCommand{AnchorLabel} inserts link destination for the current \cmd{label} command. It will be used 
in \cmd{ref},  \cmd{pageref}, and other cross referencing commands that point to the label.
It should be placed after change of \cmd{@currentlabel}. 

\DocCommand{SkipRefstepAnchor} don't insert anchor for the next \cmd{refstepcounter}.

\DocCommand{ShowRefstepAnchor} show anchor for the next \cmd{refstepcounter}.

\subsection{Paragraph Handling}
\label{sec:paragraph_handling}

Paragraph handling is one of the more complicated areas in \texfourht.
You must handle insertion of tags for paragraph opening and closing,
to prevent wrong nesting of XML tags. Mismatch of tags leads to issues with 
LuaXML post-processing of the generated files, preventing many fixes 
that are necessary for correct conversion.

\DocCommand{HtmlParOff} turns off insertion of paragraph tags in the following text.

\DocCommand{HtmlParOn} enables insertion of paragraphs tags.

\DocCommand{IgnorePar} asks to ignore the next paragraph.
\DocCommand{ShowPar} asks to take into account the following paragraphs.

\DocCommand{IgnoreIndent}  asks to ignore indentation in the next paragraph.
\DocCommand{ShowIndent}    asks to check indentation in the following paragraphs.

\DocCommand{SaveEndP}  saves the content of \cmd{EndP}, and sets it to empty content.
\DocCommand{RecallEndP} resets the content of \cmd{EndP}.

\DocCommand{SaveHtmlPar} saves current configuration for paragraphs. It can be
useful before local declaration of \cmd{Configure}\marg{HtmlPar}.
\DocCommand{RecallHtmlPar} resets configuration for paragraphs to the value saved by
\cmd{SaveHtmlPar}.


The following example adds \htmlcommand{<div>...</div>} tags around contents of the document body.
The \texcommand{\ifvmode\IgnorePar\fi} commands will prevent insertion of the \htmlcommand{<p>} tag 
before \htmlcommand{<div>} if we are in \TeX's vertical mode. The \texcommand{\EndP} closes currently
opened paragraph, if it is opened. The \texcommand{\par\ShowPar} commands start new paragraph
after the inserted \htmlcommand{<div>} tag. It is necessary to explicitly start paragraphs sometimes.

\begin{texsource}
\Configure{@BODY}
{\ifvmode\IgnorePar\fi\EndP
 \HCode{<div>}\par\ShowPar}
\Configure{@/BODY}
{\ifvmode\IgnorePar\fi\EndP
 \HCode{</div>}}
\end{texsource}


\DocConfigure{HtmlPar} {content at the start non-indented paragraphs} 
   {content at the start indented paragraphs}
   {insertion into \cmd{EndP}, at the start of non-indented paragraphs}
   {insertion into \cmd{EndP}, at the start of indented paragraphs}\EndDoc

Example:

\begin{texsource}
\Configure{HtmlPar}
{\EndP\HCode{<p class="indent">}}
{\EndP\HCode{<p class="noindent">}}
{\HCode{</p>}}
{\HCode{</p>}}
\end{texsource}

% https://tex.stackexchange.com/a/66172/2891


\subsection{Logical Document Structure Commands}
I've created an alternative commands to \cmd{HCode} or \cmd{Tg}. 
The idea is to define semantic names for logical blocks of the document, such as titles, authors,
sections etc. HTML elements and attributes can be assigned to these
logical blocks. There are commands for inline and block level elements,
eliminating need for constructs like \texcommand{\ifvmode\IgnorePar\fi\EndP}
etc.

\DocCommand{NewLogicalBlock}\marg{name} create a new logical block.
\DocCommand{SetBlockProperty}\marg{name}\marg{attribute name}\marg{value} set block attribute .
\DocCommand{SetTag}\marg{name}{tag name} assign element name to the logical block
\DocCommand{BlockElementStart}\marg{name}\marg{additional attributes} start block level element.
\DocCommand{BlockElementEnd}\marg{name} close block level element.
\DocCommand{InlineElementStart}\marg{name}\marg{additional attributes} start inline level element.
\DocCommand{InlineElementEnd}\marg{name} close inline level element.


The default tag name for declared logical blocks is \htmlcommand{<span>}. You
need to use the \cmd{SetTag} command only when you want to use a different
element.

The attributes can be set either using \cmd{SetBlockProperty}, or in the second
argument to  \cmd{BlockElementStart} and \cmd{InlineElementStart} commands. First method should be
used for static attributes that don't change, for instance \textit{class}. The second method
is preferred for dynamic attributes, such as \textit{id}, which should be different for 
every element.

The main idea behind this mechanism is to allow easy work with new HTML5
elements and attributes for WAI-ARIA or Schema.org properties. I hope that
this should allow us to make somehow more clear configurations for basic
document building blocks.

Example:


\begin{texsource}
\NewLogicalBlock{textit}
\SetBlockProperty{textit}{class}{textit}

\NewLogicalBlock{maketitle}
\SetTag{maketitle}{header}

\NewLogicalBlock{titlehead}
\SetTag{titlehead}{h1}
\SetBlockProperty{titlehead}{class}{titleHead}

% configure \textit using inline level elements
\Configure{textit}
{\NoFonts\InlineElementStart{textit}{}}
{\InlineElementEnd{textit}\EndNoFonts}

% configure \maketitle using block level elements
\Configure{maketitle}{%
{\Configure{maketitle}{}{}{}{}%
\Tag{TITLE+}{\@title}}
\BlockElementStart{maketitle}{}}
{\BlockElementEnd{maketitle}}
{\NoFonts\BlockElementStart{titlehead}{}}
{\BlockElementEnd{titlehead}\EndNoFonts}
\end{texsource}






