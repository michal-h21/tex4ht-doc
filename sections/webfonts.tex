

The declared font family is not used automatically, it is necessary to select
it using the \term{font-family} Css property.

The default font family name which should be used in the Css
\term{font-family} command for a declared font is \term{rmfamily}. 
It use the Latin Modern font installed on the viewer's system. 
The Css font family and the local font name can be changed using
\verb|\Configure{FontFamily}{cssfamilyname}{LocalFontName}| command.

\begin{texsource}
\Configure{FontFamily}{rmfamily}{Latin Modern}
\end{texsource}

The font shapes can be configure using \verb|\Configure{NormalFont}|, 
\texcommand{\Configure}\allowbreak\texcommand{{ItalicFont}}, \verb|\Configure{BoldItalicFont}| and
\verb|Configure{BoldFont}|. The argument should be font file in the format
supported by browsers, such as \textit{woff} or \textit{otf}.


\begin{texsource}
\Configure{NormalFont}{normal-font-file.otf}
\Configure{BoldFont}{bold-font-file.otf}
\Configure{BoldItalicFont}{bold-italic-font-file.otf}
\Configure{ItalicFont}{italic-font-file.otf}
% Add another font family
\Configure{FontFamily}{hello}{Linux Libertine O}
\Configure{NormalFont}{hello-font-file.otf}
\Css{body{
  font-family: rmfamily, "AnotherFontFamilyName", serif;
}}
\Css{span.hello{font-family: hello, sans-serif;}}
\end{texsource}
