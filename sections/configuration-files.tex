\section{Private Configuration Files}\label{sec:private-configuration}

It is highly recommended to leave source LaTeX and TeX files intact, and not
introduce \texfourht{} configurations there. The configurations should be introduced
indirectly in private configuration files. Source files containing just native
LaTeX and TeX code permit their compilation to different output formats,
including PostScript and PDF, by \texfourht{} and other tools.

Packages used by the general LaTeX community typically provide better support
than one can expect from tailoring private commands and configurations for such
commands. It is also expected to take less effort to learn the features of
existing packages than designing new ones. Consequently, one is advised to
investigate available resources before committing to work on private features. 

\subsection{Requesting Private Configuration Files}

Private configuration files can be requested by passing the \shellcmd{-c}
option to \makefourht:

\begin{shellcommand}
make4ht -c mycfg.cfg myfile 
\end{shellcommand}


A configuration file should take the following form for \LaTeX\ files:

\begin{texsource}
...early definitions...
\Preamble{xhtml,options}
...definitions...
\begin{document}
...insertions into the header of the html file...
\EndPreamble
\end{texsource}

The \cmd{Preamble} command should always use \option{xhtml} 
as the first option. Otherwise, \texfourht\ switches to the HTML 4 mode and XML
post-processing tools used by \makefourht{} to fix some common issues will fail! For 
more information about available options, see the \nameref{sec:texfouhtoptions} 
section (\ref{sec:texfouhtoptions}).

\begin{texsource}
\Preamble{xhtml} 
\begin{document} 
  \Css{body { color : red; }} 
\EndPreamble 
\end{texsource}

\paragraph{Notes}

\begin{itemize}
  \item Notice that for a LaTeX file the \texcommand{\begin{document}}
    instruction should be present both in the configuration file and the source
    file.

  \item Instructions defined within a source file may be redefined in a
    configuration file. Such a feature enables to keep source files intact for
    compilation to different formats by different tools.

    For instance, a definition of the form \texcommand{\renewcommand\mycommand{...}} within a
    configuration file provided for the following \LaTeX\ source:
\end{itemize}


\begin{texsource}
\documentclass{...} 
\newcommand\mycommand{...} 
\begin{document} 
Use \mycommand{...} 
\end{document} 
\end{texsource}

\subsection{Configuration file management}

It is possible to reuse common \texfourht\ configurations used in several
configuration files.  They can be inserted in a custom LaTeX package, but there
is one important thing to be aware of. The configuration hooks are inserted to
the patched commands when the compilation reaches the  
\texcommand{\begin{document}} command, so configurations for these hooks
declared before the hook definition have no effect. It is necessary to include
them in the \cmd{AtBeginDocument} command.

Sample package, \file{commonconfigurations.sty}:

\begin{texsource}
\ProvidesPackage{commonconfigurations}
\AtBeginDocument{%
\Configure{@HEAD}
{\HCode{<meta name="test" content="test"/>\Hnewline}}
}
\endinput
\end{texsource}

It can be requested in a configuration file using \cmd{RequirePackage} command.

\begin{texsource}
\Preamble{xhtml}
\RequirePackage{commonconfigurations}
\begin{document}
\EndPreamble
\end{texsource}

\section{Custom output formats}

% ToDo: reuse content from: https://tex.stackexchange.com/a/477891/2891 and https://www.tug.org/applications/tex4ht/mn11.html#QQ1-11-66
