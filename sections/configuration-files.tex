\section{Private Configuration Files}\label{sec:private-configuration}

It is highly recommended to leave source LaTeX and TeX files intact, and not
introduce TeX4ht configurations there. The configurations should be introduced
indirectly in private configuration files. Source files containing just native
LaTeX and TeX code permit their compilation to different output formats,
including PostScript and PDF, by TeX4ht and other tools.

Packages used by the general LaTeX community typically provide better support
than one can expect from tailoring private commands and configurations for such
commands. It is also expected to take less effort to learn the features of
existing packages than designing new ones. Consequently, one is advised to
investigate available resources before committing to work on private features. 

\subsection{Requesting Private Configuration Files}

The leading entry, in the first list of options of the \shellcmd{htlatex}-like
commands, can equal \option{html} or \option{xhtml}. If this is not the case,
the entry is assumed to be the name of a configuration file. The extension
‘cfg’ is assumed for names of configuration files that are listed without their
extension.

A configuration file should take the following form for LaTeX files.

\begin{texsource}
...early definitions...
\Preamble{options}
...definitions...
\begin{document}
...insertions into the header of the html file...
\EndPreamble
\end{texsource}

It is up to the user to decide the distribution of entries between the
\texcommand{\Preamble} and the htlatex-like commands.

Example: The command \shellcmd{htlatex myfile "mycfg,2"} requests the
compilation of a file named \file{myfile.tex}, in the presence of a
configuration file named \file{mycfg.cfg}. The configuration file might have the
following content.

\begin{texsource}
\Preamble{html} 
\begin{document} 
  \Css{body { color : red; }} 
\EndPreamble 
\end{texsource}

Notes

\begin{itemize}
  \item Notice that for a LaTeX file the \texcommand{\begin{document}}
    instruction should be present both in the configuration file and the source
    file.

  \item Instructions defined within a source file may be redefined in a
    configuration file. Such a feature enables to keep source files intact for
    compilation to different formats by different tools.
\end{itemize}

For instance, a definition of the form \texcommand{\renewcommand\mycommand{...}} within a
configuration file provided for the following LaTeX source.

\begin{texsource}
\documentclass{...} 
\newcommand\mycommand{...} 
\begin{document} 
Use \mycommand{...} 
\end{document} 
\end{texsource}

\subsection{Configuration file management}

It is possible to reuse common \texfourht\ configurations used in several
configuration files.  They can be inserted in a custom LaTeX package, but there
is one important thing to be aware of. The configuration hooks are inserted to
the patched commands when the compilation reaches the  
\texcommand{\begin{document}} command, so configurations for these hooks
declared before the hook definition have no effect. It is necessary to include
them in the \texcommand{\AtBeginDocument} command.

Sample package, \file{commonconfigurations.sty}:

\begin{texsource}
\ProvidesPackage{commonconfigurations}
\AtBeginDocument{%
\Configure{@HEAD}
{\HCode{<meta name="test" content="test"/>\Hnewline}}
}
\endinput
\end{texsource}

It can be requested in a configuration file using \texcommand{\RequirePackage} command.

\begin{texsource}
\Preamble{xhtml}
\RequirePackage{commonconfigurations}
\begin{document}
\EndPreamble
\end{texsource}


