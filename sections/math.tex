\section{Math}
\subsection{Default math handling}

Default method for displaying math in the HTML output is a mixture of HTML elements and pictures. Pictures
are used for display math and elements that are hard to display using HTML and CSS only, like fractions or square roots.

The downside of this method is that pictures may not align properly to the text baseline in inline math. Furthermore, by default,
bitmap graphics is produced, which cannot be resized without losing of quality. To use vector SVG graphics instead, use the \option{svg} 
option. To prevent mixing of HTML, use the \option{pic-m} option. There are also similar options for requiring pictures for equations
and various math environments. See the chapter \namerefpage{chap:options}.


\subsection{MathML}

\mathml\ is a XML markup for the math encoding. 
It is supported in many aplications including OpenOffice Writer or Firefox web browser. 
In the HTML output, you can enable \mathml{} using the \option{mathml} option. In other output formats,
for example ODT, \mathml{} is used by default.

The advantage over use of images is that the rendered math fits style of the surrounding text,
and it can be resized without losing of quality.

Disadvantage is that it is not fully supported by some Web browsers fully, in particular
by Google Chrome and other browsers that are derived from Chrome's engine. Fortunatelly,
the support can be added using JavaScript. Popular option is to use MathJax:

\begin{texsource}
$ make4ht filename.tex "mathml,mathjax"
\end{texsource}

\paragraph{\mathml{} troupbleshooting}

The \mathml\ code produced by \texfourht\ may contain some issues. For example,
one common issue may happen when the math contain unmatched delimiters:

\begin{texsource}
 Mail address: $\lparen$hello@world.com$\rparen$
\end{texsource}

In such cases, the \option{matml-} may help. 

The \extension{common\_domfilters}
extension (see section \ref{sec:make4ht-extensions} for more
information about \term{make4ht} extensions), fixes some common \mathml\
errors that cannot be easily fixed on the \TeX\ level. 

\paragraph{Additional information}

If you want to add original \LaTeX\ source as an annotation to your MathML formulas,
you can try a configuration from \href{https://tex.stackexchange.com/a/637910/2891}{this answer}.

\subsection{MathJax}
\texfourht\ supports MathJax, library for math rendering in HTML documents. 
 It supports two modes -- \LaTeX\ math and \mathml.

The \term{MathJax} processing can be required using the \option{mathjax} option.

The address of the \term{MathJax} script and its configuration string can be
specified in the \configuration{MathjaxSource} configuration. Default value of this configuration is:

\begin{texsource}
\Configure{MathJaxSource}
{https://cdn.jsdelivr.net/npm/mathjax@3/es5/tex-chtml-full.js}
\end{texsource}

\paragraph{\LaTeX\ mode}

In the \LaTeX math mode, \TeX\ macros used in the math mode are preserved in
the output HTML document. They are parsed and rendered by MathJax when the
document is displayed by a web browser. The downside of this mode is that
commands unknown to MathJax must be configured in a special configuration for
MathJax. Cross-references to equations and other numbered math environments
don't work out of the box. You can try the Lua scripts proposed in 
\href{https://tex.stackexchange.com/a/597913/2891}{this post on TeX.sx} as
a workaround.

By default, inline and display math, as well as math environments, are kept as
raw LaTeX code in the \HTML\ output. 

The additional configuration for \term{MathJax} can be provided in the
\configuration{MathJaxConfig} configuration.
The following example provides support for some custom \LaTeX\ macros.

\begin{texsource}
\Preamble{xhtml}
\Configure{MathJaxConfig}{{
    tex: {
      tags: "ams",
      \detokenize{%
      macros: {
        sc: "\\small\\rm",
        sl: "\\it",
      }
  }
}
}}
\begin{document}
\EndPreamble

\end{texsource}


The configuration needs to be passed as a JavaScript object, this means that
you need to use extra \verb|{}| brackets.
The \texcommand{\detokenize} macro is used to avoid possible issues with catcodes
in the configuration. Backslashes must be doubled in the
JavaScript strings, so they need to be written twice for all custom macros.
Contents of this configuration is already enclosed in the
\texcommand{\HCode} command, so you cannot use this command in the configuration.

If you want to define macros with parameters, you may run to issues with the \# character, used for parameters.
You need to change the catcode of this character to letter before the \configuration{MathJaxConfig} configuration:

\begin{texsource}
\Preamble{xhtml}
% change catcode of # to letter
% in order to support #1 in custom
% MathJax macros
\catcode`\#=11
\Configure{MathJaxConfig}{{
    tex: {
      \detokenize{%
      macros: { 
        % our sample macro takes a parameter
        hello: ["\\sqrt{#1}",1],
      }
  }
}
}}
% don't forget to change the catcode 
% back to the original value
\catcode`\#=6
\begin{document}
\EndPreamble
\end{texsource}

Note that number of opening and closing brackets is important, they need to be balanced. 
For example, if you use \verb|\left\\{|, you need to add special closing bracket for \verb|\\{|. 
The problem is that this closing bracket can cause syntax error in the generated JavaScript, 
as the opening bracket was contents of the string. You can fix that by enclosing the closing bracket
in JavaScript comment string (\verb|/* } */|. In this way, \LaTeX\ will see the correct number of brackets,
but JavaScript will ignore the extra ones:

\begin{texsource}
\Preamble{xhtml}
\catcode`\#=11
\Configure{MathJaxConfig}{{
    tex: {
      macros: {
      leftbracket: ["\\left\\{#1\\right.",1], /*} */
    }
  }
}}
\catcode`\#=6
\begin{document}
\EndPreamble
\end{texsource}

\paragraph{Custom environments in MathJax.}\label{sec:mathjax_env}

If you want to add support for a new environment, you can use the \verb|\VerbMath| command, and you will
also need to pass suitable configuration to MathJax. For example for the following TeX file:

\begin{texsource}
\documentclass[12pt]{book}    
\usepackage{amsmath} 
\usepackage{breqn}

\begin{document}    
\begin{dgroup*}
\begin{dmath*}
     A = B
   \end{dmath*}     %error is around here           
\end{dgroup*}
\end{document}
\end{texsource}

Can be configured using this configuration file:

\begin{texsource}
\Preamble{xhtml} 
\Configure{MathJaxConfig}{{ 
    tex: { 
      tags: "ams", 
      \detokenize{% 
        environments: {
          "dgroup*": ["", ""],
          "dmath*": ["", ""],
        } 
      } 
    } 
}} 
\VerbMath{dgroup*}
\begin{document} 
\EndPreamble
\end{texsource}

The \verb|\VerbMath| command can be used just for the \verb|dgroup*| environment, as it will pass the
whole contents, including the \verb|dgroup*| environment, to the HTML file. But you will need to 
provide MathJax configuration for both of these environments, as they are not supported by default.



\paragraph{MathJax's \mathml\ mode}

Math is converted to \mathml\ by \texfourht, MathJax then renders it. Custom
commands and cross-references work in this mode.

The \mathml\ MathJax mode can be required using the \option{mathml,mathjax} option.

\paragraph{Table of contents issues}

Some math commands may cause issues when they are used in section titles in the MathJax mode. 
This can be fixed using the \texcommand{\fixmathjaxtoc} command:

\begin{texsource}
\fixmathjaxtoc\int
\end{texsource}

