\section{Math}
\subsection{Default math handling}

Default method for displaying math in the HTML output is a mixture of HTML elements and pictures. Pictures
are used for display math and elements that are hard to display using HTML and CSS only, like fractions or square roots.

The downside of this method is that pictures may not align properly to the text baseline in inline math. Furthermore, by default,
bitmap graphics is produced, which cannot be resized without losing of quality. To use vector SVG graphics instead, use the \option{svg} 
option. To prevent mixing of HTML, use the \option{pic-m} option. There are also similar options for requiring pictures for equations
and various math environments. See the chapter \namerefpage{chap:options}.


\subsection{MathML}

\mathml\ is a XML markup for the math encoding. 
It is supported in many aplications including OpenOffice Writer or Firefox web browser. 
In the HTML output, you can enable \mathml{} using the \option{mathml} option. In other output formats,
for example ODT, \mathml{} is used by default.

The advantage over use of images is that the rendered math fits style of the surrounding text,
and it can be resized without losing of quality.

Disadvantage is that it is not fully supported by some Web browsers fully, in particular
by Google Chrome and other browsers that are derived from Chrome's engine. Fortunatelly,
the support can be added using JavaScript. Popular option is to use MathJax:

\begin{texsource}
$ make4ht filename.tex "mathml,mathjax"
\end{texsource}

\paragraph{\mathml{} troupbleshooting}

The \mathml\ code produced by \texfourht\ may contain some issues. For example,
one common issue may happen when the math contain unmatched delimiters:

\begin{texsource}
 Mail address: $\lparen$hello@world.com$\rparen$
\end{texsource}

In such cases, the \option{matml-} may help. 

The \extension{common\_domfilters}
extension (see section \ref{sec:make4ht-extensions} for more
information about \term{make4ht} extensions), fixes some common \mathml\
errors that cannot be easily fixed on the \TeX\ level. 

\paragraph{Additional information}

If you want to add original \LaTeX\ source as an annotation to your MathML formulas,
you can try a configuration from \href{https://tex.stackexchange.com/a/693839/2891}{this answer}.

\subsection{MathJax}
\texfourht\ supports MathJax, library for math rendering in HTML documents. 
 It supports two modes -- \LaTeX\ math and \mathml.

The \term{MathJax} processing can be required using the \option{mathjax} option.

\DocConfigure{MathJaxSource} {MathJax URL}\EndDoc

The address of the \term{MathJax} script and its configuration string. Default value of this configuration is:

\begin{texsource}
\Configure{MathJaxSource}
{https://cdn.jsdelivr.net/npm/mathjax@3/es5/tex-chtml-full.js}
\end{texsource}

\subsection{MathJax's \LaTeX\ mode}



In the \LaTeX\ math mode, \TeX\ macros used in the math mode are preserved in
the output HTML document. 
By default, inline and display math, as well as math environments should work.
They are parsed and rendered by MathJax when the
document is displayed by a web browser.

The downside of this mode is that commands unknown to MathJax must be configured in a special configuration for
MathJax. See the how-to~\namerefpage{sec:howto-mathjax} for the examples 
on how to configure MathJax for custom commands, new environments, cross-references
and other issues.

\DocConfigure{MathJaxConfig} {JavaScript object}\EndDoc

See the \href{https://docs.mathjax.org/en/latest/options/index.html}{MathJax documentation}
for the possible configuration options. 



\DocConfigure{MathJaxMacros} {path to file with macro definitions}\EndDoc

Load file with declarations of custom commands, so they can be recognized 
by MathJax. Note that you need to escape characters that could cause issues
in the HTML post-processing, namely \verb|<|, \verb|>| and \verb|&|. Use 
HTML entities instead.

\DocCommand{VerbMath}\marg{environment name}

Redefine the given environment so it pass its contents verbatim to the output HTML file.
You may need to configure MathJax to recognize it.


\paragraph{Table of contents issues}

Some math commands may cause issues when they are used in section titles in the MathJax mode. 
This can be fixed using the \texcommand{\fixmathjaxtoc} command:

\begin{texsource}
\fixmathjaxtoc\int
\end{texsource}

\subsection{MathJax's \mathml{} mode}

Math is converted to \mathml\ by \texfourht, MathJax then renders it. Custom
commands and cross-references work in this mode.

The \mathml\ MathJax mode can be required using the \option{mathml,mathjax} option.


