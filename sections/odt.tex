The OpenDocument Format uses XML configuration file for document styling. To
declare new document style, \texfourht\ provides command
\texcommand{\NewConfigureOO}. The declared style then needs to be configured using command \texcommand{\ConfigureOO}.

Usage of these commands can be illustrated by the following example:

\begin{texsource}
  \Configure{SectionTitleTest}{\ifvmode\IgnorePar\fi\EndP\HCode{<text:p text:style-name="section-title">}}{\HCode{</text:p>}}

\NewConfigureOO{section-title}
\ConfigureOO{section-title}{<style:style style:name="section-title" style:family="paragraph" style:class="text">
      <style:text-properties style:text-underline-style="solid"
       style:text-underline-width="auto"
       style:text-underline-color="font-color"
       />
</style:style>}
\end{texsource}

Document style  \term{section-title} had been declared in this example. The
\xml\ code  for this style can be used without the \texcommand{\HCode} command
in \texcommand{\ConfigureOO}.

The configuration \term{SectionTitleTest} inserts element \verb|<text:p>|. The
\verb|text:style-name| corresponds to attribute \verb| style:name| of
\verb|style:style| element in the style configuration.
%Information about \texcommand{\NewConfigureOO} and styling and how to correctly use text styles (using configuration for HtmlPar)

%\url{https://tex.stackexchange.com/a/471283/2891}, \url{https://tex.stackexchange.com/a/100287/2891}

\subsection{Extra Configurations for OpenDocument Format}


To use default ODF styles for sectioning commands, use the following configurations:
% todo: explain better
\begin{texsource}
\Configure{Heading-2}{Heading 1}
\Configure{Heading-3}{Heading 2}
\end{texsource}

