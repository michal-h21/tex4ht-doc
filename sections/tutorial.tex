\hypertarget{what-is-tex4ht}{%
\subsection{\texorpdfstring{What is
\texttt{tex4ht}?}{What is tex4ht?}}\label{what-is-tex4ht}}

\href{https://www.tug.org/tex4ht/}{\texttt{tex4ht}} is a system which
converts LaTeX to various output formats, including \texttt{html},
\href{http://en.wikipedia.org/wiki/XHTML}{\texttt{xhtml}},
\href{http://en.wikipedia.org/wiki/OpenDocument}{\texttt{odt}},
\href{http://en.wikipedia.org/wiki/DocBook}{\texttt{docbook}} or
\href{http://en.wikipedia.org/wiki/Text_Encoding_Initiative}{\texttt{tei}}.
\texttt{html} and \texttt{odt} are the most common and best-supported
conversion targets.

\texttt{tex4ht} allows authors to use LaTeX input--widely employed for
high-quality typography, especially mathematical typography--to produce
output in other formats, especially \texttt{html} (for web pages) and
\texttt{xhtml} (for ebooks and other applications).

\hypertarget{system-description}{%
\subsection{System Description}\label{system-description}}

\texttt{tex4ht} consists of three basic building blocks and various
scripts which tie these blocks together.

\begin{enumerate}
\def\labelenumi{\arabic{enumi}.}
\item
  \texttt{tex4ht.sty} is a TeX package which inserts configured output
  codes (i.e., html tags) into TeX's \texttt{.dvi} output file. Many
  documents can be translated to html without users needing to supply
  tags explicitly, but there are macros to insert html directly into the
  output if the need arises.
\item
  \texttt{tex4ht} is an executable (program), which extracts information
  stored in the \texttt{.dvi} file including text and output codes, and
  prepares auxiliary files for image conversion and other tasks. Note
  that although whole system is named \texttt{tex4ht}, this command
  cannot be executed on \texttt{.tex} file; it works only with
  \texttt{.dvi} file
\item
  \texttt{t4ht} is a program which converts images, generates
  \texttt{css} file, and runs various commands requested in the
  \texttt{.tex} file
\end{enumerate}

A number of helper shell scripts (commands) exist, so that users do not
need to invoke these commands manually. The best known of these is
\texttt{htlatex}, which by default converts LaTeX to \texttt{html}.
Using different options, you can convert to any output format supported
by the \texttt{tex4ht} system.

In fact, you can convert to almost any format using \texttt{tex4ht},
even to formats not based on \texttt{xml}, but to do so involves
providing extensive configuration files.

\hypertarget{basic-usage}{%
\subsection{Basic usage}\label{basic-usage}}

The basic usage of the \texttt{htlatex} command (script) is as follows:

\begin{verbatim}
 htlatex filename "options for tex4ht.sty" "options for tex4ht" "options for t4ht" "LaTeX options"
\end{verbatim}

As you can see, \texttt{htlatex} has five parameters; only first one,
the filename, is mandatory. Also note that options must be generally be
enclosed in quotes so that they can be passed literally to the
underlying commands.

The calling command ``driver'' is \texttt{mk4ht}, which is similar to
\texttt{htlatex}, but slips in a new first parameter indicating the
system to be used. Values for this parameter include \texttt{htlatex}
which produces the same results as the htlatex script, \texttt{oolatex}
for Open Document Format conversion, \texttt{dblatex} for docbook, or
\texttt{teilatex} for TEI. The \texttt{mk4ht} command is quite general,
allowing user-generated configuration files. For further information,
see
\href{https://www.tug.org/applications/tex4ht/mn-commands.html\#QQ1-9-33}{calling
commands} on the \texttt{tex4ht} website.

As an example, to compile to Open Document Format, you would type this
at the commmand prompt:

\begin{verbatim}
mk4ht oolatex sample.tex
\end{verbatim}

A more recent option is to use Michal Hoftich's
\href{https://github.com/michal-h21/make4ht}{make4ht} build system for
\texttt{tex4ht}. It allows the user to call various commands during
compilation, such as \texttt{bibtex}, \texttt{biber}, or \texttt{xindy};
to postprocess output files with \texttt{Lua} scripts or commands such
as \texttt{tidy} or \texttt{xslt} processors; and to specify the command
to be used for image conversion.

In this tutorial, we will show usage of both \texttt{htlatex} and
\texttt{make4ht}.

\hypertarget{simple-example}{%
\subsection{Simple example}\label{simple-example}}

Lets start with conversion of simple LaTeX file to html. Let's say we
have following multilingual LaTeX file:

\begin{verbatim}
\documentclass{article}
\usepackage[english,czech]{babel}
\usepackage[T1]{fontenc}
\usepackage[utf8]{inputenc}
\begin{document}
Příliš žluťoučký kůň úpěl \textit{ďábelské} ódy.
\begin{otherlanguage}{english}
    Some text in English
\end{otherlanguage}
\end{document}
\end{verbatim}

things to notice are use of two languages, Czech is the main document
language, English is secondary. Note usage of \texttt{otherlanguage}
environment. It is provided by \texttt{babel} package and locally
switches document languages, so correct hyphenation and other language
dependent stuff are used. We could use
\texttt{\textbackslash{}selectlanguage} command, but I would like to
discourage usage of switching commands such is this one, or font
switching commands like \texttt{\textbackslash{}bfseries}, for one
reason: it is impossible to configure them correctly for end element
insertion. For font switching commands, situation is saved by
\texttt{tex4ht} command, which inserts formatting instructions for each
font change. But generally, such commands don't play nice with nature of
\texttt{xml} based formats, where every started element must be closed
on the same hierarchical level. So they must have same parent element.
Usage of \texttt{otherlanguage} environment will allow us to make proper
configuration and insert opening and closing tags at correct places.

But beware of following situation:

\begin{verbatim}
Hello world.
\begin{someenv}
Just start some environment.

But run it through several paragraphs
\end{someenv}
\end{verbatim}

say that we insert
\texttt{\textless{}div\ class="someenv"\textgreater{}} and
\texttt{\textless{}/div\textgreater{}} tags around \texttt{someenv}
environment. By default this may produce following structure:

\begin{verbatim}
<p>Hello world.
<div class="someenv">Just start some environment.
</p>

<p>But run it through several paragraphs
</div></p>
\end{verbatim}

as you can see, generated html code is incorrect, as opening and closing
\texttt{div} tags have different parent elements. \texttt{someenv} can
be configured to close current paragraph, but it may be not what you
want.

Best way to prevent tag mismatch may be something like:

\begin{verbatim}
Hello world.
\begin{someenv}
Just start some environment.
\end{someenv}

\begin{someenv}
But run it through several paragraphs
\end{someenv}
\end{verbatim}

But stop talking about traps you may fall into and lets compile our
example! For start use of both of \texttt{htlatex} and \texttt{make4ht}
will be showed, we will focus on \texttt{make4ht} later.

With \texttt{htlatex}, we may use

\begin{verbatim}
 htlatex sample1
\end{verbatim}

and with \texttt{make4ht}

\begin{verbatim}
 make4ht sample1
\end{verbatim}

lets look on text part generated by \texttt{htlatex}:

\begin{verbatim}
<!--l. 6--><p class="noindent" >P&#x0159;íli&#353; &#382;lu&#x0165;ou&#x010D;k&#x00FD; k&#x016F;&#x0148; úp&#x011B;l <span 
class="ecti-1000">&#x010F;</span><span 
class="ecti-1000">ábelsk</span><span 
class="ecti-1000">é </span>ódy. Some text in English
\end{verbatim}

and by \texttt{make4ht}:

\begin{verbatim}
<!--l. 6--><p class="noindent" >P&#x0159;íli&#353; &#382;lu&#x0165;ou&#x010D;k&#x00FD; k&#x016F;&#x0148; úp&#x011B;l <span 
class="ecti-1000">&#x010F;</span><span 
class="ecti-1000">ábelsk</span><span 
class="ecti-1000">é </span>ódy. Some text in English
</p> 
\end{verbatim}

only difference is missing \texttt{\textless{}/p\textgreater{}} tag in
output of \texttt{htlatex}, because \texttt{html\ 4.01} is produced by
\texttt{htlatex} by default. \texttt{make4ht} on the other hand produces
\texttt{xhtml} by default, so closing tag must be presented.

To get \texttt{xhtml} output from \texttt{htlatex}, use
\texttt{tex4ht.sty} option \texttt{xhtml}. This option must be first
option in the option list passed to \texttt{tex4ht.sty}. Value of the
first option must be either \texttt{html}, \texttt{xhtml} or name of
custom config file. We will cover these config files later, as they are
key component in customization of \texttt{tex4ht} output.

So in order to get same output as from \texttt{make4ht}, we must use
following command:

\begin{verbatim}
 htlatex sample1 xhtml
\end{verbatim}

Now we should get rid of ugly entities which encode accented letters.
This is somewhat ugly with \texttt{htlatex}:

\begin{verbatim}
 htlatex sample1 "xhtml,charset=utf-8" " -cunihtf -utf8"
\end{verbatim}

\texttt{charset=utf-8"} produces meta element which declares document to
be in \texttt{utf-8} encoding. Important are two options for
\texttt{tex4ht} command, \texttt{-c} and \texttt{-utf8}.

ToDo: add description of process of conversion from \texttt{htf} fonts
to utf8 using unicode.4hf. It is directed from \texttt{tex4ht.env} file.

With \texttt{make4ht}, situation is easier, as all we need to do is to
add \texttt{-u} option:

\begin{verbatim}
 make4ht -u sample1.tex
\end{verbatim}

resulting file:

\begin{verbatim}
<!--l. 6--><p class="noindent" >Příliš žluťoučký kůň úpěl <span 
class="ecti-1000">ď</span><span 
class="ecti-1000">ábelsk</span><span 
class="ecti-1000">é </span>ódy. Some text in English
</p> 
\end{verbatim}

Entities are gone, but other persists. What we see is caused by a bug in
\texttt{tex4ht} command. It decorates text which is set in non-default
font with \texttt{\textless{}span\textgreater{}} elements. Unfortunately
it doesn't play well with accented letters as we can see. This has easy
solution, fortunately. We just need to dive into \texttt{tex4ht}
configuration. Yay!

\hypertarget{configurations}{%
\subsection{Configurations}\label{configurations}}

We already saw that we can use command line options to configure the
output. For full list of options for \texttt{tex4ht.sty}, see an
\href{http://www.cvr.cc/?p=504}{article on CVR's blog}. These options
mainly influence appearance or math, footnotes, tables, etc. Note that
these options aren't fixed set, anyone can add new options and not all
options are supported in each output format supported by
\texttt{tex4ht}. Generally these options work with \texttt{html} (and
\texttt{xhtml}) output.

Other option is to use custom config file (\texttt{.cfg}). This is a TeX
file with some basic structure:

\begin{verbatim}
 optional stuff like requiring LaTeX packages etc
 ...
 \Preamble{xhtml,tex4ht.sty options}
 ...
 tex4ht configurations
 ...
 \begin{document} 
 ...
 more tex4ht configurations
 ...
 \EndPreamble
\end{verbatim}

Most important command for configuring is
\texttt{\textbackslash{}Configure}. This command has variable number of
arguments, in the simplest form it does have two arguments:
\texttt{\textbackslash{}Configure\{configname\}\{insert\ for\ a\ first\ hook\}}.

At this place we should talk about hooks. In order to insert html tags,
LaTeX macros are redefined and in the definitions special hooks are
inserted. These hooks are declared with
\texttt{\textbackslash{}NewConfigure\{configname\}\{number\ of\ hooks\}}
in special file named as redefined package name with suffix
\texttt{.4ht}. These hooks are then seeded in configure files for
particular output formats, or in the \texttt{.cfg} file.

To illustrate that, we can show some simple example. Lets say we have
simple package \texttt{hello.sty}:

\begin{verbatim}
\ProvidesPackage{hello} 
\newcommand\hello{\textbf{hello world}}
\endinput
\end{verbatim}

we can provide hooks in file named \texttt{hello.4ht}. Say we just want
to insert tags at beginning and at end of \texttt{\textbackslash{}hello}
command:

\begin{verbatim}
% provide configure for \hello command. we can choose any name
% but most convenient is to name hooks after redefined command
% we declare two hooks, to be inserted before and after the command
\NewConfigure{hello}{2}
% now we need to redefine \hello. save it to tmp command
\let\tmp:hello\hello
% note that `:` can be part of command name in `.4ht` files. 
% now insert the hooks. they are named as \a:hook, \b:hook, ..., \h:hook
% depending on how many hooks were declared
\renewcommand\hello{\a:hello\tmp:hello\b:hello} 
\end{verbatim}

because we want to surround contents produced by
\texttt{\textbackslash{}hello} with tags, we need to declare two hooks.
This is the most usual case for normal commands which just produce some
text. Old contents of macro are saved in temporary macro and then
command is redefined to insert hooks and original contents stored in
temporary macro.

Now we can change our sample to use \texttt{hello} package:

\begin{verbatim}
\documentclass{article}
\usepackage[english,czech]{babel} 
\usepackage[T1]{fontenc}
\usepackage[utf8]{inputenc} 
\usepackage{hello}
\begin{document} Příliš žluťoučký kůň úpěl \textit{ďábelské} ódy.
\begin{otherlanguage}{english} Some text in English, \hello
\end{otherlanguage} 
\end{document}
\end{verbatim}

we haven't provided any configurations for \texttt{hello} yet, but you
can see that text \texttt{hello\ world} is in \textbf{bold} font anyway.
This is the same case as \texttt{\textbackslash{}textit} which is
converted as \emph{italic}. Basic font styles are inserted by
\texttt{tex4ht} command during extraction of text from \texttt{dvi} to a
output format. So it is the right time to finally show how to configure
both \texttt{textit} and \texttt{hello} to produce some better tags than
they provide by default.

Basic structure of a config file has been shown before, so now we will
just add basic configurations for \texttt{\textbackslash{}textit} and
\texttt{\textbackslash{}hello}:

\begin{verbatim}
\Preamble{xhtml}
\Configure{textit}{\HCode{<span class="textit">}}{\HCode{</span>}}
\Configure{hello}{\HCode{<span class="hello">}}{\HCode{</span>}}
\Css{.textit{font-style:italic;}}
\Css{.hello{font-weight:bold;}}
\begin{document}
\EndPreamble
\end{verbatim}

For documentation of default configurations, see
\href{http://michal-h21.github.io/src4ht/tex4ht-info.html}{tex4ht info},
most useful are
\href{http://michal-h21.github.io/src4ht/tex4ht-infose2.html}{LaTeX} and
\href{http://michal-h21.github.io/src4ht/tex4ht-infose1.html}{tex4ht}
sections. Documentation for basic font commands such as
\texttt{\textbackslash{}textit} or \texttt{\textbackslash{}textbf} is
provided in
\href{http://michal-h21.github.io/src4ht/tex4ht-infose2.html}{LaTeX}
section. We can see that configuration takes two parameters, insertion
before and after content. Same situation is with \texttt{hello}
configuration we defined earlier, hooks are inserted before and after
the content.

To insert \texttt{html} tags, we need to use
\texttt{\textbackslash{}HCode} commands, special characters such as
\texttt{\textless{}},\texttt{\textgreater{}} or \texttt{\&} are escaped
otherwise. In our example we insert \texttt{span} elements with some
\texttt{class} attribute to distinguish them. Because these classes
doesn't have any visual appearance by default, we use
\texttt{\textbackslash{}Css} commands to add some styling. Yes, you need
to know both \texttt{html} and \texttt{css} to effectively configure
\texttt{tex4ht}!

If we look at \texttt{html} output now, we can see that things don't
look much better than initially:

\begin{verbatim}
<!--l. 6--><p class="noindent" >Příliš žluťoučký kůň úpěl <span class="textit"><span 
class="ecti-1000">ď</span><span 
class="ecti-1000">ábelsk</span><span 
class="ecti-1000">é</span></span> ódy. Some text in English, <span class="hello"><span 
class="ecbx-1000">hello world</span></span>
</p> 
\end{verbatim}

our new tags were inserted, but unnecessary elements inserted by
\texttt{tex4ht} processor are still present. Fortunately, we can
suppress insertion of these elements with
\texttt{\textbackslash{}NoFonts} command, and later enable again with
\texttt{\textbackslash{}EndNoFonts}. We can also use \texttt{tex4ht.sty}
option \texttt{NoFonts}, which will suppress font processing in whole
document, but you should use this with caution, as it may have some side
effects.

Let's take a look how would out configurations look with
\texttt{\textbackslash{}NoFonts} command:

\begin{verbatim}
\Preamble{xhtml}
\Configure{textit}{\HCode{<span class="textit">}\NoFonts}
{\EndNoFonts\HCode{</span>}}
\Configure{hello}{\HCode{<span class="hello">}\NoFonts}
{\EndNoFonts\HCode{</span>}}
\Css{.textit{font-style:italic;}}
\Css{.hello{font-weight:bold;}}
\begin{document}
\EndPreamble
\end{verbatim}

the output now looks much better:

\begin{verbatim}
<!--l. 6--><p class="noindent" >Příliš žluťoučký kůň úpěl <span class="textit">ďábelské</span> ódy. Some text in English, <span class="hello">hello world</span>
</p> 
\end{verbatim}

It may seems that we can be happy at this point, but things aren't as
easy as we may hope, because we haven't talked about one thing:

\hypertarget{paragraphs}{%
\subsubsection{Paragraphs}\label{paragraphs}}

What if we add some more paragraphs in English to our sample file?

\begin{verbatim}
\documentclass{article}
\usepackage[english,czech]{babel} 
\usepackage[T1]{fontenc}
\usepackage[utf8]{inputenc} 
\usepackage{hello}
\begin{document} Příliš žluťoučký kůň úpěl \textit{ďábelské} ódy.
\begin{otherlanguage}{english} Some text in English, \hello
\end{otherlanguage} 

\begin{otherlanguage}{english} 

\textit{What will do} \verb|\textit| at the beginning of paragraph?

And also, what about configuration for \verb|otherlanguage| environment?

\end{otherlanguage}

\end{document}
\end{verbatim}

What if we want to insert elements with \texttt{lang} attribute to
specify language of text in the \texttt{html}. It might be useful from
semantic point of view, we can also enable hyphenation in the
\texttt{css} and it works only when correct languages are marked in the
source.

This exercise will be little bit more difficult
