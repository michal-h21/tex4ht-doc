\section{What is \TeX4ht?}\label{what-is-tex4ht}

\href{https://www.tug.org/tex4ht/}{\texttt{TeX4ht}} is a system that
converts LaTeX to various output formats, including \texttt{HTML},
\href{http://en.wikipedia.org/wiki/OpenDocument}{\texttt{ODT}},
\href{http://en.wikipedia.org/wiki/DocBook}{\texttt{DocBook}} or
\href{http://en.wikipedia.org/wiki/Text_Encoding_Initiative}{\texttt{TEI}}.
\texttt{HTML} and \texttt{ODT} formats are the most common and best-supported
conversion targets.

\texttt{TeX4ht} allows authors to convert \LaTeX\ input to 
several output formats, like \texttt{HTML} (for web pages) or
\texttt{ePub} (for ebooks and other applications).

\hypertarget{basic-usage}{%
\section{Basic Usage}\label{sec:tutorial-basic-usage}}

Conversion is invoked using the \makefourht\ command:

\begin{shellcommand}
$ make4ht filename.tex
\end{shellcommand}


Let us start with conversion of a simple \LaTeX\ file to HTML  
with the following \LaTeX\ file:

\texinput{examples/otherlang/babel.tex}

You can compile it using the following command:

\begin{shellcommand}
$ make4ht -lm draft filename.tex
\end{shellcommand}

The resulting HTML file contains the following code:

\htmlinput{examples/otherlang/babel-lua-body.html}

As you can see, multiple options can be joined for \makefourht.  The above invocation is equal to the following:

\begin{shellcommand}
$ make4ht -l -m draft filename.tex
\end{shellcommand}

You can also use the long options:

\begin{shellcommand}
$ make4ht --lua --mode draft filename.tex
\end{shellcommand}

What do these options mean? 

The option \texttt{--lua} tells \makefourht\  to use Lua\LaTeX\ as a
compilation engine. There is also an option \texttt{-x} (or \texttt{--xetex})
that allows the use of Xe\LaTeX\ for the compilation. If neither of these options
is used, the file will be converted with the default PDF\LaTeX\ engine.

For example, if we compile the sample file without the \texttt{-l} option, we would get
a different result:

\htmlinput{examples/otherlang/babel-body.html}

Notice that the accents in the \textit{ť} and \textit{ď} letters are detached from the base letters.
That is because \texfourht{} uses information about characters in the DVI file. Current \LaTeX
supports basic accented characters out of the box, but sometimes, they don't work as expected.


The option \texttt{----mode} sets the compilation mode. \makefourht\  has one built-in mode,
named \texttt{draft}. 
By default, \makefourht{}  compiles your \TeX{} file three times, to obtain the correct hyperlinks 
and other features that depend on auxilary files. The \texttt{draft} mode uses
only one compilation run, so it is much faster.



\makefourht\  converts \LaTeX\ file to a HTML 5 document. You can request
conversion to other formats using the \texttt{-f} option. For example,
to convert a document to the OpenDocument Format, use the following:

\begin{shellcommand}
$ make4ht -f odt filename.tex
\end{shellcommand}

More information about \makefourht and its command line options and other features can be found in
section \namerefpage{sec:make4ht-intro}.


\section{\texfourht\ Options}

The simplest way to change some aspects of the design is to use \texfourht{} options. They can be passed
as a first positional argument after filename to \makefourht:

\begin{shellcommand}
$ make4ht filename.tex "option1,option2"
\end{shellcommand}

For example, \texfourht{} produces one HTML file for a document, but each footnote is placed in a separate file.
If you have a large document, you may want to use a separate page for each chapter, with a list of footnotes
at the end of these chapters. You can use the following options: 

\begin{shellcommand}
$ make4ht filename.tex "3,sec-filename,fn-in"
\end{shellcommand}


There are other numeric options, each of them breaks document into separate HTML pages on a different sectioning level. Option 1 does not break pages at all, 2 at parts, 3 at chapters, 4 at sections, 5 at subsections, 6 at sub-subsections, and 7 at paragraphs. The \verb|sec-filename| option will produce HTML file names that are based on section titles, instead of their numbers. The \verb|fn-in| option prints footnotes at the end of each HTML page.

There are also options that change the handling of math. Normally, HTML elements are used for simple math, and pictures are used for more complex features, such as fractions or square roots. This usually does not look good, so what are other options?

Generally, it is best to use \mathml{}, as it supports correct vertical alignment for inline math, and the font size matches the surrounding text. Unfortunately, some web browsers do not support it yet. We can use MathJax to render math in these browsers. 

\begin{shellcommand}
$ make4ht filename.tex "mathml,mathjax"
\end{shellcommand}

On the other hand, if you want to use pictures for math exclusively, you can try the \option{pic-m} option, which requires pictures even for inline math. There are also similar options for equations and other math environments.

\begin{shellcommand}
$ make4ht filename.tex "pic-m,pic-equation"
\end{shellcommand}

The generated pictures are in the PNG format, which is raster and depends on the resolution on the device where the document is displayed. You may want to use vector SVG format instead, as it should produce better quality of pictures:

\begin{shellcommand}
$ make4ht filename.tex "pic-m,pic-equation,svg"
\end{shellcommand}

For more information on options, see chapter \namerefpage{chap:options}.

\section{\makefourht{} extensions}

\makefourht{} has an extension support. These extensions can modify various aspects of the conversion process, for example, post-process the generated files, cache images, or add support for Rmarkdown files.
Extensions can be enabled using the \verb|-f format_name+extension_name| option. 

For example, there is a \verb|preprocess_input| extension, which adds support for Markdown or Rtex documents. It can process a following 
Rmarkdown document:

\texinput{examples/otherlang/rmarkdownsample.Rmd}

Compile it using the following command:

\begin{shellcommand}
$ make4ht -f html5+preprocess_input sample.Rmd
\end{shellcommand}

It producess a following HTML file:

\htmlinput{examples/otherlang/rmarkdownsample-body.html}

If your document produces many pictures, the compilation can take a long time. To make it faster, you can use the \verb|dvisvgm_hashes| extension. It caches the SVG images
and creates them only for the changed math environments.

\begin{shellcommand}
$ make4ht -f html5+dvisvgm_hashes filename.tex "pic-m,pic-equation,svg"
\end{shellcommand}

\makefourht{} also loads some extensions by default. They fix some issues in the generated HTML files using the LuaXML package. To disable this processing,
use \verb|-extension_name|:


\begin{shellcommand}
$ make4ht -f html5-common_domfilters filename.tex
\end{shellcommand}

You can find a list of extensions in \href{https://www.kodymirus.cz/make4ht/make4ht-doc.html#extensions}{\makefourht{} documentation}.

\section{Configurations}
\label{sec:tutorial-configurations}

You may want to insert some custom HTML tags. It is a bit more complicated for
\LaTeX commands, but it is easy for environments. You can configure the code that is 
inserted before and after environment using the \texcommand{\ConfigureEnv} command.
It has a following syntax:

\begin{texsource}
\ConfigureEnv{<environment name>}{before env}{after env}
{before-list}{after-list}
\end{texsource}

We can ignore the arguments \texttt{before-list} and \texttt{after-list}, as they
are used only for list like environments, such as \texttt{itemize}. So we just need to






But beware of the following situation:

\begin{texsource}
Hello world.
\begin{someenv}
Just start some environment.

But run it through several paragraphs
\end{someenv}
\end{texsource}

say that we insert
\htmlcommand{<div class="someenv">} and
\htmlcommand{</div>} tags around  the \texttt{someenv}
environment. By default this may produce following structure:

\begin{htmlsource}
<p>Hello world.
<div class="someenv">Just start some environment.
</p>

<p>But run it through several paragraphs
</div></p>
\end{htmlsource}

as you can see, generated html code is incorrect, as opening and closing
\htmlcommand{<div>} tags have different parent elements. \texttt{someenv} environment can
be configured to close current paragraph, but it may be not what you
want.

Best way to prevent tag mismatch may be something like:

\begin{texsource}
Hello world.
\begin{someenv}
Just start some environment.
\end{someenv}

\begin{someenv}
But run it through several paragraphs
\end{someenv}
\end{texsource}

and with \texttt{make4ht}

\begin{shellcommand}
make4ht sample1
\end{shellcommand}

lets look on text part generated by \texttt{htlatex}:

\begin{htmlsource}
<!--l. 6--><p class="noindent" >P&#x0159;íli&#353; &#382;lu&#x0165;ou&#x010D;k&#x00FD; k&#x016F;&#x0148; úp&#x011B;l <span 
class="ecti-1000">&#x010F;</span><span 
class="ecti-1000">ábelsk</span><span 
class="ecti-1000">é </span>ódy. Some text in English
\end{htmlsource}

and by \texttt{make4ht}:

\begin{htmlsource}
<!--l. 6--><p class="noindent" >P&#x0159;íli&#353; &#382;lu&#x0165;ou&#x010D;k&#x00FD; k&#x016F;&#x0148; úp&#x011B;l <span 
class="ecti-1000">&#x010F;</span><span 
class="ecti-1000">ábelsk</span><span 
class="ecti-1000">é </span>ódy. Some text in English
</p> 
\end{htmlsource}

only difference is missing \texttt{\textless{}/p\textgreater{}} tag in
output of \texttt{htlatex}, because \texttt{html\ 4.01} is produced by
\texttt{htlatex} by default. \texttt{make4ht} on the other hand produces
\texttt{xhtml} by default, so closing tag must be presented.

To get \texttt{xhtml} output from \texttt{htlatex}, use
\texttt{tex4ht.sty} option \texttt{xhtml}. This option must be first
option in the option list passed to \texttt{tex4ht.sty}. Value of the
first option must be either \texttt{html}, \texttt{xhtml} or name of
custom config file. We will cover these config files later, as they are
key component in customization of \texttt{TeX4ht} output.

So in order to get same output as from \texttt{make4ht}, we must use
following command:

\begin{shellcommand}
htlatex sample1 xhtml
\end{shellcommand}

Now we should get rid of ugly entities which encode accented letters.
This is somewhat ugly with \texttt{htlatex}:

\begin{shellcommand}
htlatex sample1 "xhtml,charset=utf-8" " -cunihtf -utf8"
\end{shellcommand}

\texttt{charset=utf-8"} produces meta element which declares document to
be in \texttt{utf-8} encoding. Important are two options for
\texttt{tex4ht} command, \texttt{-c} and \texttt{-utf8}.

ToDo: add description of process of conversion from \texttt{htf} fonts
to utf8 using unicode.4hf. It is directed from \texttt{tex4ht.env} file.

With \texttt{make4ht}, situation is easier, as all we need to do is to
add \texttt{-u} option:

\begin{shellcommand}
make4ht -u sample1.tex
\end{shellcommand}

resulting file:

\begin{htmlsource}
<!--l. 6--><p class="noindent" >Příliš žluťoučký kůň úpěl <span 
class="ecti-1000">ď</span><span 
class="ecti-1000">ábelsk</span><span 
class="ecti-1000">é </span>ódy. Some text in English
</p> 
\end{htmlsource}

Entities are gone, but other persists. What we see is caused by a bug in
\texttt{tex4ht} command. It decorates text which is set in non-default
font with \texttt{\textless{}span\textgreater{}} elements. Unfortunately
it doesn't play well with accented letters as we can see. This has easy
solution, fortunately. We just need to dive into \texttt{TeX4ht}
configuration. Yay!

\hypertarget{configurations}{%
\section{Configurations}\label{configurations}}

We already saw that we can use command line options to configure the
output. For full list of options for \texttt{tex4ht.sty}, see an
\href{http://www.cvr.cc/?p=504}{article on CVR's blog}. These options
mainly influence appearance or math, footnotes, tables, etc. Note that
these options aren't fixed set, anyone can add new options and not all
options are supported in each output format supported by
\texttt{tex4ht}. Generally these options work with \texttt{html} (and
\texttt{xhtml}) output.

Other option is to use custom config file (\texttt{.cfg}). This is a TeX
file with some basic structure:

\begin{texsource}
 optional stuff like requiring LaTeX packages etc
 ...
 \Preamble{xhtml,tex4ht.sty options}
 ...
 TeX4ht configurations
 ...
 \begin{document} 
 ...
 more TeX4ht configurations
 ...
 \EndPreamble
\end{texsource}

Most important command for configuring is
\texttt{\textbackslash{}Configure}. This command has variable number of
arguments, in the simplest form it does have two arguments:
\texttt{\textbackslash{}Configure\{configname\}\{insert\ for\ a\ first\ hook\}}.

At this place we should talk about hooks. In order to insert html tags,
LaTeX macros are redefined and in the definitions special hooks are
inserted. These hooks are declared with
\texttt{\textbackslash{}NewConfigure\{configname\}\{number\ of\ hooks\}}
in special file named as redefined package name with suffix
\texttt{.4ht}. These hooks are then seeded in configure files for
particular output formats, or in the \texttt{.cfg} file.

To illustrate that, we can show some simple example. Lets say we have
simple package \texttt{hello.sty}:

\begin{texsource}
\ProvidesPackage{hello} 
\newcommand\hello{\textbf{hello world}}
\endinput
\end{texsource}

we can provide hooks in file named \texttt{hello.4ht}. Say we just want
to insert tags at beginning and at end of \texttt{\textbackslash{}hello}
command:

\begin{texsource}
% provide configure for \hello command. we can choose any name
% but most convenient is to name hooks after redefined command
% we declare two hooks, to be inserted before and after the command
\NewConfigure{hello}{2}
% now we need to redefine \hello. save it to tmp command
\let\tmp:hello\hello
% note that `:` can be part of command name in `.4ht` files. 
% now insert the hooks. they are named as \a:hook, \b:hook, ..., \h:hook
% depending on how many hooks were declared
\renewcommand\hello{\a:hello\tmp:hello\b:hello} 
\end{texsource}

because we want to surround contents produced by
\texttt{\textbackslash{}hello} with tags, we need to declare two hooks.
This is the most usual case for normal commands which just produce some
text. Old contents of macro are saved in temporary macro and then
command is redefined to insert hooks and original contents stored in
temporary macro.

Now we can change our sample to use \texttt{hello} package:

\begin{texsource}
\documentclass{article}
\usepackage[english,czech]{babel} 
\usepackage[T1]{fontenc}
\usepackage[utf8]{inputenc} 
\usepackage{hello}
\begin{document} Příliš žluťoučký kůň úpěl \textit{ďábelské} ódy.
\begin{otherlanguage}{english} Some text in English, \hello
\end{otherlanguage} 
\end{document}
\end{texsource}

we haven't provided any configurations for \texttt{hello} yet, but you
can see that text \texttt{hello\ world} is in \textbf{bold} font anyway.
This is the same case as \texttt{\textbackslash{}textit} which is
converted as \emph{italic}. Basic font styles are inserted by
\texttt{tex4ht} command during extraction of text from \texttt{dvi} to a
output format. So it is the right time to finally show how to configure
both \texttt{textit} and \texttt{hello} to produce some better tags than
they provide by default.

Basic structure of a config file has been shown before, so now we will
just add basic configurations for \texttt{\textbackslash{}textit} and
\texttt{\textbackslash{}hello}:

\begin{texsource}
\Preamble{xhtml}
\Configure{textit}{\HCode{<span class="textit">}}{\HCode{</span>}}
\Configure{hello}{\HCode{<span class="hello">}}{\HCode{</span>}}
\Css{.textit{font-style:italic;}}
\Css{.hello{font-weight:bold;}}
\begin{document}
\EndPreamble
\end{texsource}

For documentation of default configurations, see
\href{http://michal-h21.github.io/src4ht/tex4ht-info.html}{TeX4ht info},
most useful are
\href{http://michal-h21.github.io/src4ht/tex4ht-infose2.html}{LaTeX} and
\href{http://michal-h21.github.io/src4ht/tex4ht-infose1.html}{TeX4ht}
sections. Documentation for basic font commands such as
\texttt{\textbackslash{}textit} or \texttt{\textbackslash{}textbf} is
provided in
\href{http://michal-h21.github.io/src4ht/tex4ht-infose2.html}{LaTeX}
section. We can see that configuration takes two parameters, insertion
before and after content. Same situation is with \texttt{hello}
configuration we defined earlier, hooks are inserted before and after
the content.

To insert \texttt{html} tags, we need to use
\texttt{\textbackslash{}HCode} commands, special characters such as
\texttt{\textless{}},\texttt{\textgreater{}} or \texttt{\&} are escaped
otherwise. In our example we insert \texttt{span} elements with some
\texttt{class} attribute to distinguish them. Because these classes
doesn't have any visual appearance by default, we use
\texttt{\textbackslash{}Css} commands to add some styling. Yes, you need
to know both \texttt{html} and \texttt{css} to effectively configure
\texttt{TeX4ht}!

If we look at \texttt{html} output now, we can see that things don't
look much better than initially:

\begin{htmlsource}
<!--l. 6--><p class="noindent" >Příliš žluťoučký kůň úpěl <span class="textit"><span 
class="ecti-1000">ď</span><span 
class="ecti-1000">ábelsk</span><span 
class="ecti-1000">é</span></span> ódy. Some text in English, <span class="hello"><span 
class="ecbx-1000">hello world</span></span>
</p> 
\end{htmlsource}

our new tags were inserted, but unnecessary elements inserted by
\texttt{tex4ht} processor are still present. Fortunately, we can
suppress insertion of these elements with
\texttt{\textbackslash{}NoFonts} command, and later enable again with
\texttt{\textbackslash{}EndNoFonts}. We can also use \texttt{tex4ht.sty}
option \texttt{NoFonts}, which will suppress font processing in whole
document, but you should use this with caution, as it may have some side
effects.

Let's take a look how would out configurations look with
\texttt{\textbackslash{}NoFonts} command:

\begin{texsource}
\Preamble{xhtml}
\Configure{textit}{\HCode{<span class="textit">}\NoFonts}
{\EndNoFonts\HCode{</span>}}
\Configure{hello}{\HCode{<span class="hello">}\NoFonts}
{\EndNoFonts\HCode{</span>}}
\Css{.textit{font-style:italic;}}
\Css{.hello{font-weight:bold;}}
\begin{document}
\EndPreamble
\end{texsource}

the output now looks much better:

\begin{htmlsource}
<!--l. 6--><p class="noindent" >Příliš žluťoučký kůň úpěl <span class="textit">ďábelské</span> ódy. Some text in English, <span class="hello">hello world</span>
</p> 
\end{htmlsource}

It may seems that we can be happy at this point, but things aren't as
easy as we may hope, because we haven't talked about one thing:

\hypertarget{paragraphs}{%
\section{Paragraphs}\label{paragraphs}}

What if we add some more paragraphs in English to our sample file?

\begin{texsource}
\documentclass{article}
\usepackage[english,czech]{babel} 
\usepackage[T1]{fontenc}
\usepackage[utf8]{inputenc} 
\usepackage{hello}
\begin{document} Příliš žluťoučký kůň úpěl \textit{ďábelské} ódy.
\begin{otherlanguage}{english} Some text in English, \hello
\end{otherlanguage} 

\begin{otherlanguage}{english} 

\textit{What will do} \verb|\textit| at the beginning of paragraph?

And also, what about configuration for \verb|otherlanguage| environment?

\end{otherlanguage}

\end{document}
\end{texsource}

What if we want to insert elements with \texttt{lang} attribute to
specify language of text in the \texttt{html}. It might be useful from
semantic point of view, we can also enable hyphenation in the
\texttt{css} and it works only when correct languages are marked in the
source.

This exercise will be little bit more difficult
