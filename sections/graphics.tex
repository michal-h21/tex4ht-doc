\subsection{Low level features}

The next command imports external pictures, and the two commands that follow
request pictorial representations for local content. The attributes, and the
replacement parameters with their enclosing rectangular brackets, are optional. 

\DocCommand{Picture}\oarg{replacement-for-textual-browser}\marg{file-name attributes}

This command references the specified pictorial file. 

\begin{texsource}
\Picture[picture description]{foo.png class="picture"}
\end{texsource}

\DocCommand{Picture+}\oarg{replacement-for-text-browsers}\marg{file-name attributes} content \cmd{EndPicture}
\DocCommand{Picture*}\oarg{replacement-for-text-browsers}\marg{file-name attributes} content \cmd{EndPicture}

These commands produce a picture for the provided content, store the outcome
within a file of the specified name, and create a reference to the picture
within the document. The starred variant should be used in the vertical mode, the variant with the plus sign
can be used inside paragraphs.

Typical usage of these commands is for support of diagrams in the \TeX{} code, such as TikZ or PSTricks.

The component ‘\texttt{[replacement-for-textual-browser]}’ and the file name can be
omitted. If no name is provided for the file, the system assigns a name of its
own.

\begin{texsource}
\Picture+{ align="right"}%
Text within a picture.
\EndPicture
\end{texsource}


\DocConfigure{Picture} {Extension name pictures generated by DVI conversion, stored in \cmd{PictExt}}\EndDoc

Default: 

\begin{texsource}
\Configure{Picture}{.png}
\end{texsource}

  The extension names of bitmap files of glyphs of htf fonts may be
  determined within a g-entry in the environment file \texttt{tex4ht.env}, or a
  g-flag of the \shellcmd{tex4ht} utility.

\DocConfigure{Picture-alt} {alt value for \cmd{Picture+}\marg{...}  and \cmd{Picture*}\marg{...}}\EndDoc


\DocConfigure{Picture+} {before the dvi picture code} {after the dvi picture code}\EndDoc
\DocConfigure{Picture*} {before the dvi picture code} {after the dvi picture code}\EndDoc

  Typically, the plus `+' variant is introduced as an inline
  contribution into paragraphs, and the star `*' variant as an
  independent block between paragraphs.

\DocConfigure{PictureAlt} {definitions before alt} {definitions after alt}\EndDoc
\DocConfigure{PictureAlt*+} {definitions before alt} {definitions after alt}\EndDoc
\DocConfigure{PictureAlt*+[]} {definitions before alt} {definitions after alt}\EndDoc

Apply to \cmd{Picture}, \cmd{Picture*+}, and \cmd{Picture*+[...]}


\DocConfigure{IMG}
{before file name}
{between file name and alt}
{close alt for  \cmd{Picture} without * or +}
{close alt for  \cmd{Picture} with * and +}
{right delimiter}\EndDoc

  Example:

\begin{texsource}
\Configure{IMG}
  {\ht:special{t4ht=<img src="}}
  {\ht:special{t4ht=" alt="}}
  {" }
  {\ht:special{t4ht=" }}
  {\ht:special{t4ht=/>}}
\end{texsource}

\DocCommand{NextPictureFile}\marg{filename} 

   Requests a file name for the next created picture.

\cmd{PictureFile}

   Records the filename of the most recent created picture.

\subsection{Configurations for the \package{Graphics} package bundle}

\DocConfigure{graphics}{before graphics}{after graphics}\EndDoc


Examples:

\begin{texsource}
\Configure{graphics}
   {\Picture+[PIC]{ class="graphics"}}
   {\EndPicture }
\end{texsource}


\DocConfigure{graphics*}
{extension name}
{insertion}\EndDoc


Allows to configure \texfourht{} for graphics files named in
the \cmd{includegraphics} macro, based on the type of the files.


An empty insertion for the second argument cancels previous requests for the
specified extension.

You can utilise the macros that contain information about the image, for example
\cmd{Gin@base} (file name), \cmd{Gin@ext} (extension), \cmd{Gin@req@width} (requested image width), \cmd{Gin@req@height} (requested image height),
\cmd{noBoundingBox} (defined iff bounding box is unknown)

    Example:

\begin{texsource}
\Configure{graphics*}
{jpg}
{\Picture[pict]{\csname Gin@base\endcsname.jpg}}

\Configure{graphics*}
{wmf}
{\Needs{"convert \csname Gin@base\endcsname.wmf
 \csname Gin@base\endcsname.gif"}%
 \Picture[pict]{\csname Gin@base\endcsname.gif
 width="\expandafter\the\csname Gin@req@width\endcsname"
 height="\expandafter\the\csname Gin@req@height\endcsname"}%
}

\Configure{graphics*}
{eps}
{\openin15=\csname Gin@base\endcsname\PictExt\relax
 \ifeof15 % test if the converted file already exists
 \Needs{"convert \csname Gin@base\endcsname.eps
 \csname Gin@base\endcsname\PictExt"}%
 \fi
 \closein15
 \Picture[pict]{\csname Gin@base\endcsname\PictExt}%
}
\end{texsource}

\subsection{PDF support}
\DocConfigure{PdfConvert}{}{}\EndDoc
\DocConfigure{Ghostscript} {name of the executable for GhostScript}\EndDoc

\subsection{TikZ }

Animations using Animate package: \url{https://tex.stackexchange.com/a/404600/2891}

Issues with drivers: \url{https://tex.stackexchange.com/a/471460/2891}.
\subsection{Pstricks}
