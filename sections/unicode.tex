\subsection{Unicode}
\label{sec:unicode}

Generally speaking, \texfourht\ supports \term{Unicode}, but there are some
issues to be aware of. Most complete support exists for Lua\LaTeX, thanks to
special Lua script which is automatically loaded during the compilation. No
additional packages are necessary.

PDF\LaTeX\ doesn't support nativelly, but it is possible to emulate it using the
\package{inputenc} and \package{fontenc} packages:

\begin{texsource}
\documentclass{article}
\usepackage[utf8]{inputenc}
\usepackage[T1]{fontenc}
\begin{document}
Unicode text
\end{document}
\end{texsource}

Xe\LaTeX\ is an Unicode format, similarly to Lua\LaTeX. The supporting
mechanism for \texfourht\ is different in this case and full Unicode range is
not supported out of the box. By default, only most Latin based characters are
supported. For other scripts, such as Greek or Cyrillic, two ways to enable
support exists. 

First option is to define new font family using \package{fontspec} \texcommand{\newfontfamily} with the \texttt{Script} option.

\begin{texsource}
\newfontfamily\greekfont{Linux Libertine O}[Script=Greek]
\end{texsource}


The second option is to declare load support for a script in the custom config
file using the \texcommand{\xeuniuseblock}:


\begin{texsource}
\xeuniuseblock{Greek}
\end{texsource}

The block names are based on \href{https://en.wikipedia.org/wiki/Unicode_block}{Unicode blocks}.

It is also possible to declare all characters in an Unicode range. The command
\texcommand{\xeuniregisterblockhex} takes two hexadecimal parameters with
Unicode range to be declared.

\begin{texsource}
\xeuniregisterblockhex{0100}{017F}
\end{texsource}

Individual character can be declared using the \texcommand{\xeuniregisterchar} command:

\begin{texsource}
\xeuniregisterchar{"1F00}
\end{texsource}

In contrast to \texcommand{\xeuniregisterblockhex}, it uses decimal numbers by
default, so it is necessary to use the \verb|"| character in front of
a hexadecimal number.

\begin{warning}
It is possible to run into issues because of the way how Xe\LaTeX\ Unicode
support works. Common problem is filename support, for example in included
graphics. In general, it is better to avoid such filenames. If it is not possible, try to use the \texcommand{\detokenize} command.
\begin{texsource}
  \includegraphics{\detokenize{háček.jpg}}
\end{texsource}
\end{warning}
