
\section{Writing basic support for a new package}
\begin{itemize}
  \item \url{https://tex.stackexchange.com/a/402283/2891}
\end{itemize}

\section{Two types of .4ht files}

\subsection{Inserting configurable hooks for packages}

\subsection{Configure the hooks in output format configuration files}

\section{How to add support for a package to the \texfourht\ literate sources}

To add a proper support for a new package, it is necessary to edit the 
\texfourht\ literate sources. The configurable hooks need to be placed in the \file{tex4ht-4ht.tex},
the configuration of these hooks must be added to the output format configuration files.
The most common output format is \HTML, which can be configured in \file{tex4ht-html4.tex}, or 
\file{tex4ht-html5.tex} if \HTMLV\ features are used. It is also necessary to update the
\file{mktex4ht-cnf.tex}.

\subsection{Example}

Given following package \file{sample.sty}:

\begin{texsource}
\ProvidesPackage{sample}
\newcommand\hello{world}
\endinput
\end{texsource}

This simple package defines command \texcommand{\hello}, which simply prints the word \enquote{hello} when used in a document.

Let's say that we want to insert some \HTML\ tags before and after the text content printed by the command.

Basic template for \file{tex4ht-4ht.tex}

% examples/basicpackage/sample.4ht
\begin{texsource}
\<sample.4ht\><<<
% sample.4ht (|version), generated from |jobname.tex
% Copyright 2017 TeX Users Group
|<TeX4ht license text|>
\NewConfigure{hello}{2}
\pend:def\hello{\a:hello}
\append:def\hello{\b:hello}
\Hinput{sample}
\endinput
>>> \AddFile{9}{sample}
\end{texsource}

Configuration for each package must follow this basic template. The \ProTeX\ system is used as system for literate programming.

The \texcommand{\<name\><<<code>>>} block defines new macro which can be then called using \texcommand{|<name|>}. The license text
is included in the example this way. In order to generate the \file{sample.4ht} file, we need to use \texttt{sample.4ht} as a name
in the code block and command \texcommand{\AddFile{9}{sample}} after the block definition\footnote{I have no idea what the number
in the first parameter means.}.

Each package configuration  must include \texcommand{\Hinput{packagename}}, in order to load the configurations for the package.

The command \texcommand{\NewConfigure{hello}{2}} declares new configuration \texttt{hello}, with two configurable hooks. 
These hooks are named  \texcommand{\a:hello} and \texcommand{\b:hello}. The hooks must be inserted into the 
\texcommand{\hello}, which can be easily done using the \texcommand{\pend:def} and \texcommand{\append:def} commands. These
commands can insert code  at the beginning, respective at the end of the redefined command.

The configuration for \HTML\ must be placed in the \file{tex4ht-html4.tex} file:


% examples/basicpackage/config.cfg
\begin{texsource}
\<configure html4 sample\><<<
\Configure{hello}{\HCode{<span class="hello">}}{\HCode{</span>}}
\Css{.hello{color:red;}}
>>>
\end{texsource}

The configuration for a package must be placed in \texcommand{\<configure html4 packagename\>} block.
% ToDo: write more info


The package name must be also included in \file{mktex4ht-cnf.tex}:

\begin{texsource}
\AddFile{9}{sample}
\end{texsource}

The \file{.4ht} files can be generated simply using the \shellcmd{make} command.

The following sample \TeX\ file:

% examples/basicpackage/hello.tex
\begin{texsource}
\documentclass{article}
\usepackage{sample}
\begin{document}
  \hello\ world.
\end{document}
\end{texsource}

Produces a following \HTML\ code:

\begin{htmlsource}
<!--l. 4--><p class="noindent" >
<span class="hello">world</span> world. 
</p> 
\end{htmlsource}


\section{Commands Usable in the \file{.4ht} files}

% \HLet - https://tex.stackexchange.com/a/471724/2891

% \append:def and variants

% \HRestore - I don't know what it does


\section{ProTeX}


The literate programming system used in the previous section is called ProTeX. We should discuss some main ideas behind this system.

\subsection{ProTeX}
% copied from
% https://www.slac.stanford.edu/comp/unix/package/tex/tex4ht/mn2.html - it
% seems like an older version of documentation which contains some information later ommited

Literate programming is a discipline that promotes the writing of programs the
way one explains them to human beings. ProTeX is a literate programming system
fully implemented in terms of TeX, and it is compatible with LaTeX and other
TeX-base systems. TeX4ht, and ProTeX itself, are examples of applications
written in ProTeX.


\begin{texsource}
\input ProTex.sty
\AlProTex{extension,<<<>>>,list,title,escape-character}
\<title\><<<
code fragment
>>>  
|<title|>
\OutputCode\<...\> 
\end{texsource}

Some explanation:

\begin{texsource}
\input ProTex.sty
\AlProTex{extension,<<<>>>,list,title,escape-character}
\end{texsource}

The escape-character stands for `, @, |, or ?. If omitted, it stands for \verb|\|. 

\begin{texsource}
\<title\><<<
code fragment
>>>

\end{texsource}

This structure provides names to code fragments (the fragments should not be too large in size).


\begin{texsource}
 |<title|>
 \end{texsource}

 This command acts as a place holder for the code segment associated to the title (\texttt{|} stands for the escape character). 

\begin{texsource}
   \OutputCode\<...\>
 \end{texsource}

This command creates a file for the code whose root node is specified.



