% \section{Introduction}

This chapter deals with \texfourht\ development. It starts with a basic
tutorial for a new package support, shows commands useful in the process,
different types of \texfourht\ configuration files, and the syntax and structure of 
literate source files.

\section{Tutorial: Basic Support For a New Package}

In this tutorial, we will try to show how to provide \texfourht\ support for a
simple \LaTeX\ package. 

% from https://tex.stackexchange.com/a/402283/2891
\texfourht\ tries to load a special \file{.4ht} file for each package loaded
by \LaTeX. This special file can contain modifications to commands provided by the package, like 
redefinitions of macros that cause clashes between the package and \texfourht, and most importantly
they insert special macros, called hooks, that are then used to include the output format tags.

Let's say that you have a custom package, called \file{mynote.sty}

\begin{texsource}
\newcommand\notetitle{Note:~}
\newcommand\note[1]{\textbf{\notetitle}#1}
\newcommand\highlight[1]{\textbf{#1}}
\endinput
\end{texsource}

It defines two user commands, \cmd{note} and \cmd{highlight}. 
They can be used in the following way:


\begin{texsource}
\documentclass{article}
\usepackage{mynote}
\begin{document}
\note{This is a note}

Try to highlight \highlight{something}.
\end{document}
\end{texsource}

\texfourht\ produces usable output for both of these commands out of the box, 
thanks to the support for \TeX\ fonts. But you may want to use custom HTML 
tags instead. To achieve that, you need to insert special commands, called hooks 
in \texfourht, to package commands. These hooks can be then configured to
insert tags in the output format.

To introduce hooks, you need to create a hook seeding configuration file for the package,
called \file{<name>.4ht}. For example, to seed hooks for the \file{mynote.sty} package, create file
\file{mynote.4ht}:

\begin{texsource}
\NewConfigure{note}{3}

% Use \HLet when you want to completely redefine a command
\def\:tempa#1{\a:note\notetitle\b:note~#1\c:note}
\HLet\note\:tempa

\NewConfigure{highlight}{2}
\pend:defI\highlight{\a:highlight}
\append:defI\highlight{\b:highlight}

\Hinput{mynote}
\endinput
\end{texsource}

There is several things to note. First is that the \verb|:| character 
can formt part of a command name in \file{.4ht} files. It is similar
to use of the \verb|@| character in \LaTeX\ packages. It allows us to 
create command names that don't clash with other command names.

The hooks are created using the \cmd{NewConfigure} command. They can be
later filled with the \cmd{Configure} command. To have an effect, hooks
must be inserted to the existing commands. There are two ways how to do that.
For simpler commands, where we want to insert tags only before and after 
the contents produced by the patched command, we can use the \cmd{pend:def<X>} and 
\cmd{append:def<X>} commands, where the \verb|<X>| is a roman number of parameters
that the patched command expects. In this example, it expects one parameter, 
so we can use the \cmd{pend:defI} command. For commands without parameters, use 
\cmd{pend:def}.

Of course, you can also insert hooks using other mechanisms, for example using
\LaTeX's hook system:

\begin{texsource}
\AddToHook{cmd/highlight/before}{\a:highlight}
\AddToHook{cmd/highlight/after}{\b:highlight}
\end{texsource}

The second way for hook insertion, useful for commands where we want to insert
tags also inside it's contents, we can use the \cmd{HLet} command. It is a
variant of the \cmd{let} command.  In contrast to \cmd{let}, it saves the
original command as \cmd{o:<command name>:}.  Commands redefined by \cmd{HLet}
also support the \cmd{Picture} command, where the original version of the
command will be used. This way, pictures will produce the same result as they
would produce in the PDF mode.

In our example, we redefined the \cmd{note} command to use a hook between note title
and note text. This enables us to style both the title and the text differently.


The configuration file for our hooks could look like this:

\begin{texsource}
\Preamble{xhtml}
\Configure{note}
{\ifvmode\IgnorePar\fi\EndP\HCode{<div class="note"><span class="notetitle">}}
{\HCode{</span><span class="notebody">}}
{\HCode{</span></div>}}
\Css{.notetitle{font-weight: bold;}}

\Configure{highlight}{\HCode{<span class="highlight">}\NoFonts}{\EndNoFonts\HCode{</span>}}
\Css{.highlight{font-weight:bold;}}
\begin{document}
\EndPreamble
\end{texsource}

As the \cmd{note} command should be used on it's own paragraph, we need to 
fix paragraph closing. See the \namerefpage{sec:paragraph_handling} section for
more information about this issue. More details about configuration files and configurations are
in section \namerefpage{sec:private-configuration}.

The HTML code produced by our configuration looks like this:

\begin{htmlsource}
<div class='note'><span class='notetitle'>Note: </span><span class='notebody'> This is a note</span></div>
<!--  l. 6  --><p class='indent'>   Try to highlight <span class='highlight'>something</span>.
</p>
\end{htmlsource}


\section{Commands Usable in the \file{.4ht} files}

\DocCommand{NewConfigure}\marg{name}\marg{number of defined hooks}

This command defines macros with an alphabetic prefix in the form of 
\cmd{a:name} \ldots \cmd{i:name}, depending on the number of defined hooks.
The maximum number is 9.

\begin{texsource}
\NewConfigure{try}{2}
\def\try#1{\a:try#1\b:try}
\Configure{try}{* }{}  
\try{ho} 
% produces "* ho"
\end{texsource}

\DocCommand{NewConfigure}\marg{name}\oarg{number or parameters}\marg{code}

Variant of \cmd{NewConfigure} that doesn't define hooks with 
alphabetic prefixes, but it passes argumens of \cmd{Configure}
as \TeX\ arguments. See this exampe:

\begin{texsource}
\NewConfigure{try}[2]{\def\hookI{#1}\def\hookII{#2}}
\def\try#1{\hookI#1\hookII}
\Configure{try}{* }{}  
\try{ho} 
% produces "* ho"
\end{texsource}

When you use \texcommand{\Configure{try}}, it defines \cmd{hookI} and \cmd{hookII}
commands. They can be then used in the redefined \cmd{try} command.

\DocCommand{HLet}\marg{Redefined command name}\marg{new command}

Variant of \cmd{let} that saves the original command under \cmd{\o:<name>:} name.
It can detect use of the redefined command inside picture. In such case, it will use
the original command to produce correct visual result in the picture.

\begin{texsource}
\def\:tempa#1{\a:note note:\b:note~#1\c:note}
\HLet\note\:tempa
\Configure{note}{*}{*}{*}
\note{hello}
% produces: "* note:* hello*
\end{texsource}

\DocCommand{HRestore}\marg{command name}

Restore command redefined using \cmd{HLet} to it's original content.

\DocCommand{pend:def<X>}\marg{redefined command}\marg{code to be inserted at the begin}

\DocCommand{append:def<X>}\marg{redefined command}\marg{code to be inserted at the end}




% \HLet - https://tex.stackexchange.com/a/471724/2891
% - \HLet supports the Picture mechanism - it calls the original command inside \Picture*
%   so we don't get content addded by TeX4ht in pictures. This is important, because we
%   want this content to look in the same way as in the PDF

% \append:def and variants

% \HRestore - I don't know what it does

\section{Two types of .4ht files}

% text from the old documentation:
% https://tug.org/tex4ht/doc/mn11.html#QQ1-11-66

The compilation starts by opening tex4ht.sty and loading a fraction of its code.
The main purpose of this phase is to request the loading of the system at a
later time (for instance, upon reaching \texcommand{\begin{document}}). The motivation for
the late loading is to allow TeX4ht to collect as much information as possible
about the environment requested by the source file, and help the system reshape
that environment with minimal interference from elsewhere.

The system uses two kinds of (4ht) configuration files. The files of the first
kind mainly seed hooks into the macros loaded by the source file (for instance,
\file{latex.4ht}, \file{fontmath.4ht}, and \file{article.4ht}).
The files of the second kind mainly
attach meaning to the hooks (for instance, \file{html4.4ht}, \file{unicode.4ht}, and
\file{mathml.4ht}).

Different source files may request the loading of different style files and in
different orders. The hook seeding files are loaded in response to the loading
of the style files, and in a compatible order. Since the different style files
may redefine the syntax and semantics of macros, \texfourht\ follows a similar route
of defining and redefining the hooks and their meanings.

The meaning attaching files are normally requested through option names
introduced in the \file{tex4ht.4ht} system file. It defines options for all output formats
supported by \texfourht. For instance, \option{html5}, \option{ooffice} for the ODT output,
\option{tei}, and so on.

% For instance, the mzlatex command
% refers to the mozilla option name of tex4ht.4ht, and the oolatex command refers
% to the ooffice option name. 

The user may add option names, and redefine old
ones, within a new file named tex4ht.usr.

% \subsection{Inserting configurable hooks for packages}



% \subsection{Configure the hooks in output format configuration files}

\section{\texfourht\ literate sources}

To add a proper support for a new package, it is necessary to edit the 
\texfourht\ literate sources. 
The source files are available in the \href{https://puszcza.gnu.org.ua/projects/tex4ht/}{\texfourht\ source repository}.
You can retrieve them using a SVN client. 

\begin{shellcommand}
$ svn checkout https://svn.gnu.org.ua/sources/tex4ht/
$ cd tex4ht/trunk/lit/
\end{shellcommand}


The configurable hooks for all packages are contained by the \file{tex4ht-4ht.tex} file.
Configurations of these hooks is placed in the output format configuration files.
The most common output format is \HTML, which can be configured in \file{tex4ht-html4.tex}, or 
\file{tex4ht-html5.tex} if \HTMLV\ features are used. You can also update sources for other output
formats, for example \file{tex4th-ooffice.tex} for the ODT format, or \file{tex4ht-tei.tex} for TEI.
The sources of the \file{tex4ht.sty} package are available in \file{tex4ht-sty.tex}.

To compile all literate sources, run the \shellcmd{make} command. You will need basic UNIX utilities 
for this to succeed, as well as \shellcmd{m4} and \shellcmd{javac}. You can also compile particular source
files. Most of them can be compiled using \LaTeX, but some of them, for example \file{tex4ht-4ht.tex}, needs
to be compiled using \shellcmd{etex}.

\subsection{How to add support for a package to the \texfourht\ literate sources}

Given following package \file{sample.sty}:

\begin{texsource}
\ProvidesPackage{sample}
\newcommand\hello{hello}
\endinput
\end{texsource}

This simple package defines command \texcommand{\hello}, which simply prints the word \enquote{hello} when used in a document.

Let's say that we want to insert some \HTML\ tags before and after the text content printed by the command.

Basic template for \file{tex4ht-4ht.tex}:

% examples/basicpackage/sample.4ht
\begin{texsource}
\<sample.4ht\><<<
% sample.4ht (|version), generated from |jobname.tex
% Copyright 2017 TeX Users Group
|<TeX4ht license text|>
\NewConfigure{hello}{2}
\pend:def\hello{\a:hello}
\append:def\hello{\b:hello}
\Hinput{sample}
\endinput
>>> \AddFile{9}{sample}
\end{texsource}

Configuration for each package must follow this basic template. The \ProTeX\ system is used as system for literate programming.

The \verb|\<name\><<<code>>>| block defines new macro which can be then called using \texcommand{|<name|>}. The license text
is included in this way in the example. The instruction to generate the \file{.4ht} file is given in the 
command \texcommand{\AddFile{9}{sample}} after the block definition. The first argument to \cmd{AddFile} is an arbitrary number.


Each package configuration  must include \texcommand{\Hinput{packagename}}, in order to load the configurations for the package.

The command \texcommand{\NewConfigure{hello}{2}} declares new configuration \texttt{hello}, with two configurable hooks. 
These hooks are named  \texcommand{\a:hello} and \texcommand{\b:hello}. The hooks must be inserted into the 
\texcommand{\hello}, which can be easily done using the \texcommand{\pend:def} and \texcommand{\append:def} commands. These
commands can insert code  at the beginning, respective at the end of the redefined command.

The package name must be also included in the \file{mktex4ht-cnf.tex} file. This file is used in the generation of the 

\begin{texsource}
\AddFile{9}{sample}
\end{texsource}

You can place configuration for \HTML\ to the \file{tex4ht-html4.tex} file:

% examples/basicpackage/config.cfg
\begin{texsource}
\<configure html4 sample\><<<
\Configure{hello}{\HCode{<span class="hello">}}{\HCode{</span>}}
\Css{.hello{color:red;}}
>>>
\end{texsource}

The \texcommand{\<configure html4 packagename\>} block will produce code that 
detects use of the package \file{packagename}. It then loads configurations
for the package.


The \file{.4ht} files can be generated simply using the \shellcmd{make} command.

The following sample \TeX\ file:

% examples/basicpackage/hello.tex
\begin{texsource}
\documentclass{article}
\usepackage{sample}
\begin{document}
  \hello\ world.
\end{document}
\end{texsource}

Produces a following \HTML\ code:

\begin{htmlsource}
<!--l. 4--><p class="noindent" >
<span class="hello">hello</span> world. 
</p> 
\end{htmlsource}




\section{ProTeX}


The literate programming system used in the previous section is called ProTeX. We should discuss some main ideas behind this system.

% copied from
% https://www.slac.stanford.edu/comp/unix/package/tex/tex4ht/mn2.html - it
% seems like an older version of documentation which contains some information later ommited

Literate programming is a discipline that promotes the writing of programs the
way one explains them to human beings. ProTeX is a literate programming system
fully implemented in terms of TeX, and it is compatible with LaTeX and other
TeX-base systems. TeX4ht, and ProTeX itself, are examples of applications
written in ProTeX.


\begin{texsource}
\input ProTex.sty
\AlProTex{extension,<<<>>>,list,title,escape-character}
\<title\><<<
code fragment
>>>  
|<title|>
\OutputCode\<...\> 
\end{texsource}

Some explanation:

\begin{texsource}
\input ProTex.sty
\AlProTex{extension,<<<>>>,list,title,escape-character}
\end{texsource}

The escape-character stands for `, @, |, or ?. If omitted, it stands for \verb'|'. 

\begin{texsource}
\<title\><<<
code fragment
>>>

\end{texsource}

This structure provides names to code fragments (the fragments should not be too large in size).


\begin{texsource}
 |<title|>
 \end{texsource}

 This command acts as a place holder for the code segment associated to the title (\texttt{|} stands for the escape character). 

\begin{texsource}
   \OutputCode\<...\>
 \end{texsource}

This command creates a file for the code whose root node is specified.



