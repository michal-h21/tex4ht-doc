

\section{Tutorial: Basic Support For a New Package}

% from https://tex.stackexchange.com/a/402283/2891
\texfourht\ tries to load a special \file{.4ht} file for each package loaded
by \LaTeX. This special file can contain modifications to commands provided by the package, like 
redefinitions of macros that cause clashes between the package and \texfourht, and most importantly
they insert special macros, called hooks, that are then used to include the output format tags.

Let's say that you have a custom package, called \file{mynote.sty}

\begin{texsource}
\ProvidesPackage{mynote}
\RequirePackage{xcolor}
\definecolor{notecolor}{rgb}{1,0.2,0.2}
\newcommand\notetitle{Note:~}
\newcommand\note[1]{{\textcolor{notecolor}{\textbf{\notetitle}}}#1}
\endinput
\end{texsource}



\section{Two types of .4ht files}

% text from the old documentation:
% https://tug.org/tex4ht/doc/mn11.html#QQ1-11-66

A compilation starts by opening tex4ht.sty and loading a fraction of its code.
The main purpose of this phase is to request the loading of the system at a
later time (for instance, upon reaching \texcommand{\begin{document}}). The motivation for
the late loading is to allow TeX4ht to collect as much information as possible
about the environment requested by the source file, and help the system reshape
that environment with minimal interference from elsewhere.

The system uses two kinds of (4ht) configuration files. The files of the first
kind mainly seed hooks into the macros loaded by the source file (for instance,
latex.4ht, fontmath.4ht, and article.4ht). The files of the second kind mainly
attach meaning to the hooks (for instance, html4.4ht, unicode.4ht, and
mathml.4ht).

Different source files may request the loading of different style files and in
different orders. The hook seeding files are loaded in response to the loading
of the style files, and in a compatible order. Since the different style files
may redefine the syntax and semantics of macros, TeX4t follows a similar route
of defining and redefining the hooks and their meanings.

The meaning attaching files are normally requested through option names
introduced in the tex4ht.4ht system file. For instance, the mzlatex command
refers to the mozilla option name of tex4ht.4ht, and the oolatex command refers
to the ooffice option name. The user may add option names, and redefine old
ones, within a new file named tex4ht.usr.

\subsection{Inserting configurable hooks for packages}

\subsection{Configure the hooks in output format configuration files}

\section{How to add support for a package to the \texfourht\ literate sources}

To add a proper support for a new package, it is necessary to edit the 
\texfourht\ literate sources. The configurable hooks need to be placed in the \file{tex4ht-4ht.tex},
the configuration of these hooks must be added to the output format configuration files.
The most common output format is \HTML, which can be configured in \file{tex4ht-html4.tex}, or 
\file{tex4ht-html5.tex} if \HTMLV\ features are used. It is also necessary to update the
\file{mktex4ht-cnf.tex}.

\subsection{Example}

Given following package \file{sample.sty}:

\begin{texsource}
\ProvidesPackage{sample}
\newcommand\hello{hello}
\endinput
\end{texsource}

This simple package defines command \texcommand{\hello}, which simply prints the word \enquote{hello} when used in a document.

Let's say that we want to insert some \HTML\ tags before and after the text content printed by the command.

Basic template for \file{tex4ht-4ht.tex}

% examples/basicpackage/sample.4ht
\begin{texsource}
\<sample.4ht\><<<
% sample.4ht (|version), generated from |jobname.tex
% Copyright 2017 TeX Users Group
|<TeX4ht license text|>
\NewConfigure{hello}{2}
\pend:def\hello{\a:hello}
\append:def\hello{\b:hello}
\Hinput{sample}
\endinput
>>> \AddFile{9}{sample}
\end{texsource}

Configuration for each package must follow this basic template. The \ProTeX\ system is used as system for literate programming.

The \texcommand{\<name\><<<code>>>} block defines new macro which can be then called using \texcommand{|<name|>}. The license text
is included in the example this way. In order to generate the \file{sample.4ht} file, we need to use \texttt{sample.4ht} as a name
in the code block and command \texcommand{\AddFile{9}{sample}} after the block definition\footnote{I have no idea what the number
in the first parameter means.}.

Each package configuration  must include \texcommand{\Hinput{packagename}}, in order to load the configurations for the package.

The command \texcommand{\NewConfigure{hello}{2}} declares new configuration \texttt{hello}, with two configurable hooks. 
These hooks are named  \texcommand{\a:hello} and \texcommand{\b:hello}. The hooks must be inserted into the 
\texcommand{\hello}, which can be easily done using the \texcommand{\pend:def} and \texcommand{\append:def} commands. These
commands can insert code  at the beginning, respective at the end of the redefined command.

The configuration for \HTML\ must be placed in the \file{tex4ht-html4.tex} file:


% examples/basicpackage/config.cfg
\begin{texsource}
\<configure html4 sample\><<<
\Configure{hello}{\HCode{<span class="hello">}}{\HCode{</span>}}
\Css{.hello{color:red;}}
>>>
\end{texsource}

The configuration for a package must be placed in \texcommand{\<configure html4 packagename\>} block.
% ToDo: write more info


The package name must be also included in \file{mktex4ht-cnf.tex}:

\begin{texsource}
\AddFile{9}{sample}
\end{texsource}

The \file{.4ht} files can be generated simply using the \shellcmd{make} command.

The following sample \TeX\ file:

% examples/basicpackage/hello.tex
\begin{texsource}
\documentclass{article}
\usepackage{sample}
\begin{document}
  \hello\ world.
\end{document}
\end{texsource}

Produces a following \HTML\ code:

\begin{htmlsource}
<!--l. 4--><p class="noindent" >
<span class="hello">hello</span> world. 
</p> 
\end{htmlsource}


\section{Commands Usable in the \file{.4ht} files}

% \HLet - https://tex.stackexchange.com/a/471724/2891

% \append:def and variants

% \HRestore - I don't know what it does


\section{ProTeX}


The literate programming system used in the previous section is called ProTeX. We should discuss some main ideas behind this system.

\subsection{ProTeX}
% copied from
% https://www.slac.stanford.edu/comp/unix/package/tex/tex4ht/mn2.html - it
% seems like an older version of documentation which contains some information later ommited

Literate programming is a discipline that promotes the writing of programs the
way one explains them to human beings. ProTeX is a literate programming system
fully implemented in terms of TeX, and it is compatible with LaTeX and other
TeX-base systems. TeX4ht, and ProTeX itself, are examples of applications
written in ProTeX.


\begin{texsource}
\input ProTex.sty
\AlProTex{extension,<<<>>>,list,title,escape-character}
\<title\><<<
code fragment
>>>  
|<title|>
\OutputCode\<...\> 
\end{texsource}

Some explanation:

\begin{texsource}
\input ProTex.sty
\AlProTex{extension,<<<>>>,list,title,escape-character}
\end{texsource}

The escape-character stands for `, @, |, or ?. If omitted, it stands for \verb|\|. 

\begin{texsource}
\<title\><<<
code fragment
>>>

\end{texsource}

This structure provides names to code fragments (the fragments should not be too large in size).


\begin{texsource}
 |<title|>
 \end{texsource}

 This command acts as a place holder for the code segment associated to the title (\texttt{|} stands for the escape character). 

\begin{texsource}
   \OutputCode\<...\>
 \end{texsource}

This command creates a file for the code whose root node is specified.



