
\section{General Advice}

In general, don't postpone your conversion to the last moment. As it is quite
common to produce ebook or HTML versions of \LaTeX\ documents these days, it is 
best to prepare this process as soon as possible.

It is much easier to fix errors as soon as they appear. Often, one issue can
cause many other issues, and it can be really hard to find the culprit. 

It is a good idea to compile your document using \texfourht\ every time you add
a new package to your document preamble, as some packages may clash. You may also find
that you will need a different definitions for some commands for the HTML conversion, as the 
version suitable for PDF produce some artifacts from drawing commands. These issues are 
also best to fix as soon as possible.

\section{Compilation Errors}
\label{faq:compilation_errors}


\begin{issue}{Packages that cause issues}
\texfourht\ can clash with other packages sometimes. If you don't need features
the problematic package provides (typically, they modify stuff that makes sense in 
the PDF output, but not in HTML, like page headers, fancy section titles, etc.),
it is often easiest to disable their loading when \texfourht\ is active. 

There are multiple ways how to conditionally suppress package loading, but the easiest 
way is to just add the following to your document preamble:

\begin{texsource}
\ifdefined\HCode
... code for TeX4ht
\else
\usepackage{foo}
\fi
\end{texsource}
\end{issue}

\begin{issue}{Fragile commands}
Some commands may cause compilation errors when used in \texcommand{\section} or
\texcommand{\caption} arguments. These errors usually show when the document preamble
stops and auxilary files are loaded. They can be quite hard to debug.

Consider the following example:

\begin{texsource}
\begin{figure}[h]
\includegraphics{example-image}
\caption{$\begin{array}{c c} hello & world\end{array}$}
\end{figure}
\end{texsource}

It produces a fatal error:

\begin{shellcommand}
[ERROR]   htlatex: Compilation errors in the htlatex run
[ERROR]   htlatex: Filename     Line    Message
[ERROR]   htlatex: ./minimal.tex        7        Argument of \im:g has an extra }.
[ERROR]   htlatex: ./minimal.tex        7        Paragraph ended before \im:g was complete.
[ERROR]   htlatex: ?    ?        Incomplete \iffalse; all text was ignored after line 7.
[ERROR]   htlatex: ?    ?        Emergency stop.
[FATAL]   make4ht-lib: Fatal error. Command htlatex returned exit code 1
\end{shellcommand}

You can get more context for this error using the \verb|-a debug| option for \term{make4ht}:

\begin{shellcommand}
! Argument of \im:g has an extra }.
<inserted text> 
                \par 
l.7 ...egin{array}{c c} hello & world\end{array}}
\end{shellcommand}

The error itself is pretty cryptic, but you can see that the cause is in 

\begin{texsource}
\caption{$\begin{array}{c c} hello & world\end{array}$}
\end{texsource}

The issue is that some commands needs to be protected when used in parameters of \texcommand{\section}
or \texcommand{\caption} commands. There are two possible fixes. First is to use the \texcommand{\protect}
command:

\begin{texsource}
\caption{$\protect\begin{array}{c c} hello & world\protect\end{array}$}
\end{texsource}

Other fix is to use the \term{writetoc} configuration:

\begin{texsource}
\Configure{writetoc}{%
\def\begin{\detokenize{\begin}}
\def\end{\detokenize{\end}}
}
\end{texsource}

In this configuration, we redefine \texcommand{\begin} and \texcommand{\end} commands to print
their verbatim contents when they are used. They will be then written to the auxilary file
thanks to this configuration.
Redefinitions that happen in \term{writetoc} are active only when stuff is written to the auxilary
files, so you don't need to worry that it will break normal handling of environments.

\end{issue}

\section{DOM Processing Errors}
\label{faq:dom_processing}

\texfourht\ post-process generated HTML and XML files using LuaXML. 

\begin{issue}{Incorrect handling of paragraphs}

\end{issue}

\section{Math Issues}
\subsection{Problems With \term{\mathml}}

\begin{issue}{Single delimiters}
  Use of single delimiters like \texcommand{$\langle$} may result in invalid
  \mathml\ code. \texfourht\ can try to fix that using the \option{mathml-}
  option.
\end{issue}



