
\section{General Advice}

In general, don't postpone your conversion to the last moment. As you will 
most likely want to produce ebook or HTML versions of \LaTeX\ documents these days, 
you should try the conversion of your document from the beginning. 

It is much easier to fix errors as soon as they appear. Often, one issue can
cause many issues that follows, and it can be really hard to find the culprit. 

Compile your document using \texfourht\ every time you add
a new package, as the included code can break other packages or even \LaTeX\ kernel. 
You may also find that you will need a different definitions for some commands for the HTML conversion, as the 
version suitable for PDF produce some artifacts from drawing commands. These issues are 
also best to fix as soon as possible.

\section{Compilation Errors}
\label{faq:compilation_errors}


\begin{issue}{Packages that cause issues}
\texfourht\ can sometimes clash with other packages. If you don't need features
the problematic package provides (typically, they modify stuff that makes sense in 
the PDF output, but not in HTML, like page headers, fancy section titles, etc.),
an easiest fix is to disable their loading when \texfourht\ is active. 

The easiest way to disable package from loading is to add the following
condition to your document preamble:

\begin{texsource}
\ifdefined\HCode
... code for TeX4ht
\else
\usepackage{foo}
\fi
\end{texsource}
\end{issue}

\begin{issue}{Fragile commands}
Some commands may cause compilation errors when used in \texcommand{\section} or
\texcommand{\caption} arguments. These errors usually show when the document preamble
stops and auxilary files are loaded. They can be quite hard to debug.

Consider the following example:

\begin{texsource}
\begin{figure}[h]
\includegraphics{example-image}
\caption{$\begin{array}{c c} hello & world\end{array}$}
\end{figure}
\end{texsource}

It produces a fatal error:

\begin{shellcommand}
[ERROR]   htlatex: Compilation errors in the htlatex run
[ERROR]   htlatex: Filename     Line    Message
[ERROR]   htlatex: ./minimal.tex        7        Argument of \im:g has an extra }.
[ERROR]   htlatex: ./minimal.tex        7        Paragraph ended before \im:g was complete.
[ERROR]   htlatex: ?    ?        Incomplete \iffalse; all text was ignored after line 7.
[ERROR]   htlatex: ?    ?        Emergency stop.
[FATAL]   make4ht-lib: Fatal error. Command htlatex returned exit code 1
\end{shellcommand}

You can get more context for this error using the \verb|-a debug| option for \makefourht:

\begin{shellcommand}
! Argument of \im:g has an extra }.
<inserted text> 
                \par 
l.7 ...egin{array}{c c} hello & world\end{array}}
\end{shellcommand}

The error itself is pretty cryptic, but you can see that the cause is in 

\begin{texsource}
\caption{$\begin{array}{c c} hello & world\end{array}$}
\end{texsource}

The issue is that some commands needs to be protected when used in parameters of \texcommand{\section}
or \texcommand{\caption} commands. There are two possible fixes. First is to use the \texcommand{\protect}
command:

\begin{texsource}
\caption{$\protect\begin{array}{c c} hello & world\protect\end{array}$}
\end{texsource}

Other fix is to use the \term{writetoc} configuration:

\begin{texsource}
\Configure{writetoc}{%
\def\begin{\detokenize{\begin}}
\def\end{\detokenize{\end}}
}
\end{texsource}

In this configuration, we redefine \texcommand{\begin} and \texcommand{\end} commands to print
their verbatim contents when they are used. They will be then written to the auxilary file
thanks to this configuration.
Redefinitions that happen in \term{writetoc} are active only when stuff is written to the auxilary
files, so you don't need to worry that it will break normal handling of environments.

Last tip for dealing with errors that involve auxilary files is to delete temporary files if the 
error is fatal. In that case the temporary file will not be overwritten with a fixed version, 
resulting in fatal error even if you fixed the original issue. It is recommended to remove at least 
the \file{.aux} and \file{.xref} files, but you may need to remove also \file{.4tc} and \file{.4ct} files.
To simplify this process, you can use the \texttt{clean} mode of \makefourht:

\begin{shellcommand}
$ make4ht -m clean filename.tex
\end{shellcommand}

It will remove all temporary files, but also generated HTML files and images.

\end{issue}

\section{DOM Processing Errors}
\label{faq:dom_processing}

\texfourht\ post-process generated HTML and XML files using Lua filters. They fixes various issues that
are not possible fix on the \TeX\ level, and also add some additional features. For example, the \mathml\ 
output relies on filters to produce valid code.

These filters use the LuaXML library for the processing. When the generated HTML or XML file is not well-formed,
you may get the following warning:


\begin{shellcommand}
[WARNING] domfilter: DOM parsing of test.html failed:
[WARNING] domfilter: /home/michal/texmf/scripts/lua/LuaXML/
luaxml-mod-xml.lua:175: Unbalanced Tag (/p) [char=1353]
\end{shellcommand}

This means that DOM parsing failed and filters cannot be applied. This will result in incorrect file.

\begin{issue}{Incorrect handling of paragraphs}
Common cause of this issue is incorect handling of paragraphs in commands that produce block level
elements such as \htmlcommand{<div>} or \htmlcommand{<section>} elements.

Say that you have a following code in your document:

\begin{texsource}
\newenvironment{test}{\itshape}{}

\begin{test}
world

more paragraphs
\end{test}
\end{texsource}

And that you configure the \term{test} environment to to produce a custom \htmlcommand{<div>} element:

\begin{texsource}
\ConfigureEnv{test}{\HCode{<div class="test">}}{\HCode{</div>}}{}{}
\end{texsource}

This configuration will break the LuaXML parsing, as it produces invalid HTML code:

\begin{htmlsource}
<!--l. 27--><p class="indent" >   <div class="test"> <span 
class="cmti-12">world</span>
</p><!--l. 30--><p class="indent" >   <span 
class="cmti-12">more paragraphs</span>
</div>
</p>
\end{htmlsource}

The issue is that \htmlcommand{<p>} elements are mismatched. It is necessary that closing \htmlcommand{</p>} tag
is inserted before \htmlcommand{<div>} tag. You can use commands introduced in section \nameref{sec:paragraph_handling} 
(\ref{sec:paragraph_handling}) for this task. The correct configuration can look like this:

\begin{texsource}
\ConfigureEnv{test}{\ifvmode\IgnorePar\fi\EndP\HCode{<div class="test">}\par}
{\ifvmode\IgnorePar\fi\EndP\HCode{</div>}\par}{}{}
\end{texsource}

The \texcommand{\EndP} inserts closing \htmlcommand{</p>} tag. But new paragraph still could be started before 
\htmlcommand{<div>} if we are in the vertical mode, so we need to use also the \texcommand{\ifvmode\IgnorePar\fi}
command. This will prevent starting of the new paragraph too early. But we want to start a new paragraph after 
\htmlcommand{<div>}, so we need to use explicit \texcommand{\par} after \texcommand{\HCode}. Similar paragraph 
treatment is needed also for the closing of the environment. The resulting HTML with this configuration is correct:

\begin{htmlsource}
<div class='test'>
<!-- l. 28 --><p class='indent'>   <span class='cmti-12'>world</span>
</p><!-- l. 30 --><p class='indent'>   <span class='cmti-12'>more paragraphs</span>
</p></div>
\end{htmlsource}


\end{issue}



\section{Math Issues}
\subsection{Problems With \term{\mathml}}

\begin{issue}{Single delimiters}
  Use of single delimiters like \texcommand{$\langle$} may result in invalid
  \mathml\ code. \texfourht\ can try to fix that using the \option{mathml-}
  option.
\end{issue}

\begin{issue}{Use of \cmd{hbox} in math environments}

Don't use the \cmd{hbox} command inside math environments. The following code results in 
DOM error and wrong rendering by MathML processors. You can fix this by using \cmd{mbox} instead.

\begin{texsource}
\begin{equation*}
  \begin{array}{ll}
      \frac{k}{2^{n}}, & \hbox{ if \(\frac{k-1}{2^{n}}\leq\tau < \frac{k}{2^{n}}\).}
  \end{array}
\end{equation*}
\end{texsource}
  
\end{issue}

