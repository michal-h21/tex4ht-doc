\section{Page Breaks}

\texfourht{} can break the generated document to separate files, each containing 
contents of sectioning unit. The ‘1’, ‘2’, \ldots ,‘8’ options asks for a
tree-structured set of files, reflecting on the sectioning of the document to
the specified depth. 

Sequential prev-next links within the hierarchy, instead
of the default hierarchical ones, can be requested with the \option{next} option.
The option \option{sections+} creates titles for the sectioning commands that link
to the tables of contents. The \option{sec-filename} option requests that split  files will
be named according to the section titles, instead of sequential numbers. 
The \option{cut-fullname} option will prevent truncating of generated filenames.

Finer control is possible with the following commands and configurations:


\DocCommand{CutAt}\marg{at-unit,until-unit-1,until-unit-2,\ldots}

This directive asks the sectioning commands \cmd{at-unit} to place their units in
separate hypertext pages. The pages are to terminate upon encountering any of
the commands in the list \cmd{at-unit}, \cmd{until-unit-1}, \cmd{until-unit-2},\ldots.

Within the \cmd{CutAt} instruction, the starred commands of \LaTeX{} should be
referenced with the prefix ‘\textit{like}’ instead of the postfix of ‘*’, and appendices
through the entry ‘appendix’.

A plus character ‘+’, before the leading parameter, requests buttons that link
to the hypertext pages; e.g., ‘\cmd{CutAt}\marg{+likesection}’. 


The end points of sections not specified within the \cmd{CutAt} commands can be made
known with instructions of the form \cmd{Configure}\marg{endunit}\marg{unit,unit,...}. 

\begin{texsource}
\CutAt{section,chapter}
\tableofcontents
\chapter{...}  ......  \section{...}  ......  \section{...}  ......
\chapter{...}  ......  \section{...}  ......
\end{texsource}

Cut points at arbitrary points can be introduced by introducing section-like
commands in a manner similar to

\begin{texsource}
\NewSection\mysection{}
\CutAt{mysection}
\end{texsource}

\DocConfigure{CutAt} {unit} {before-button} {after-button}\EndDoc

\DocConfigure{+CutAt}
{sectioning type}
{before}
{after}\EndDoc

Requests delimiters for the \cmd{CutAt} buttons of the specified
     sectioning type

Example:   

\begin{texsource}
\Configure{+CutAt}{mysection}{[}{]}
\end{texsource}

\DocCommand{PauseCutAt}\marg{section type}
\DocCommand{ContCutAt}\marg{section type}

Pause and continue page breaking for the given section type. Note that it can be used for multiple section types. 

For example, if you use the command line option `4', with the \package{book}
class, which breaks pages up to the subsection level, and you want to stop
breaking sections, you need to use:

\begin{texsource}
\PauseCutAt{section}
\PauseCutAt{subsection}
\end{texsource}

To resume breaking, use:

\begin{texsource}
\ContCutAt{section}
\ContCutAt{subsection}
\end{texsource}

\DocConfigure{CutAt-filename} {filename for the next cut file}\EndDoc

   A 2-parameter hook for tailoring section-based filenames.
   The section type is available through \#1. The section title
   is accessible through \#2.

Example:

\begin{texsource}
\Configure{CutAt-filename}{\NextFile{#1-#2.html}}
\end{texsource}



