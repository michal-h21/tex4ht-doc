
\makefourht\ supports Makeindex, Xindy and Xindex indexing processors. 
You need to use a build file that calls them explicitly. The intermediate
file with indexing information uses special format with \texfourht, so 
it is not possible to just call these commands from the command line.
\makefourht\ uses some pre-processing and post-processing to make the indexing
work correctly.

Here is a simple example:

\begin{texsource}
\documentclass{article}
\usepackage{makeidx}
\makeindex
\begin{document}
Hello\index{hello} world\index{world}
\printindex
\end{document}
\end{texsource}

This file can be compiled using the following build file, \file{build.lua}:

\begin{luasource}
Make:htlatex {}
Make:makeindex {}
Make:htlatex{}
\end{luasource}

You can use \verb|Make:xindex| or \verb|Make:xindy| in place of \verb|Make:makeindex|.

As there are no page numbers in XML and HTML, each \cmd{index} command produces unique
number, and destination to the backlink from the index.


For multiple indices, you can multiple calls to indexing commands. For example:

\begin{texsource}
\documentclass{book}
\usepackage{imakeidx}
\makeindex
\makeindex[name=foo,title=Test]

\begin{document}
Test1\index{Test1}
Test2\index[foo]{Test2}
\newpage
Test3\index{Test3}
Test4\index[foo]{Test4}.
And Test1\index{Test1} again.

\printindex
\printindex[foo]
\end{document}
\end{texsource}

This example contains two indices. The \verb|foo| index needs an extra 
indexing call:

\begin{luasource}
Make:htlatex {}
Make:xindy {}
Make:xindy {idxfile = "foo.idx"}
Make:htlatex {}
\end{luasource}

