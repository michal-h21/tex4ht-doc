\texfourht\ supports Plain \TeX, but it can be a bit tricky. In general, you
will have to provide configurations to get good markup for your custom
commands. You will also need to add few commands to your source file: 

\begin{texsource}
%!TEX TS-program = tex
\input plain-4ht.tex
\document
Hello world
\enddocument
\end{texsource}

The first line contains magic comment that tells \texttt{make4ht} to use Plain
\TeX\ instead of \LaTeX for the compilation. \texcommand{\document} and \texcommand{\enddocument}
commands are important, because they are crucial for correct execution of \texfourht.

They are declared in the \texttt{plain-4ht.tex} file:

\begin{texsource}
\def\documentstyle#1{}
\documentstyle{tex4ht}
\csname tex4ht\endcsname
\def\document{}
\def\enddocument{\csname bye\endcsname}
\end{texsource}

In the PDF mode, these commands will do nothing, except for execution of the \texcommand{\bye} command.

The document can be compiled using:

\begin{shellcommand}
make4ht -f html5+detect_engine filename.tex
\end{shellcommand}

The \verb|-f html5+detect_engine| requires the \verb|detect_engine| extension,
which uses the magic command in the source file to determine which \TeX\ engine
to use. It is necessary for the correct Plain \TeX\ support.
