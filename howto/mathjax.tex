The other method of 
\begin{shellcommand}
$ make4ht foo.tex "mathjax"
\end{shellcommand}

Cross-references to equations should work other numbered math environments
don't work out of the box. You can try the Lua scripts proposed in 
\href{https://tex.stackexchange.com/a/597913/2891}{this post on TeX.sx} as
a workaround.

The additional configuration for \term{MathJax} can be provided in the
\configuration{MathJaxConfig} configuration.
The following example provides support for some custom \LaTeX\ macros.

\begin{texsource}
\Preamble{xhtml}
\Configure{MathJaxConfig}{{
    tex: {
      tags: "ams",
      \detokenize{%
      macros: {
        sc: "\\small\\rm",
        sl: "\\it",
      }
  }
}
}}
\begin{document}
\EndPreamble

\end{texsource}


The configuration needs to be passed as a JavaScript object, this means that
you need to use extra \verb|{}| brackets.
The \texcommand{\detokenize} macro is used to avoid possible issues with catcodes
in the configuration. Backslashes must be doubled in the
JavaScript strings, so they need to be written twice for all custom macros.
Contents of this configuration is already enclosed in the
\texcommand{\HCode} command, so you cannot use this command in the configuration.

If you want to define macros with parameters, you may run to issues with the \# character, used for parameters.
You need to change the catcode of this character to letter before the \configuration{MathJaxConfig} configuration:

\begin{texsource}
\Preamble{xhtml}
% change catcode of # to letter
% in order to support #1 in custom
% MathJax macros
\catcode`\#=11
\Configure{MathJaxConfig}{{
    tex: {
      \detokenize{%
      macros: { 
        % our sample macro takes a parameter
        hello: ["\\sqrt{#1}",1],
      }
  }
}
}}
% don't forget to change the catcode 
% back to the original value
\catcode`\#=6
\begin{document}
\EndPreamble
\end{texsource}

Note that number of opening and closing brackets is important, they need to be balanced. 
For example, if you use \verb|\left\\{|, you need to add special closing bracket for \verb|\\{|. 
The problem is that this closing bracket can cause syntax error in the generated JavaScript, 
as the opening bracket was contents of the string. You can fix that by enclosing the closing bracket
in JavaScript comment string (\verb|/* } */|. In this way, \LaTeX\ will see the correct number of brackets,
but JavaScript will ignore the extra ones:

\begin{texsource}
\Preamble{xhtml}
\catcode`\#=11
\Configure{MathJaxConfig}{{
    tex: {
      macros: {
      leftbracket: ["\\left\\{#1\\right.",1], /*} */
    }
  }
}}
\catcode`\#=6
\begin{document}
\EndPreamble
\end{texsource}

\paragraph{Custom environments in MathJax.}\label{sec:mathjax_env}

If you want to add support for a new environment, you can use the \verb|\VerbMath| command, and you will
also need to pass suitable configuration to MathJax. For example for the following TeX file:

\begin{texsource}
\documentclass[12pt]{book}    
\usepackage{amsmath} 
\usepackage{breqn}

\begin{document}    
\begin{dgroup*}
\begin{dmath*}
     A = B
   \end{dmath*}     %error is around here           
\end{dgroup*}
\end{document}
\end{texsource}

Can be configured using this configuration file:

\begin{texsource}
\Preamble{xhtml} 
\Configure{MathJaxConfig}{{ 
    tex: { 
      tags: "ams", 
      \detokenize{% 
        environments: {
          "dgroup*": ["", ""],
          "dmath*": ["", ""],
        } 
      } 
    } 
}} 
\VerbMath{dgroup*}
\begin{document} 
\EndPreamble
\end{texsource}

The \verb|\VerbMath| command can be used just for the \verb|dgroup*| environment, as it will pass the
whole contents, including the \verb|dgroup*| environment, to the HTML file. But you will need to 
provide MathJax configuration for both of these environments, as they are not supported by default.


