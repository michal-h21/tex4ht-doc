% info about super and subscripts with MathML
% https://tex.stackexchange.com/q/518839/2891

Subscript and superscript support in \texfourht\ is a bit complicated. The
catcodes of \texcommand{_} and \texcommand{^} characters are changed and they are
active characters instead. This is necessary in order to insert the markup tags
for these structures. One consequence of this is that you need to use a correct grouping
for content that should be affected by these operators.
For example, instead of 

\begin{texsource}
$A_\mathit{...}$
\end{texsource}

use the following, with added group around the \texcommand{\mathit} command:

\begin{texsource}
$A_{\mathit{...}}$
\end{texsource}

If you define custom commands in the document preamble, use \texcommand{\sp}
and \texcommand{\sb} instead of \texcommand{^} and \texcommand{_}.

For example:

\begin{texsource}
\newcommand \coeffX [4][X]{\mathbf{#1}_{{#2},{#3}}(#4)}
\end{texsource}

should become:

\begin{texsource}
\newcommand\coeffX [4][X]{{\mathbf{#1}}\sb{#2,#3}(#4)}
\end{texsource}

Sometimes, it is not possible to change the source files. In this situation, you 
can pre-process the source using custom script that reads the original file, fixes the code
and then pipes the modified output to \shellcmd{make4ht}.

For example, you may want to change this input:

\begin{texsource}
$k'^4_{i}$
\end{texsource}

to the correct form:

\begin{texsource}
$k^{\prime 4}_{i}$
\end{texsource}

This, and other common issues, can be done using the following Lua script:

\begin{luasource}
for line in io.lines() do
  -- fix primes
  line = line:gsub([[%'%^(.-)%_]], [[^{\prime %1}_]])
  -- fix \mathrm{hello}^2
  line = line:gsub([[(\[a-zA-Z]-)(%b{})([%^%_])]], [[{%1%2}%3]])
  -- fix x_\mathrm{hello}
  line = line:gsub([[([%_%^])(\[a-zA-Z]-)(%b{})]], "%1{%2%3}")
  -- fix 10^2
  line = line:gsub("([%d])([%_%^])", "{%1}%2")
  -- fix {}_x
  line = line:gsub("{}([%_%^])", "{\\HCode{}}%1")
  print(line)
end
\end{luasource}

It uses Lua version of regular expressions to find the code we want to fix.
The command to execute it could be:

\begin{shellcommand}
texlua filter.lua < sample.tex | make4ht -j sample - "mathml,mathjax"
\end{shellcommand}

The \shellcmd{-j} option is used to name the output file, which is necessary
because the input comes from a pipe. The dash used in the place of filename
tells \shellcmd{make4ht} to read input from the standard input.

We use \shellcmd{MathJax} to render \mathml, as \mathml\ is not supported by
all browsers. 
