\documentclass{book}
\usepackage{url}
\usepackage{xcolor}
\usepackage{array}
\usepackage{graphicx}

% \usepackage[T1]{fontenc}
\usepackage{upquote}
\usepackage{textcomp}
\usepackage{hyperref}
\usepackage{csquotes}
\usepackage{tex4ht-styles}



\usepackage{glossaries}
\title{TeX4ht Documentation}

\author{by TeX4ht Project}

\begin{document}

\maketitle

% Don't introduce table of contents in the HTML mode, as it introduces another page
\ifdefined\HCode\else\tableofcontents\fi


\chapter{Introduction}


\begin{acknowledgements}
This work was supported with financial support from \href{https://cstug.cz/}{CSTUG}.
\end{acknowledgements}

\chapter{Basic Tutorial}
\section{What is \TeX4ht?}\label{what-is-tex4ht}

\href{https://www.tug.org/tex4ht/}{\texttt{TeX4ht}} is a system that
converts LaTeX to various output formats, including \texttt{HTML},
\href{http://en.wikipedia.org/wiki/OpenDocument}{\texttt{ODT}},
\href{http://en.wikipedia.org/wiki/DocBook}{\texttt{DocBook}} or
\href{http://en.wikipedia.org/wiki/Text_Encoding_Initiative}{\texttt{TEI}}.
\texttt{HTML} and \texttt{ODT} formats are the most common and best-supported
conversion targets.

\texttt{TeX4ht} allows authors to convert \LaTeX\ input to 
several output formats, like \texttt{HTML} (for web pages) or
\texttt{ePub} (for ebooks and other applications).

\hypertarget{basic-usage}{%
\section{Basic Usage}\label{sec:tutorial-basic-usage}}

Conversion is invoked using the \makefourht\ command:

\begin{shellcommand}
$ make4ht filename.tex
\end{shellcommand}


Let us start with conversion of a simple \LaTeX\ file to HTML  
with the following \LaTeX\ file:

\texinput{examples/otherlang/babel.tex}

You can compile it using the following command:

\begin{shellcommand}
$ make4ht -lm draft filename.tex
\end{shellcommand}

The resulting HTML file contains the following code:

\htmlinput{examples/otherlang/babel-lua-body.html}

As you can see, multiple options can be joined for \makefourht.  The above invocation is equal to the following:

\begin{shellcommand}
$ make4ht -l -m draft filename.tex
\end{shellcommand}

You can also use the long options:

\begin{shellcommand}
$ make4ht --lua --mode draft filename.tex
\end{shellcommand}

What do these options mean? 

The option \shellcmd{--lua} tells \makefourht\  to use Lua\LaTeX\ as a
compilation engine. There is also an option \shellcmd{-x} (or \shellcmd{--xetex})
that allows the use of Xe\LaTeX\ for the compilation. If neither of these options
is used, the file will be converted with the default PDF\LaTeX\ engine.

For example, if we compile the sample file without the \shellcmd{-l} option, we would get
a different result:

\htmlinput{examples/otherlang/babel-body.html}

Notice that the accents in the \textit{ť} and \textit{ď} letters are detached from the base letters.
That is because \texfourht{} uses information about characters in the DVI file. Current \LaTeX{}
supports basic accented characters out of the box, but sometimes, they don't work as expected.


The option \shellcmd{--mode} sets the compilation mode. \makefourht\  has one built-in mode,
named \texttt{draft}. 
By default, \makefourht{}  compiles your \TeX{} file three times, to obtain the correct hyperlinks 
and other features that depend on auxilary files. The \texttt{draft} mode uses
only one compilation run, so it is much faster.



\makefourht\  converts \LaTeX\ file to a HTML 5 document. You can request
conversion to other formats using the \texttt{-f} option. For example,
to convert a document to the OpenDocument Format, use the following:

\begin{shellcommand}
$ make4ht -f odt filename.tex
\end{shellcommand}

More information about \makefourht and its command line options and other features can be found in
section \namerefpage{sec:make4ht-intro}.


\section{\texfourht\ Options}

The simplest way to change some aspects of the design is to use \texfourht{} options. They can be passed
as a first positional argument after filename to \makefourht:

\begin{shellcommand}
$ make4ht filename.tex "option1,option2"
\end{shellcommand}

For example, \texfourht{} produces one HTML file for a document, but each footnote is placed in a separate file.
If you have a large document, you may want to use a separate page for each chapter, with a list of footnotes
at the end of these chapters. You can use the following options: 

\begin{shellcommand}
$ make4ht filename.tex "3,sec-filename,fn-in"
\end{shellcommand}


There are other numeric options, each of them breaks document into separate HTML pages on a different sectioning level. Option 1 does not break pages at all, 2 at parts, 3 at chapters, 4 at sections, 5 at subsections, 6 at sub-subsections, and 7 at paragraphs. The \verb|sec-filename| option will produce HTML file names that are based on section titles, instead of their numbers. The \verb|fn-in| option prints footnotes at the end of each HTML page.

There are also options that change the handling of math. Normally, HTML elements are used for simple math, and pictures are used for more complex features, such as fractions or square roots. This usually does not look good, so what are other options?

Generally, it is best to use \mathml{}, as it supports correct vertical alignment for inline math, and the font size matches the surrounding text. Unfortunately, some web browsers do not support it yet. We can use MathJax to render math in these browsers. 

\begin{shellcommand}
$ make4ht filename.tex "mathml,mathjax"
\end{shellcommand}

On the other hand, if you want to use pictures for math exclusively, you can try the \option{pic-m} option, which requires pictures even for inline math. There are also similar options for equations and other math environments.

\begin{shellcommand}
$ make4ht filename.tex "pic-m,pic-equation"
\end{shellcommand}

The generated pictures are in the PNG format, which is raster and depends on the resolution on the device where the document is displayed. You may want to use vector SVG format instead, as it should produce better quality of pictures:

\begin{shellcommand}
$ make4ht filename.tex "pic-m,pic-equation,svg"
\end{shellcommand}

For more information on options, see chapter \namerefpage{chap:options}.

\section{\makefourht{} extensions}

\makefourht{} has an extension support. These extensions can modify various aspects of the conversion process, for example, post-process the generated files, cache images, or add support for Rmarkdown files.
Extensions can be enabled using the \verb|-f format_name+extension_name| option. 

For example, there is a \verb|preprocess_input| extension, which adds support for Markdown or Rtex documents. It can process a following 
Rmarkdown document:

\texinput{examples/otherlang/rmarkdownsample.Rmd}

Compile it using the following command:

\begin{shellcommand}
$ make4ht -f html5+preprocess_input sample.Rmd
\end{shellcommand}

It producess a following HTML file:

\htmlinput{examples/otherlang/rmarkdownsample-body.html}

If your document produces many pictures, the compilation can take a long time. To make it faster, you can use the \verb|dvisvgm_hashes| extension. It caches the SVG images
and creates them only for the changed math environments.

\begin{shellcommand}
$ make4ht -f html5+dvisvgm_hashes filename.tex "pic-m,pic-equation,svg"
\end{shellcommand}

\makefourht{} loads the \verb|common_domfilters| extension automatically. It fixes common issues in the generated HTML files using the LuaXML package. To disable extension
from loading, use \verb|-extension_name| syntax:


\begin{shellcommand}
$ make4ht -f html5-common_domfilters filename.tex
\end{shellcommand}

You can find a list of extensions in \href{https://www.kodymirus.cz/make4ht/make4ht-doc.html#extensions}{\makefourht{} documentation}.

\section{Debugging}

% ToDO: make4ht -a debug
% ToDo: make4ht -m clean
% ToDo: \ifdefined\HCode

\section{Configurations}
\label{sec:tutorial-configurations}

Most of the markup produced by \texfourht{} is configurable. Supported commands 
can be configured using the \texcommand{\Configure} command. We can also insert 
markup before and after environments, using \texcommand{\ConfigureEnv} command.

While it is possible to insert these commands directly to your document, it is better
to use a custom configuration file, as you would get a compilation error if you compiled 
document containing \texfourht{} commands directly by \LaTeX.

% ToDo: sample configuration file

You can find more information about syntax and available commands in section \namerefpage{sec:private-configuration}.
Here, we will show some simple examples.

\subsection{The \cmd{Configure} command}

% ToDo: describe \Configure

\subsection{Configuring Environments}

You may want to insert some custom HTML tags. It is a bit more complicated for
\LaTeX commands, but it is easy for environments. You can configure the code that is 
inserted before and after environment using the \texcommand{\ConfigureEnv} command.
It has a following syntax:

\begin{texsource}
\ConfigureEnv{<environment name>}{before env}{after env}
{before-list}{after-list}
\end{texsource}

We can ignore the arguments \texttt{before-list} and \texttt{after-list}, as they
are used only for list like environments, such as \texttt{itemize}. 
So we just need to to pass code that will be inserted in the \texttt{before env}
and \texttt{after env} arguments.




\section{Remains of the old tutorial}

% ToDo: identify useful parts, improve them and add to the rest of the tutorial, remove the rest

The following text was imported from the original \texfourht\ tutorial and needs to be
rewritten. It still contains some useful information, but there are also some obsolete pieces.


But beware of the following situation:

\begin{texsource}
Hello world.
\begin{someenv}
Just start some environment.

But run it through several paragraphs
\end{someenv}
\end{texsource}

say that we insert
\htmlcommand{<div class="someenv">} and
\htmlcommand{</div>} tags around  the \texttt{someenv}
environment. By default this may produce following structure:

\begin{htmlsource}
<p>Hello world.
<div class="someenv">Just start some environment.
</p>

<p>But run it through several paragraphs
</div></p>
\end{htmlsource}

as you can see, generated html code is incorrect, as opening and closing
\htmlcommand{<div>} tags have different parent elements. \texttt{someenv} environment can
be configured to close current paragraph, but it may be not what you
want.

Best way to prevent tag mismatch may be something like:

\begin{texsource}
Hello world.
\begin{someenv}
Just start some environment.
\end{someenv}

\begin{someenv}
But run it through several paragraphs
\end{someenv}
\end{texsource}

and with \texttt{make4ht}

\begin{shellcommand}
make4ht sample1
\end{shellcommand}

lets look on text part generated by \texttt{htlatex}:

\begin{htmlsource}
<!--l. 6--><p class="noindent" >P&#x0159;íli&#353; &#382;lu&#x0165;ou&#x010D;k&#x00FD; k&#x016F;&#x0148; úp&#x011B;l <span 
class="ecti-1000">&#x010F;</span><span 
class="ecti-1000">ábelsk</span><span 
class="ecti-1000">é </span>ódy. Some text in English
\end{htmlsource}

and by \texttt{make4ht}:

\begin{htmlsource}
<!--l. 6--><p class="noindent" >P&#x0159;íli&#353; &#382;lu&#x0165;ou&#x010D;k&#x00FD; k&#x016F;&#x0148; úp&#x011B;l <span 
class="ecti-1000">&#x010F;</span><span 
class="ecti-1000">ábelsk</span><span 
class="ecti-1000">é </span>ódy. Some text in English
</p> 
\end{htmlsource}

only difference is missing \texttt{\textless{}/p\textgreater{}} tag in
output of \texttt{htlatex}, because \texttt{html\ 4.01} is produced by
\texttt{htlatex} by default. \texttt{make4ht} on the other hand produces
\texttt{xhtml} by default, so closing tag must be presented.

To get \texttt{xhtml} output from \texttt{htlatex}, use
\texttt{tex4ht.sty} option \texttt{xhtml}. This option must be first
option in the option list passed to \texttt{tex4ht.sty}. Value of the
first option must be either \texttt{html}, \texttt{xhtml} or name of
custom config file. We will cover these config files later, as they are
key component in customization of \texttt{TeX4ht} output.

So in order to get same output as from \texttt{make4ht}, we must use
following command:

\begin{shellcommand}
htlatex sample1 xhtml
\end{shellcommand}

Now we should get rid of ugly entities which encode accented letters.
This is somewhat ugly with \texttt{htlatex}:

\begin{shellcommand}
htlatex sample1 "xhtml,charset=utf-8" " -cunihtf -utf8"
\end{shellcommand}

\texttt{charset=utf-8} produces meta element which declares document to
be in \texttt{utf-8} encoding. Important are two options for
\texttt{tex4ht} command, \texttt{-c} and \texttt{-utf8}.

ToDo: add description of process of conversion from \texttt{htf} fonts
to utf8 using unicode.4hf. It is directed from \texttt{tex4ht.env} file.

With \texttt{make4ht}, situation is easier, as all we need to do is to
add \texttt{-u} option:

\begin{shellcommand}
make4ht -u sample1.tex
\end{shellcommand}

resulting file:

\begin{htmlsource}
<!--l. 6--><p class="noindent" >Příliš žluťoučký kůň úpěl <span 
class="ecti-1000">ď</span><span 
class="ecti-1000">ábelsk</span><span 
class="ecti-1000">é </span>ódy. Some text in English
</p> 
\end{htmlsource}

Entities are gone, but other persists. What we see is caused by a bug in
\texttt{tex4ht} command. It decorates text which is set in non-default
font with \texttt{\textless{}span\textgreater{}} elements. Unfortunately
it doesn't play well with accented letters as we can see. This has easy
solution, fortunately. We just need to dive into \texttt{TeX4ht}
configuration. Yay!

\hypertarget{configurations}{%
\section{Configurations}\label{configurations}}

We already saw that we can use command line options to configure the
output. For full list of options for \texttt{tex4ht.sty}, see an
\href{http://www.cvr.cc/?p=504}{article on CVR's blog}. These options
mainly influence appearance or math, footnotes, tables, etc. Note that
these options aren't fixed set, anyone can add new options and not all
options are supported in each output format supported by
\texttt{tex4ht}. Generally these options work with \texttt{html} (and
\texttt{xhtml}) output.

Other option is to use custom config file (\texttt{.cfg}). This is a TeX
file with some basic structure:

\begin{texsource}
 optional stuff like requiring LaTeX packages etc
 ...
 \Preamble{xhtml,tex4ht.sty options}
 ...
 TeX4ht configurations
 ...
 \begin{document} 
 ...
 more TeX4ht configurations
 ...
 \EndPreamble
\end{texsource}

Most important command for configuring is
\texttt{\textbackslash{}Configure}. This command has variable number of
arguments, in the simplest form it does have two arguments:
\texttt{\textbackslash{}Configure\{configname\}\{insert\ for\ a\ first\ hook\}}.

At this place we should talk about hooks. In order to insert html tags,
LaTeX macros are redefined and in the definitions special hooks are
inserted. These hooks are declared with
\texttt{\textbackslash{}NewConfigure\{configname\}\{number\ of\ hooks\}}
in special file named as redefined package name with suffix
\texttt{.4ht}. These hooks are then seeded in configure files for
particular output formats, or in the \texttt{.cfg} file.

To illustrate that, we can show some simple example. Lets say we have
simple package \texttt{hello.sty}:

\begin{texsource}
\ProvidesPackage{hello} 
\newcommand\hello{\textbf{hello world}}
\endinput
\end{texsource}

we can provide hooks in file named \texttt{hello.4ht}. Say we just want
to insert tags at beginning and at end of \texttt{\textbackslash{}hello}
command:

\begin{texsource}
% provide configure for \hello command. we can choose any name
% but most convenient is to name hooks after redefined command
% we declare two hooks, to be inserted before and after the command
\NewConfigure{hello}{2}
% now we need to redefine \hello. save it to tmp command
\let\tmp:hello\hello
% note that `:` can be part of command name in `.4ht` files. 
% now insert the hooks. they are named as \a:hook, \b:hook, ..., \h:hook
% depending on how many hooks were declared
\renewcommand\hello{\a:hello\tmp:hello\b:hello} 
\end{texsource}

because we want to surround contents produced by
\texttt{\textbackslash{}hello} with tags, we need to declare two hooks.
This is the most usual case for normal commands which just produce some
text. Old contents of macro are saved in temporary macro and then
command is redefined to insert hooks and original contents stored in
temporary macro.

Now we can change our sample to use \texttt{hello} package:

\begin{texsource}
\documentclass{article}
\usepackage[english,czech]{babel} 
\usepackage[T1]{fontenc}
\usepackage[utf8]{inputenc} 
\usepackage{hello}
\begin{document} Příliš žluťoučký kůň úpěl \textit{ďábelské} ódy.
\begin{otherlanguage}{english} Some text in English, \hello
\end{otherlanguage} 
\end{document}
\end{texsource}

we haven't provided any configurations for \texttt{hello} yet, but you
can see that text \texttt{hello\ world} is in \textbf{bold} font anyway.
This is the same case as \texttt{\textbackslash{}textit} which is
converted as \emph{italic}. Basic font styles are inserted by
\texttt{tex4ht} command during extraction of text from \texttt{dvi} to a
output format. So it is the right time to finally show how to configure
both \texttt{textit} and \texttt{hello} to produce some better tags than
they provide by default.

Basic structure of a config file has been shown before, so now we will
just add basic configurations for \texttt{\textbackslash{}textit} and
\texttt{\textbackslash{}hello}:

\begin{texsource}
\Preamble{xhtml}
\Configure{textit}{\HCode{<span class="textit">}}{\HCode{</span>}}
\Configure{hello}{\HCode{<span class="hello">}}{\HCode{</span>}}
\Css{.textit{font-style:italic;}}
\Css{.hello{font-weight:bold;}}
\begin{document}
\EndPreamble
\end{texsource}

For documentation of default configurations, see
\href{http://michal-h21.github.io/src4ht/tex4ht-info.html}{TeX4ht info},
most useful are
\href{http://michal-h21.github.io/src4ht/tex4ht-infose2.html}{LaTeX} and
\href{http://michal-h21.github.io/src4ht/tex4ht-infose1.html}{TeX4ht}
sections. Documentation for basic font commands such as
\texttt{\textbackslash{}textit} or \texttt{\textbackslash{}textbf} is
provided in
\href{http://michal-h21.github.io/src4ht/tex4ht-infose2.html}{LaTeX}
section. We can see that configuration takes two parameters, insertion
before and after content. Same situation is with \texttt{hello}
configuration we defined earlier, hooks are inserted before and after
the content.

To insert \texttt{html} tags, we need to use
\texttt{\textbackslash{}HCode} commands, special characters such as
\texttt{\textless{}},\texttt{\textgreater{}} or \texttt{\&} are escaped
otherwise. In our example we insert \texttt{span} elements with some
\texttt{class} attribute to distinguish them. Because these classes
doesn't have any visual appearance by default, we use
\texttt{\textbackslash{}Css} commands to add some styling. Yes, you need
to know both \texttt{html} and \texttt{css} to effectively configure
\texttt{TeX4ht}!

If we look at \texttt{html} output now, we can see that things don't
look much better than initially:

\begin{htmlsource}
<!--l. 6--><p class="noindent" >Příliš žluťoučký kůň úpěl <span class="textit"><span 
class="ecti-1000">ď</span><span 
class="ecti-1000">ábelsk</span><span 
class="ecti-1000">é</span></span> ódy. Some text in English, <span class="hello"><span 
class="ecbx-1000">hello world</span></span>
</p> 
\end{htmlsource}

our new tags were inserted, but unnecessary elements inserted by
\texttt{tex4ht} processor are still present. Fortunately, we can
suppress insertion of these elements with
\texttt{\textbackslash{}NoFonts} command, and later enable again with
\texttt{\textbackslash{}EndNoFonts}. We can also use \texttt{tex4ht.sty}
option \texttt{NoFonts}, which will suppress font processing in whole
document, but you should use this with caution, as it may have some side
effects.

Let's take a look how would out configurations look with
\texttt{\textbackslash{}NoFonts} command:

\begin{texsource}
\Preamble{xhtml}
\Configure{textit}{\HCode{<span class="textit">}\NoFonts}
{\EndNoFonts\HCode{</span>}}
\Configure{hello}{\HCode{<span class="hello">}\NoFonts}
{\EndNoFonts\HCode{</span>}}
\Css{.textit{font-style:italic;}}
\Css{.hello{font-weight:bold;}}
\begin{document}
\EndPreamble
\end{texsource}

the output now looks much better:

\begin{htmlsource}
<!--l. 6--><p class="noindent" >Příliš žluťoučký kůň úpěl <span class="textit">ďábelské</span> ódy. Some text in English, <span class="hello">hello world</span>
</p> 
\end{htmlsource}

It may seems that we can be happy at this point, but things aren't as
easy as we may hope, because we haven't talked about one thing:

\hypertarget{paragraphs}{%
\section{Paragraphs}\label{paragraphs}}

What if we add some more paragraphs in English to our sample file?

\begin{texsource}
\documentclass{article}
\usepackage[english,czech]{babel} 
\usepackage[T1]{fontenc}
\usepackage[utf8]{inputenc} 
\usepackage{hello}
\begin{document} Příliš žluťoučký kůň úpěl \textit{ďábelské} ódy.
\begin{otherlanguage}{english} Some text in English, \hello
\end{otherlanguage} 

\begin{otherlanguage}{english} 

\textit{What will do} \verb|\textit| at the beginning of paragraph?

And also, what about configuration for \verb|otherlanguage| environment?

\end{otherlanguage}

\end{document}
\end{texsource}

What if we want to insert elements with \texttt{lang} attribute to
specify language of text in the \texttt{html}. It might be useful from
semantic point of view, we can also enable hyphenation in the
\texttt{css} and it works only when correct languages are marked in the
source.

This exercise will be little bit more difficult

\chapter{How to}

\section{Change design}
\subsection{Basics}

By default, \texfourht\ separates paragraphs by space. If you want to use text indenting instead, try the \option{p-indent} option.

\subsection{CSS}
\subsection{Web fonts}

It is easy to change fonts in web pages using CSS, but if you want to use
actual font files for the distribution, for example, in an Epub file, \texfourht
provides the following configurations. This example shows how to use OTF files
for the EB Garamond font. 

\begin{texsource}
% define font family. The first argument is family name
% that can be used in CSS font-family rule
% second argument is the family name declared by the font
\Configure{FontFamily}{rmfamily}{EB Garamond}
% declare font filenames for particular font styles. 
% the font files must be placed in the location that is
% declared.
% these four font styles are supported
\Configure{NormalFont}{EBGaramond-Regular.otf}
\Configure{ItalicFont}{EBGaramond-Italic.otf}
\Configure{BoldFont}{EBGaramond-Bold.otf}
\Configure{BoldItalicFont}{EBGaramond-BoldItalic.otf}
% use CSS to use the declared font family in the document
% serif is used as a fallback
\Css{body{font-family: "rmfamily", serif;}}
\end{texsource}


\section{Math}

\subsection{MathJax}


\subsection{MathML}
\subsection{Subscripts and superscripts}
% info about super and subscripts with MathML
% https://tex.stackexchange.com/q/518839/2891

\subsection{MathJax Node}


\section{Graphics}
\subsection{Include graphics (svg,pdf)}
\subsection{Change image size and resolution}
You can use \texcommand{\Configure{Gin-dim}} command to change the way how image dimensions
are calculated. By default, \texfourht\ relies on information about image
dimensions provided by the Graphics package. If you use explicit dimensions
(like \texcommand{width=0.5\textwidth}), the actual dimension calculated by \TeX\ is used.

One problem is that if you set only one dimension, for example width, the other
dimension will be set to the same value. You usually don't want this, unless
your image is a square. To get the correct value for all dimensions, Graphics
uses a \verb|.xbb| file for image. It can be created using the following command:

\begin{shellcommand}
ebb -x *.jpg
\end{shellcommand}

Run analogous command for every other supported image format you use. This will
ensure that correct values are used for implicitly calculated dimensions.

\subsection{Use relative size for images}

The following configuration file example can be used to get image dimensions
relative to the original page width. Thanks to the \LaTeX\ 3 project, we can 
use the \package{l3fp} package to calculate the image dimensions in percents. 

\begin{texsource}
\Preamble{xhtml}
\makeatletter
\ExplSyntaxOn
\Configure{Gin-dim}
{style="width:\fp_eval:n{round(\Gin@req@width/\textwidth*100,2)}\%"}
\ExplSyntaxOff
\makeatother
\begin{document}
\EndPreamble
\end{texsource}

In this configuration, we divide the image width passed to
\texcommand{\includegraphics} by the document text width. 
This portion is then multiplied and rounded to get the correct percent value.
The calculated value is used in the \verb|style| attribute of the generated \verb|<img>| element.
The \package{l3fp} command \texcommand{\fp_eval:n} is used for the calculation. 
\LaTeX\ 3 is part of \LaTeX\ kernel, so we don't need to require packages that it provides.

The following example code:

\begin{texsource}
\includegraphics[width=0.5\textwidth]{example-image.png}
\end{texsource}

produces following HTML code:

\begin{htmlsource}
<img style='width:50%' alt='PIC' src='example-image.png' />
\end{htmlsource}

You can also require the above method using the \option{Gin-percent}
option.

To disable setting of the image dimensions in HTML code completely
use the following configuration:

\begin{texsource}
\Preamble{xhtml}
\Configure{Gin-dim}{}
\Css{img {
    max-width: 100\%;
    height: auto;
}}
\begin{document}
\EndPreamble
\end{texsource}

This configuration uses CSS to set the maximum image dimension. This ensures that
the image is donwsized on devices smaller than is the actual image size.


\subsection{Change image's alternative text}

The \package{Graphicx} package recently added alternative text support,
which can be used to describe the image. This is necessary
for the accessibility purposes.

If you use an older system, without support for this attribute, 
you can use the following example to define it yourself, and use it 
with the \texcommand{\includegraphics} command. The attribute is named 
\texttt{alt}. Here is a sample file that shows how to do that:

\texinput{examples/imagealt/sample.tex}

The \texcommand{\define@key{Gin}{alt}{}} command defines a dummy command that
is used by regular \LaTeX. We need to redefine this key for \texfourht, so it uses the
configuration that can specify the image alternative text, \configuration{GraphicsAlt}.

\texinput{examples/imagealt/mycfg.cfg} 

The resulting HTML file now contains image with the \texttt{alt} attribute:

\htmlinput{examples/imagealt/sample-body.html}


\section{Plain \TeX}

\texfourht\ supports Plain \TeX, but it can be a bit tricky. In general, you
will have to provide configurations to get good markup for your custom
commands. You will also need to add few commands to your source file: 

\begin{texsource}
%!TEX TS-program = tex
\input plain-4ht.tex
\document
Hello world
\enddocument
\end{texsource}

The first line contains magic comment that tells \texttt{make4ht} to use Plain
\TeX\ instead of \LaTeX for the compilation. \texcommand{\document} and \texcommand{\enddocument}
commands are important, because they are crucial for correct execution of \texfourht.

They are declared in the \texttt{plain-4ht.tex} file:

\begin{texsource}
\def\documentstyle#1{}
\documentstyle{tex4ht}
\csname tex4ht\endcsname
\def\document{}
\def\enddocument{\csname bye\endcsname}
\end{texsource}

In the PDF mode, these commands will do nothing, except for execution of the \texcommand{\bye} command.

The document can be compiled using:

\begin{shellcommand}
make4ht -f html5+detect_engine filename.tex
\end{shellcommand}

The \verb|-f html5+detect_engine| requires the \verb|detect_engine| extension,
which uses the magic command in the source file to determine which \TeX\ engine
to use. It is necessary for the correct Plain \TeX\ support.


\section{Blogging}

\section{Work with external commands}
\subsection{Indexing}
\subsection{Bibliographies}
\subsection{R}
\subsection{Markdown}
\subsection{PythonTeX}



The translation of a LaTeX source file into HTML involves of loading tex4ht.sty
and *.4ht style files, choosing the desirable options for the translation,
compiling the source into dvi code with the native LaTeX engine, and
postprocessing the outcome with the tex4ht and t4ht programs (see \nameref{sec:overview}). 


\section{Overview of the Translation Process}\label{sec:overview}

The system can be activated with a sequence of commands of the following form, typically embedded within a script.

\begin{shellcommand}
latex      x            (or ‘tex x’) 
latex      x 
latex      x 
tex4ht     x 
t4ht       x 
\end{shellcommand}

The three compilations with La(TeX) are needed to ensure proper links. The approach is illustrated in the following picture. 

\begin{description}
  \item[x.tex]

This is a source TeX/LaTeX/OtherTeX file that imports the style files tex4ht.sty and *.4ht. The style files define the features for the output.

\item[tex4ht]

The output of \TeX{} is a standard dvi file interleaved with special
instructions for the postprocessor \shellcmd{tex4ht} to use. The special
instructions come from implicit and explicit requests made in the source file
through commands of \texfourht.

The utility tex4ht translates the dvi code into standard text, while obeying
the requests it gets from the special instructions. The special instructions
may request the creation of files, insertion of html code, filtering of
pictures, and so forth.

In the extreme case that the source code contains no commands of TeX4ht, tex4ht
gets pure dvi code and it outputs (almost) plain text with no hypertext
elements in it.

The special (\texcommand{\special}) instructions seeded in the dvi code are not understood
by dvi processors other than those of TeX4ht.

\item[x.idv]

This is a dvi file extracted from x.dvi, and it contains the pictures needed in
the html files.

\item[x.lg]

This is a log file listing the pictures of x.idv, the png files that should be
created, CSS information, and user directives introduced through the
‘\texcommand{\Needs{...}}’ command.

\item[t4ht]
This is an interpreter for executing the requests made in the x.lg script.

\end{description}



\texfourht\ supports following output formats:

\begin{description}
  \item[HTML 5] used by default, when you don't provide the \shellcmd{-f} option.
  \item[XHTML]  alternative if you don't want to use new HTML elements.
  \item[ODT]    suitable for LibreOffice or MS Word.
\end{description}

There are also historicaly other formats with more limited support. Basic structures,
like sections or lists should work, but specific formatting for most of \LaTeX\ packages 
can be missing.

\begin{description}
  \item[DocBook]  
  \item[TEI] 
  \item[JATS] support for this format is limited, even basic document structure is missing.
\end{description}

Additionally, you can create e-book formats such as EPUB or MOBI. For historical reasons, these formats
are created using separate tool, \texttt{tex4ebook}. It supports the folowing formats:

\begin{description}
  \item[AZW]
  \item[AZW 3]
  \item[EPUB]
  \item[EPUB 3]
  \item[MOBI]
\end{description}

To convert your file to any of these formats, use it as a lowercase name, without spaces, in the
\texttt{-f} option of \makefourht:

\begin{shellcommand}
$ make4ht -f odt filename.tex
$ tex4ebook -f epub3 filename.tex
\end{shellcommand}


\chapter{\texfourht\ Options}
\label{chap:options}

% the list of options has been copied from the CVR's blog
% http://cvr.cc/?p=504



Following is an incomplete list of options that can be passed to the \fourhtsty\ package.
These options can be used to modify the compilation process, for example to
select a \term{SVG} format for generated images, to request math environments
to be convert as images or to split sections as separate HTML pages. 

The options may be defined in the \fourhtfile\ files and may depend on the
output format, so it is not feasible to provide their full list. Most of the
following options work only in the \HTML\ output.

There are several ways how to pass the options to \texfourht. The
non-recommended way is to pass them as options to \fourhtsty\ using
\texcommand{\usepackage} command in the \TeX\ file. 

\begin{texsource}
...
\usepackage[xhtml,option1,option2]{tex4ht}
...
\end{texsource}

Recommended solutions don't require modifications of the \TeX\ files.

One way is to pass the options to \texttt{make4ht} as an command line
argument, next to the filename:
% it is run from the command line. These can also be provided as options when \texfourht
% package is loaded in a \LaTeX\ document with the default usepackage command or to
% the \verb|\Preamble| command in the custom config file

\begin{shellcommand}
make4ht filename.tex "fn-in"
\end{shellcommand}

For more information on calling scripts see the section \ref{sec:calling-commands}.

The second way is to pass options in the \texcommand{\Preamble} command in a \cfgfile. Note that first option in the 
\texcommand{\Preamble} command must be \option{xhtml}.

\begin{texsource}
\Preamble{xhtml,fn-in}
...
\begin{document}
\EndPreamble
\end{texsource}

See the section \namerefpage{sec:private-configuration} for more details on
configuration files.


\section{List of options}
\label{sec:texfouhtoptions}

% \begin{tabular}{>{\ttfamily}p{8em} l} 
%   -css & to ignore CSS code, use command line option -css. \\
%   -xtpipes & to avoid xtpipes post-processing the output. This might be useful for docbook XML output.\\
%   0 & pagination shall be obtained through the option 0 or 1, at locations marked with PageBreak.\\
%   1, 2, 3, 4, 5, 6, 7& for automatic sectioning pagination (to break at various section levels), use the appropriate command line option 1, 2, 3, 4, 5, 6, 7.
\begingroup
\catcode`\#=11 \catcode`\^=11 \catcode`\_=11


\begin{description}

\item[-css] to ignore \css\ code, use command line option \verb=-css=.

\item[-xtpipes] to avoid \verb=xtpipes= post-processing the
  output. This might be useful for Docbook \xml\ or ODT output.

  % \item[/bib]
  % \item[/obeylines]
  % \item[0.0]

\item[0] pagination shall be obtained through the option \verb=0= or
  \verb=1=, at locations marked with \verb=\PageBreak=.

\item[1, 2, 3, 4, 5, 6, 7] for automatic sectioning pagination (to
  break at various section levels), use the appropriate command line
  option \verb=1, 2, 3, 4,= \verb=5, 6, 7=. Option 1 break pages
  at parts, 2 at chapters, 3 at sections, 4 at subsections,
  5 at subsubsections and 6 at paragraphs. For document classes without 
  the \cmd{chapter} command, like \package{article}, 2 breaks sections, 3 subsections,
  etc.

\item[DOCTYPE] to request a \verb=DOCTYPE= declaration, use the
  command line option \verb=DOCTYPE=.

\item[Gin-dim] for key dimensions of the graphic, try this option.

\item[Gin-dim+] for key dimensions when the bounding box is not
  available.

\item[NoFonts] don't use original font style information.
\item[NoSections] don't use frames for text alignment environments in 
  the ODT output.

\item[PMath] Option to choose positioned math. Example: 
  \verb=\def\({\PMath$}=;\allowbreak \verb=\def\){$\EndPMath}=;
  \verb=\def\[{\PMath$$}=; \verb=\def\]{$$\EndPMath}=.

\item[RL2LR] to reverse the direction of RL sentences.

%\item[ShowFont]

\item[TocLink] option to request links from the tables of contents.
  
\item[\textasciicircum 13] option for active superscript character.

\item[\_13] option for active subscript character.

\item[accent-] This option is available only together with
  \option{new-accents}. It produces pictures for some math accents.

%\item[base]

\item[bib-] for degraded bibliography friendlier for conversion to
  \verb=.doc=.

\item[bibtex2] Option \verb=bibtex2= requires compilation of
  \verb=\jobname j.aux= with bibtex.

%\item[broken-index]

\item[charset] for alternate character set, use the command line
  option \verb+charset="..."+ (e.g., \verb+charset="utf8"+).

%\item[core]

\item[css-in] the inline \css\ code will be extracted from the input of
  the previous compilation, so an extra compilaion might be needed for
  this option to make it effective.

\item[css2] for \css2 code.

% \item[css]
% \item[debug-]
% \item[debug]
% \item[draw]
% \item[dtd]

\item[early\textasciicircum] for default catcode of superscript in the
  \verb=\Preamble=.

\item[early\_] for default catcode of subscript in the
  \verb=\Preamble=.

%\item[edit]

\item[endnotes] for end notes instead of footnotes, use this option.

%\item[enum]

\item[enumerate+] for enumerated list elements that keep the list couter value. This
  will use the description list like \verb=<dt>...</dt>= for the list
  counter.

\item[enumerate-] for enumerated list element's \verb=<li>='s with
  value attributes, use this command line option. This will be an
  ordered list with the value of list counter provided as an attribute
  namely, \verb=value= of the \verb=<li>= element.

%\item[family]
\item[fancylogo] try to visually emulate \verb|\TeX| and \verb|\LaTeX| logos.

\item[fn-in] for inline footnotes use this option.

\item[fn-out] for offline footnotes.

\item[fonts] use HTML elements and CSS for \latex\ font commands, such as
  \verb|\textit|.

\item[fonts+] for marking of the base font, use this option.

\item[font] for adjusted font size, use the command line option
  \verb+font=...+ (e.g., font=-2).

\item[frames-] for frames support. \verb=frames= is also valid option
  for frames support.

\item[frames-fn] for content, \chfont{TOC}\ and footnotes in
  three frames.

\item[frames] for \chfont{TOC}\ and content in two frames.

%\item[fussy]

\item[gif] for bitmaps of pictures in \verb=.gif= format, use this
  option.

\item[graphics-] if the included graphics are of degraded quality, try
  the command line options \verb=graphics-num= or \verb=graphics-=.
  The \verb=num= should provide the density of pixels in the bitmaps
  (e.g., 110).

%\item[graphics-dim]

\item[hidden-ref] option to hide clickable index and bibliography
  references.

% \item[hooks++]
% \item[hooks+]
% \item[hooks]
% \item[hshow]
% \item[htm3]
% \item[htm4]
% \item[htm5]
% \item[htm]

\item[html+] for stricter \HTML\ code.

%\item[html]

\item[imgdir] for addressing images in a subdirectory, use the option
  \verb=\imgdir:.../=.

\item[image-maps] for \verb=image-maps= support.

\item[index] for \emph{n}-column index, use the command line option,
  \verb+index=n+ (e.g., index=2).

\item[info-oo] for extra tracing information while generating open
  office output.

\item[info] for extra information in the \verb=\jobname.log= file.

\item[java] for \verb=java=support.

\item[javahelp] for \verb=JavaHelp= output format, use this command
  line option.

\item[javascript] for \verb=javascript= support.

\item[jh-] for sources failing to produce \xml\ versions of \HTML, try
  this command line option.

%\item[jh1.0]

\item[jpg] for bitmaps of pictures in \verb=.jpg= format, use this
  option.

\item[li-] for enumerated list elements li's with value attributes.

\item[math-] option to use when sources fail to produce clean math
  code.

\item[mathjax] use \term{MathJax} for the math rendering.
%\item[mathaccent-]

\item[mathltx-] option to use when sources fail to produce clean
  \verb=mathltx= code.

\item[mathml-] option to use when sources fail to produce clean
  \mathml code.

\item[mathplayer] for \mathml\ on Internet Explorer + MathPlayer.

\item[minitoc\textless] for mini tocs immediately after the header use the
  command line option, \verb=minitoc<=.

\item[mouseover] for pop ups on mouse over.

\item[new-accents] alternative configurations for accented characters. 

\item[next] for linear cross-links of pages, use this option.

\item[nikud] for Hebrew vowels, use the command line option,
  \verb=nikud=.

\item[no-DOCTYPE] to remove \texttt{DOCTYPE}\space declaration from
  the output.

\item[no-VERSION] to remove \verb+<?xml version="..."?>+ processing
  instruction from the output.

\item[NoFonts] disable ht-fonts processing in the document.

% \item[no-align]
% \item[no-array]
% \item[no-bib]
% \item[no-cases]
\item[no-halign] suppress \texcommand{\halign} redefinition. It doesn't work with the \texcommand{tabular} environment.
% \item[no-matrix]
% \item[no-pmatrix]

\item[no\textasciicircum] for non-active \verb=^= (superscript), use the option
  \verb=no^=.

\item[no\_] for non-active \verb=_= (subscript command), use the
  command line option, \verb=no_=.

\item[no\_\textasciicircum] for both non-active superscript and subscript, use the
  option \verb=no_^=.

\item[nolayers] to remove overlays of slides, use this option.

\item[nominitoc] this will eliminate mini tables of contents from the
  output.

\item[notoc*] for tocs without \verb=*= entries, use this option. The
  \verb=notoc*= option is applicable only to pages that are
  automatically decomposed into separate web pages along section
  divides. It shall be used when \verb=\addcontentsline= instructions
  are present in the sources.

\item[obj-toc] for frames-like object based table of contents, use the
  command line option \verb=obj-toc=.

%\item[old-longtable]

\item[p-width] for width specifications of tabular \verb=p= entries,
  use this option.

\item[p-indent] for indented paragraphs, without blank spaces.

\item[pic-RL] for pictorial RL.

\item[pic-align] for pictorial align environment.

\item[pic-array] for pictorial array.

\item[pic-cases] for pictorial cases environment.

\item[pic-eqalign] for pictorial equalign environment.

\item[pic-eqnarray] for pictorial eqnarray.

\item[pic-equation] for pictorial equations.

\item[pic-fbox] for pictorial or bitmapped fbox'es.

\item[pic-framebox] for bitmap fameboxes.

\item[pic-longtable] for bitmapped longtable.

\item[pic-m+] for pictorial \verb=$...$= and \verb=$$...$$=
  environments with \latex\ alt, use the command line option
  \verb=pic-m+= (not safe).

\item[pic-m] for pictorial \verb=$...$= environments, use the command
  line option \verb=pic-m= (not recommended).

\item[pic-matrix] for pictorial matrix.

% \item[pic-tabbing']

% \item[pic-tabbing]

% \item[pic-table]

\item[pic-tabular] use this option for pictorial tabular.

\item[plain-] for scaled down implimentation.

% \item[pmathml-css]

% \item[pmathml]

% \item[postscript]

\item[prog-ref] for pointers to code files from root fragments, use
  the command line option \verb=prof-ref=. This is for debugging.

\item[refcaption] for links into captions, instead of flat heads, use
  this option.

\item[rl2lr] to reverse the direction of Hebrew words, use this
  option.

\item[sec-filename] for file names derived from section titles, use
  the command line option \verb=sec-filename=.

\item[sections+] for back links to table of contents, use this option.

% \item[sections-]
% \item[settabs-]
% \item[stackrel-]

\item[svg-] for external \svg\ files, try this option.

\item[svg-obj] same as above.

\item[svg] for dvi pictures in \verb=svg= format.

\item[svg-inline] same as the \option{svg}, but the \svg\ files are included in the document body.

\item[tab-eq] for tab-based layout of equation environment, use this
  option.

%\item[th4]

\item[trace-onmo] for mouseover tracing of compilation, use the
  command line option, \verb=trace-onmo=.

% \item[uni-emacspeak]
% \item[uni-html4]
% \item[uniaccents]
% \item[unicode]

% \item[url-]

\item[url-enc] for \chfont{URL}\space encoding within href, use this
  option.  \verb=\Configure{url-encoder}= can be used to fine tune
  encoding.

\item[url-il2-pl] for il2-pl \chfont{URL} encoding.

\item[ver] for vertically stacked frames. Effective when \verb=frames=
  option is requested.

% \item[verify+]
% \item[verify]

\item[xht] for file name extension, \verb=.xht=, use this command line
  option.

\item[xhtml] for \xml\ code, use the command line option, \verb=xml= or
  \verb=xhtml=.

\item[xml] See previous entry.

% \item[xmldtd]

\end{description}

\section{Options for the ODT output}

\begin{description}
  \item[bib-] produces degraded bibliography that should be friendlier for conversion to Word.
  \item[description-inline] use run-in style for description lists.
  \item[endnotes] convert footnotes to endnotes.
  \item[hidden-ref] hide clickable index and bibliography references.
  \item[NoSections] don't use sections for text alignment environments.
  \item[tab-eq] tab based layout of equations.
  \item[timestamp] save creation date in the document metadata.
  \item[TocLink] request links from table of contents.
\end{description}
\endgroup


\chapter{Configurations}

\section{Private Configuration Files}\label{sec:private-configuration}

It is highly recommended to leave source LaTeX and TeX files intact, and not
introduce \texfourht{} configurations there. The configurations should be introduced
indirectly in private configuration files. Source files containing just native
LaTeX and TeX code permit their compilation to different output formats,
including PostScript and PDF, by \texfourht{} and other tools.

Packages used by the general LaTeX community typically provide better support
than one can expect from tailoring private commands and configurations for such
commands. It is also expected to take less effort to learn the features of
existing packages than designing new ones. Consequently, one is advised to
investigate available resources before committing to work on private features. 

\subsection{Requesting Private Configuration Files}

Private configuration files can be requested by passing the \shellcmd{-c}
option to \makefourht:

\begin{shellcommand}
make4ht -c mycfg.cfg myfile 
\end{shellcommand}


A configuration file should take the following form for \LaTeX\ files:

\begin{texsource}
...early definitions...
\Preamble{xhtml,options}
...definitions...
\begin{document}
...insertions into the header of the html file...
\EndPreamble
\end{texsource}

The \cmd{Preamble} command should always use \option{xhtml} 
as the first option. Otherwise, \texfourht\ switches to the HTML 4 mode and XML
post-processing tools used by \makefourht{} to fix some common issues will fail! For 
more information about available options, see the \nameref{sec:texfouhtoptions} 
section (\ref{sec:texfouhtoptions}).

\begin{texsource}
\Preamble{xhtml} 
\begin{document} 
  \Css{body { color : red; }} 
\EndPreamble 
\end{texsource}

\paragraph{Notes}

\begin{itemize}
  \item Notice that for a LaTeX file the \texcommand{\begin{document}}
    instruction should be present both in the configuration file and the source
    file.

  \item Instructions defined within a source file may be redefined in a
    configuration file. Such a feature enables to keep source files intact for
    compilation to different formats by different tools.

    For instance, a definition of the form \texcommand{\renewcommand\mycommand{...}} within a
    configuration file provided for the following \LaTeX\ source:
\end{itemize}


\begin{texsource}
\documentclass{...} 
\newcommand\mycommand{...} 
\begin{document} 
Use \mycommand{...} 
\end{document} 
\end{texsource}

\subsection{Configuration file management}

It is possible to reuse common \texfourht\ configurations used in several
configuration files.  They can be inserted in a custom LaTeX package, but there
is one important thing to be aware of. The configuration hooks are inserted to
the patched commands when the compilation reaches the  
\texcommand{\begin{document}} command, so configurations for these hooks
declared before the hook definition have no effect. It is necessary to include
them in the \cmd{AtBeginDocument} command.

Sample package, \file{commonconfigurations.sty}:

\begin{texsource}
\ProvidesPackage{commonconfigurations}
\AtBeginDocument{%
\Configure{@HEAD}
{\HCode{<meta name="test" content="test"/>\Hnewline}}
}
\endinput
\end{texsource}

It can be requested in a configuration file using \cmd{RequirePackage} command.

\begin{texsource}
\Preamble{xhtml}
\RequirePackage{commonconfigurations}
\begin{document}
\EndPreamble
\end{texsource}

\section{Custom output formats}

% ToDo: reuse content from: https://tex.stackexchange.com/a/477891/2891 and https://www.tug.org/applications/tex4ht/mn11.html#QQ1-11-66

I've created some alternative commands to \verb|\HCode| or \verb|\Tg|. The idea is to define
semantic names for logical elements of the document, such as titles, authors,
sections etc. It is possible to assign HTML elements and attributes to these
logical elements. There are commands for inline and block level elements,
which should eliminate the need for constructs like \verb|\ifvmode\IgnorePar\fi\EndP|
etc.

I think it will be best to show some concrete examples:


\begin{texsource}
\NewLogicalBlock{textit}
\SetBlockProperty{textit}{class}{textit}

\NewLogicalBlock{maketitle}
\SetTag{maketitle}{header}

\NewLogicalBlock{titlehead}
\SetTag{titlehead}{h1}
\SetBlockProperty{titlehead}{class}{titleHead}

\Configure{textit}
{\NoFonts\InlineElementStart{textit}{}}
{\InlineElementEnd{textit}\EndNoFonts}

\Configure{maketitle}{%
{\Configure{maketitle}{}{}{}{}%


\Tag{TITLE+}{\@title}}
\BlockElementStart{maketitle}{}}
{\BlockElementEnd{maketitle}}
{\NoFonts\BlockElementStart{titlehead}{}}
{\BlockElementEnd{titlehead}\EndNoFonts}
\end{texsource}



The \verb|\NewLogicalBlock| creates a new logical element. The used tag is configured
using \verb|\SetTag|, the attributes are set using \verb|\SetBlockProperty|. The blocks are
inserted to the document using \verb|\InlineElementStart| ...  \verb|\InlineElementEnd| or
\verb|\BlockElementStart| ... \verb|\BlockElementEnd| pairs. The start commands take two
parameters, first is the logical block name, the second can be local
parameters which shouldn't be used for the given logical block globally.

The main idea behind this mechanism is to allow easy work with new HTML5
elements and attributes for WAI-ARIA or Schema.org properties. I hope that
this should allow us to make somehow more clear configurations for basic
document building blocks.

\section{Document Styling Using CSS}
\label{sec:css-styling}

\texfourht\ provides several commands that can be used for changing of the
document appearance using Cascading Style Sheets (\css). Only basic styling for
the document is provided by default. Additional styles are added by configurations for the
fonts, packages and commands used in the document. Full control of the document
styling can be achieved using following commands and configurations.


\cmd{Css}\marg{content}

This command sends its content to the CSS file of the document. 

\DocConfigure{AddCss} {CSS file name}\EndDoc

Add local CSS file.

\begin{texsource}
\Css{body{max-width:70ch;margin: 1rem auto;}}
\Configure{AddCss}{custom.css}
% link to a CSS on some web:
\Configure{@HEAD}{\HCode{<link type="text/css" rel="stylesheet" href="https://example.com/style.css" />}}
\end{texsource}

\DocConfigure{CssMediaQueryStart}{media type}{media query condition}\EndDoc
\cmd{CssMediaQueryEnd}

These two commands are used to define a CSS media query block within your
document. Everything between \cmd{CssMediaQueryStart} and \cmd{CssMediaQueryEnd} is
wrapped inside a media query and is included in the output CSS accordingly.

\begin{description}
  \item[media type] (optional) The media type, such as screen, print, all, etc. If empty, only the media condition is used.

  \item[media query condition] The condition for the media query, e.g., \texttt{max-width: 600px}.
  
\end{description}


Example:


\begin{texsource}
\CssMediaQueryStart{screen}{max-width: 600px}
  \Css{body {background-color: lightblue;}}
  \Css{h1 {color: navy;}}
  \Css{p {font-size: 20px;}}
\CssMediaQueryEnd
\end{texsource}

This will produce:

\begin{csssource}
@media screen and (max-width: 600px) {
  body {background-color: lightblue;}
  h1 {color: navy;}
  p {font-size: 20px;}
}
\end{csssource}

\begin{itemize}
  \item Any number of \cmd{Css}\marg{...} commands can be placed between the start and end markers.

  \item These commands are useful for responsive design, allowing conditional styling based on the screen size or output medium.

\end{itemize}


\cmd{CssFile}[list-of-css-files]content\cmd{EndCssFile}

The CSS file \texfourht\ used by default initially consists just
a single line,  \texcommand{/* css.sty */}. This line is later
replaced with the code submitted by the \texcommand{\Css{...}} commands.

The \texcommand{\CssFile} command allows to specify an alternative to the initial CSS file.
The alternative file consists of the code loaded from listed files, and of the
content explicitly specified in its body.

\begin{texsource}
\ConfigureList{mylist} 
{\HCode{<div class="mylist">}} {\HCode{</div>}} {* }{} 
       
\begin{document} 
       
\CssFile 
/* css.sty */ 
.mylist { color : red; } 
\EndCssFile 
\end{texsource}

The names in the list of files should be separated by commas, and the rectangular brackets are optional when the list is empty.

The file should include a line having the content of \verb|/* css.sty */|. If
more than one such line is included, the content of the \texcommand{\Css{...}} commands
replace the first occurrence of this line. Arbitrary many space characters may
appear around the substrings ‘/*’ and ‘*/’. 



The declared font family is not used automatically, it is necessary to select
it using the \term{font-family} Css property.

The default font family name which should be used in the Css
\term{font-family} command for a declared font is \term{rmfamily}. 
It use the Latin Modern font installed on the viewer's system. 
The Css font family and the local font name can be changed using
\verb|\Configure{FontFamily}{cssfamilyname}{LocalFontName}| command.

\begin{texsource}
\Configure{FontFamily}{rmfamily}{Latin Modern}
\end{texsource}

The font shapes can be configure using \verb|\Configure{NormalFont}|, 
\texcommand{\Configure}\allowbreak\texcommand{{ItalicFont}}, \verb|\Configure{BoldItalicFont}| and
\verb|Configure{BoldFont}|. The argument should be font file in the format
supported by browsers, such as \textit{woff} or \textit{otf}.


\begin{texsource}
\Configure{NormalFont}{normal-font-file.otf}
\Configure{BoldFont}{bold-font-file.otf}
\Configure{BoldItalicFont}{bold-italic-font-file.otf}
\Configure{ItalicFont}{italic-font-file.otf}
% Add another font family
\Configure{FontFamily}{hello}{Linux Libertine O}
\Configure{NormalFont}{hello-font-file.otf}
\Css{body{
  font-family: rmfamily, "AnotherFontFamilyName", serif;
}}
\Css{span.hello{font-family: hello, sans-serif;}}
\end{texsource}

\section{Use JavaScript}
\label{sec:javascript}

\DocConfigure{AddJs} {filename of JS file to be included}\EndDoc 

This configuration inserts link to a local JavaScript file. 

\DocCommand{JavaScript} ... \cmd{EndJavaScript} insert JS code in the verbatim mode.
This command is available only when the \option{javascript} option is active. It can put JS code
directly to the document. If used in the configuration file, it should be placed after
\texcommand{\begin{document}}.

Full example of configuration file that use both of these methods:

\begin{texsource}
\Preamble{xhtml,javascript}
\Configure{AddJs}{sample.js}
\begin{document}
\JavaScript
console.log("Hello JS");
\EndJavaScript
\EndPreamble

\end{texsource}


\subsection{File extension}

\DocConfigure {ext} {default extension name for target files  (recorded in \texcommand{\:html})} \EndDoc

It can also be requested through a command line option \shellcmd{ext=...}.

\subsection{HTML header}
\DocConfigure {PROLOG} {Comma separated list of hooks to appear before HTML} \EndDoc

Each hook E is declared to be configurable by an instruction of the form \texcommand{\NewConfigure{E}{1}}


Example:

\begin{texsource}
\Configure{PROLOG}{VERSION,DOCTYPE,*XML-STYLESHEET}
\Configure{VERSION}{\HCode{<?xml version="1.0"?>}}
\end{texsource}

A star '*' prefix calls for accumulative configurations


\subsection{Titles}

\DocConfigure{TITLE+} {Title contents}\EndDoc

Set contents of the \htmlcommand{<title>} element.

There are also similar commands for pages split on sectioning commands. For example, for document that
is split on chapters, use:

\begin{texsource}
\Configure{TITLE+}{My web site |\@title}
\Configure{chapterTITLE+}{My web site | \thechapter\space#1}
\Configure{likechapterTITLE+}{My web site | #1}
\end{texsource}

\texttt{likechapter} prefix is used for \texcommand{\chapter*} command. The text of sectioning command
can be accessed using \texcommand{#1}.


% \section{Document Navigation}
\section{Cross-links}
\label{sec:cross-links}

Cross-links provide navigation between HTML pages broken into multiple files from a single source document.

The following configurations modify behaviour of cross-links between pages in a multi page document.

\DocConfigure{crosslinks} {left-delimiter} {right-delimiter} {next} {prev} {prev-tail} {front} {tail} {up}\EndDoc

This command configures the appearance of the cross-links between hypertext pages obtained for sectioning commands.

\begin{texsource}
 \Configure{crosslinks}
   {}{}{$\scriptstyle\Rightarrow$}
   {$\scriptstyle\Leftarrow$}
   {}{}{}{$\scriptstyle\Uparrow$}
\end{texsource}

\DocConfigure{crosslinks*} {1--7 arguments}\EndDoc

Links to be included and their order. Available
  options: \textit{next}, \textit{prev}, \textit{prevtail}, \textit{tail}, \textit{front}, \textit{up}.
  The last argument must be empty.

Default values:

\begin{texsource}
\Configure{crosslinks*}{next}
   {prev}{prevtail}
   {tail}{front}
   {up}{}
\end{texsource}

\DocConfigure{crosslinks+} {before-top-links} {after-top-links} {before-bottom-links} {after-bottob-links}\EndDoc

The top cross links are omitted, if both \verb|#1| and \verb|#2| are empty.
The bottom cross links are omitted, if both \verb|#3| and \verb|#4| are empty.

\DocConfigure{next} {the anchor of the next button of the front page}\EndDoc

Default: The value provided in \texcommand{\Configure{crosslinks}}

\DocConfigure{next+} {before} {after}\EndDoc

\begin{description}
  \item[\#1]  before the next button of the front page, when the `next'
       option is active.
  \item[\#2]  after the button
\end{description}

    Default: The values provided in \texcommand{\Configure{crosslinks}}

\DocConfigure{crosslinks:next} {local configurations for the delimiters and hooks}\EndDoc
\DocConfigure{crosslinks:prev} {local configurations for the delimiters and hooks}\EndDoc
\DocConfigure{crosslinks:prevtail} {local configurations for the delimiters and hooks}\EndDoc
\DocConfigure{crosslinks:tail} {local configurations for the delimiters and hooks}\EndDoc
\DocConfigure{crosslinks:front} {local configurations for the delimiters and hooks}\EndDoc
\DocConfigure{crosslinks:up} {local configurations for the delimiters and hooks}\EndDoc


\DocConfigure{crosslinks-}{before} {after}\EndDoc

Asks to show linkless buttons with the following insertions.

The default values are used, if both \verb|#1| and \verb|#2| are empty

   Examples:

\begin{texsource}
\Configure{crosslinks-}{}{}

\Configure{crosslinks-}
    {\HCode{<span class="hidden">}[}
    {]\HCode{</span>} }
\Css{span.hidden {visibility:hidden;}}
\end{texsource}



\section{Tables of Contents}

\section{Sections}
\section{Lists}
\section{Tables}

\section{Fonts}
\subsection{Basic font commands}

Information about the \option{fonts} option and \term{MathML} issues. 
Example configuration:
\url{https://tex.stackexchange.com/a/416613/2891}

\section{Multi-lingual support}

RTL support in the ODT output: \url{https://tex.stackexchange.com/a/470434/2891}.

\subsection{Right-to-left text}

There is a difference in the RTL support for HTML and ODT output formats. In HTML, RTL can be requested using:

\DocConfigure{LRdir} { value for the dir attribute}\EndDoc

Example:

\begin{texsource}
\ConfigureEnv{arab}
{\Configure{LRdir}{ dir="rtl"}}
{\Configure{LRdir}{}}{}{}
\end{texsource}

This configuration sets the direction to \term{rtl} inside the \term{arab} environment and resets it after the environment end.

In the ODT output, different mechanism is used:

\begin{texsource}
\ConfigureEnv{arab}{\@rltrue}{\@rlfalse}{}{}
\end{texsource}

\subsection{Unicode}
\label{sec:unicode}

Generally speaking, \texfourht\ supports \term{Unicode}, but there are some
issues to be aware of. Most complete support exists for Lua\LaTeX, thanks to
special Lua script which is automatically loaded during the compilation. No
additional packages are necessary.

PDF\LaTeX\ doesn't support nativelly, but it is possible to emulate it using the
\package{inputenc} and \package{fontenc} packages:

\begin{texsource}
\documentclass{article}
\usepackage[utf8]{inputenc}
\usepackage[T1]{fontenc}
\begin{document}
Unicode text
\end{document}
\end{texsource}

Xe\LaTeX\ is an Unicode format, similarly to Lua\LaTeX. The supporting
mechanism for \texfourht\ is different in this case and full Unicode range is
not supported out of the box. By default, only most Latin based characters are
supported. For other scripts, such as Greek or Cyrillic, two ways to enable
support exists. 

First option is to define new font family using \package{fontspec} \texcommand{\newfontfamily} with the \texttt{Script} option.

\begin{texsource}
\newfontfamily\greekfont{Linux Libertine O}[Script=Greek]
\end{texsource}


The second option is to declare load support for a script in the custom config
file using the \texcommand{\xeuniuseblock}:


\begin{texsource}
\xeuniuseblock{Greek}
\end{texsource}

The block names are based on \href{https://en.wikipedia.org/wiki/Unicode_block}{Unicode blocks}.

It is also possible to declare all characters in an Unicode range. The command
\texcommand{\xeuniregisterblockhex} takes two hexadecimal parameters with
Unicode range to be declared.

\begin{texsource}
\xeuniregisterblockhex{0100}{017F}
\end{texsource}

Individual character can be declared using the \texcommand{\xeuniregisterchar} command:

\begin{texsource}
\xeuniregisterchar{"1F00}
\end{texsource}

In contrast to \texcommand{\xeuniregisterblockhex}, it uses decimal numbers by
default, so it is necessary to use the \texttt{"} character in front of
a hexadecimal number.

\begin{warning}
It is possible to run into issues because of the way how Xe\LaTeX\ Unicode
support works. Common problem is filename support, for example in included
graphics. In general, it is better to avoid such filenames. If it is not possible, try to use the \texcommand{\detokenize} command.
\begin{texsource}
  \includegraphics{\detokenize{háček.jpg}}
\end{texsource}
\end{warning}

\section{Colors}

Information about the \texcommand{\color} command:
\url{https://tex.stackexchange.com/a/195677/2891}.
Example of possible configuration for the \texcommand{\color} command: 
\url{https://tex.stackexchange.com/q/470179/2891}.

Example of extracting color information to the CSS and custom color environment support:
\url{https://tex.stackexchange.com/a/422281/2891}. Extracting of color information to the HTML attributes:
\url{https://tex.stackexchange.com/a/390151/2891}.



\section{Graphics and Pictures}

\subsection{PDF support}
\DocConfigure{PdfConvert}{}{}\EndDoc
\DocConfigure{Ghostscropt}{name of the executable for GhostScript}\EndDoc


\section{TikZ }

Animations using Animate package: \url{https://tex.stackexchange.com/a/404600/2891}

Issues with drivers: \url{https://tex.stackexchange.com/a/471460/2891}.
\section{Pstricks}

\section{Math}
\subsection{Default math handling}
\subsection{MathML}

\mathml\ is a XML markup for the math encoding. It is supported in many
aplications including OpenOffice Writer or Firefox web browser. 
The advantage over use of images % Todo: write about advantages of MathML.

The \mathml\ code produced by \texfourht\ may contain some issues. For example,
one common issue may happen when the math contain unmatched delimiters:

\begin{texsource}
 Mail address: $\lparen$hello@world.com$\rparen$
\end{texsource}

In such cases, the \option{matml-} may help. 

It is also advisable to always use \extension{common\_domfilters}
\term{make4ht} extension (see section \ref{sec:make4ht-extensions} for more
information about \term{make4ht} extensions), as it fixes some common \mathml\
errors that cannot be easily fixed on the \TeX\ level.


If you want to add original \LaTeX\ source as an annotation to your MathML formulas,
you can try a configuration from this \href{https://tex.stackexchange.com/a/637910/2891}{this answer}.

Add information about the \url{https://github.com/pshihn/math-ml} - it adds
support for MathML in all modern web browsers with HTML 5.


\subsection{MathJax}
\texfourht\ supports MathJax, library for math rendering in HTML documents. 
 It supports two modes -- \LaTeX\ math and \mathml.

The \term{MathJax} processing can be required using the \option{mathjax} option.

The address of the \term{MathJax} script and its configuration string can be
specified in the \configuration{MathjaxSource} configuration. Default value of this configuration is:

\begin{texsource}
\Configure{MathjaxSource}
{https://cdn.jsdelivr.net/npm/mathjax@3/es5/tex-chtml-full.js}
\end{texsource}

\paragraph{\LaTeX\ mode}

In the \LaTeX math mode, \TeX\ macros used in the math mode are preserved in
the output HTML document. They are parsed and rendered by MathJax when the
document is displayed by a web browser. The downside of this mode is that
commands unknown to MathJax must be configured in a special configuration for
MathJax. Cross-references to equations and other numbered math environments
don't work out of the box. You can try the Lua scripts proposed in 
\href{https://tex.stackexchange.com/a/597913/2891}{this post on TeX.sx} as
a workaround.

By default, inline and display math, as well as math environments, are kept as
raw LaTeX code in the \HTML\ output. 

The additional configuration for \term{MathJax} can be provided in the
\configuration{MathJaxConfig} configuration.
The following example provides support for some custom \LaTeX\ macros.

\begin{texsource}
\Preamble{xhtml}
\Configure{MathJaxConfig}{{
    tex: {
      tags: "ams",
      \detokenize{%
      macros: {
        sc: "\\small\\rm",
        sl: "\\it",
      }
  }
}
}}
\begin{document}
\EndPreamble

\end{texsource}


The configuration needs to be passed as a JavaScript object, this means that
you need to use extra \verb|{}| brackets.
The \texcommand{\detokenize} macro is used to avoid issues with backslash
characters used in the macro definitions. Backslashes must be doubled in the
JavaScript strings. Contents of this configuration are already enclosed in the
\texcommand{\HCode} command, so you cannot use it in this configuration.

If you want to define macros with parameters, you may run to issues with the \# character, used for parameters.
You need to change the catcode of this character to letter before the \configuration{MathJaxConfig} configuration:

\begin{texsource}
\Preamble{xhtml}
% change catcode of # to letter
% in order to support #1 in custom
% MathJax macros
\catcode`\#=11
\Configure{MathJaxConfig}{{
    tex: {
      \detokenize{%
      macros: { 
        % our sample macro takes a parameter
        hello: ["\\sqrt{#1}",1],
      }
  }
}
}}
% don't forget to change the catcode 
% back to the original value
\catcode`\#=6
\begin{document}
\EndPreamble
\end{texsource}

\paragraph{Custom environments in MathJax.}\label{sec:mathjax_env}

If you want to add support for a new environment, you can use the \verb|\VerbMath| command, and you will
also need to pass suitable configuration to MathJax. For example for the following TeX file:

\begin{texsource}
\documentclass[12pt]{book}    
\usepackage{amsmath} 
\usepackage{breqn}

\begin{document}    
\begin{dgroup*}
\begin{dmath*}
     A = B
   \end{dmath*}     %error is around here           
\end{dgroup*}
\end{document}
\end{texsource}

Can be configured using this configuration file:

\begin{texsource}
\Preamble{xhtml} 
\Configure{MathJaxConfig}{{ 
    tex: { 
      tags: "ams", 
      \detokenize{% 
        environments: {
          "dgroup*": ["", ""],
          "dmath*": ["", ""],
        } 
      } 
    } 
}} 
\VerbMath{dgroup*}
\begin{document} 
\EndPreamble
\end{texsource}

The \verb|\VerbMath| command can be used just for the \verb|dgroup*| environment, as it will pass the
whole contents, including the \verb|dgroup*| environment, to the HTML file. But you will need to 
provide MathJax configuration for both of these environments, as they are not supported by default.



\paragraph{\mathml\ mode}

Math is converted to \mathml\ by \texfourht, MathJax then renders it. Custom
commands and cross-references work in this mode.

The \mathml\ MathJax mode can be required using the \option{mathml,mathjax} option.

\paragraph{Table of contents issues}

Some math commands may cause issues when they are used in section titles in the MathJax mode. 
This can be fixed using the \texcommand{\fixmathjaxtoc} command:

\begin{texsource}
\fixmathjaxtoc\int
\end{texsource}


\section{Bibliographies}
\section{Indexing}

\section{OpenDocument Format}
The OpenDocument Format uses XML configuration file for document styling. To
declare new document style, \texfourht\ provides command
\texcommand{\NewConfigureOO}. The declared style then needs to be configured using command \texcommand{\ConfigureOO}.

Usage of these commands can be illustrated by the following example:

\begin{texsource}
  \Configure{SectionTitleTest}{\ifvmode\IgnorePar\fi\EndP\HCode{<text:p text:style-name="section-title">}}{\HCode{</text:p>}}

\NewConfigureOO{section-title}
\ConfigureOO{section-title}{<style:style style:name="section-title" style:family="paragraph" style:class="text">
      <style:text-properties style:text-underline-style="solid"
       style:text-underline-width="auto"
       style:text-underline-color="font-color"
       />
</style:style>}
\end{texsource}

Document style  \term{section-title} had been declared in this example. The
\xml\ code  for this style can be used without the \texcommand{\HCode} command
in \texcommand{\ConfigureOO}.

The configuration \term{SectionTitleTest} inserts element \verb|<text:p>|. The
\verb|text:style-name| corresponds to attribute \verb| style:name| of
\verb|style:style| element in the style configuration.
%Information about \texcommand{\NewConfigureOO} and styling and how to correctly use text styles (using configuration for HtmlPar)

%\url{https://tex.stackexchange.com/a/471283/2891}, \url{https://tex.stackexchange.com/a/100287/2891}

\subsection{Extra Configurations for OpenDocument Format}


To use default ODF styles for sectioning commands, use the following configurations:
% todo: explain better
\begin{texsource}
\Configure{Heading-2}{Heading 1}
\Configure{Heading-3}{Heading 2}
\end{texsource}



\chapter{Make4ht Build Files}
\label{sec:make4ht-build-files}
\section{Commands execution}
\section{Filters}

Some samples:

\begin{itemize}
  \item Render math by Mathjax during tex4ht compilation \url{https://tex.stackexchange.com/a/402159/2891}
\end{itemize}
\section{Image conversion}

\chapter{FAQ and Troubleshooting}
\section{Math issues}

\subsection{Problems with \term{\mathml}}

\begin{issue}{Single delimiters}
  Use of single delimiters like \texcommand{$\langle$} may result in invalid
  \mathml\ code. \texfourht\ can try to fix that using the \option{mathml-}
  option.
\end{issue}



\chapter{For Developers}
% \section{Introduction}

This chapter deals with \texfourht\ development. It starts with a basic
tutorial for a new package support, shows commands useful in the process,
different types of \texfourht\ configuration files, and the syntax and structure of 
literate source files.

\section{Tutorial: Basic Support For a New Package}

In this tutorial, we will try to show how to provide \texfourht\ support for a
simple \LaTeX\ package. 

% from https://tex.stackexchange.com/a/402283/2891
\texfourht\ tries to load a special \file{.4ht} file for each package loaded
by \LaTeX. This special file can contain modifications to commands provided by the package, like 
redefinitions of macros that cause clashes between the package and \texfourht, and most importantly
they insert special macros, called hooks, that are then used to include the output format tags.

Let's say that you have a custom package, called \file{mynote.sty}

\begin{texsource}
\newcommand\notetitle{Note:~}
\newcommand\note[1]{\textbf{\notetitle}#1}
\newcommand\highlight[1]{\textbf{#1}}
\endinput
\end{texsource}

It defines two user commands, \cmd{note} and \cmd{highlight}. 
They can be used in the following way:


\begin{texsource}
\documentclass{article}
\usepackage{mynote}
\begin{document}
\note{This is a note}

Try to highlight \highlight{something}.
\end{document}
\end{texsource}

\texfourht\ produces usable output for both of these commands out of the box, 
thanks to the support for \TeX\ fonts. But you may want to use custom HTML 
tags instead. To achieve that, you need to insert special commands, called hooks 
in \texfourht, to package commands. These hooks can be then configured to
insert tags in the output format.

To introduce hooks, you need to create a hook seeding configuration file for the package,
called \file{<name>.4ht}. For example, to seed hooks for the \file{mynote.sty} package, create file
\file{mynote.4ht}:

\begin{texsource}
\NewConfigure{note}{3}

% Use \HLet when you want to completely redefine a command
\def\:tempa#1{\a:note\notetitle\b:note~#1\c:note}
\HLet\note\:tempa

\NewConfigure{highlight}{2}
\pend:defI\highlight{\a:highlight}
\append:defI\highlight{\b:highlight}

\Hinput{mynote}
\endinput
\end{texsource}

There is several things to note. First is that the \verb|:| character 
can be included as a part of a command name in \file{.4ht} files. It is similar
to use of the \verb|@| character in \LaTeX\ packages. It allows us to 
create command names that don't clash with other command names.

The hooks are created using the \cmd{NewConfigure} command. They can be
later filled with the \cmd{Configure} command. To have an effect, hooks
must be inserted to the existing commands. There are two ways how to do that.
For simpler commands, where we want to insert tags only before and after 
the contents produced by the patched command, we can use the \cmd{pend:def<X>} and 
\cmd{append:def<X>} commands, where the \verb|<X>| is a roman number of parameters
that the patched command expects. In this example, it expects one parameter, 
so we can use the \cmd{pend:defI} command. For commands without parameters, use 
\cmd{pend:def}.

Of course, you can also insert hooks using other mechanisms, for example using
\LaTeX's hook system:

\begin{texsource}
\AddToHook{cmd/highlight/before}{\a:highlight}
\AddToHook{cmd/highlight/after}{\b:highlight}
\end{texsource}

The second way for hook insertion, useful for commands where we want to insert
tags also inside it's contents, is to use the \cmd{HLet} command. It is a
variant of the \cmd{let} command.  In contrast to \cmd{let}, it saves the
original command as \cmd{o:<command name>:}.  Commands redefined by \cmd{HLet}
also support the \cmd{Picture} command, where the original version of the
command will be used. This way, pictures will produce the same result as they
would produce in the PDF mode.

In our example, we redefined the \cmd{note} command to use a hook between note title
and note text. This enables us to style both the title and the text differently.


The configuration file for our hooks could look like this:

\begin{texsource}
\Preamble{xhtml}
\Configure{note}
{\ifvmode\IgnorePar\fi\EndP\HCode{<div class="note"><span class="notetitle">}}
{\HCode{</span><span class="notebody">}}
{\HCode{</span></div>}}
\Css{.notetitle{font-weight: bold;}}

\Configure{highlight}{\HCode{<span class="highlight">}\NoFonts}{\EndNoFonts\HCode{</span>}}
\Css{.highlight{font-weight:bold;}}
\begin{document}
\EndPreamble
\end{texsource}

As the \cmd{note} command should be used on it's own paragraph, we need to 
fix paragraph closing. See the \namerefpage{sec:paragraph_handling} section for
more information about this issue. More details about configuration files and configurations are
in section \namerefpage{sec:private-configuration}.

The HTML code produced by our configuration looks like this:

\begin{htmlsource}
<div class='note'><span class='notetitle'>Note: </span><span class='notebody'> This is a note</span></div>
<!--  l. 6  --><p class='indent'>   Try to highlight <span class='highlight'>something</span>.
</p>
\end{htmlsource}


\section{Commands Usable in the \file{.4ht} files}

\DocCommand{NewConfigure}\marg{name}\marg{number of defined hooks}

This command defines macros with an alphabetic prefix in the form of 
\cmd{a:name} \ldots \cmd{i:name}, depending on the number of defined hooks.
The maximum number is 9.

\begin{texsource}
\NewConfigure{try}{2}
\def\try#1{\a:try#1\b:try}
\Configure{try}{* }{}  
\try{ho} 
% produces "* ho"
\end{texsource}

\DocCommand{NewConfigure}\marg{name}\oarg{number or parameters}\marg{code}

Variant of \cmd{NewConfigure} that doesn't define hooks with 
alphabetic prefixes, but it passes argumens of \cmd{Configure}
as \TeX\ arguments. See this exampe:

\begin{texsource}
\NewConfigure{try}[2]{\def\hookI{#1}\def\hookII{#2}}
\def\try#1{\hookI#1\hookII}
\Configure{try}{* }{}  
\try{ho} 
% produces "* ho"
\end{texsource}

When you use \texcommand{\Configure{try}}, it defines \cmd{hookI} and \cmd{hookII}
commands. They can be then used in the redefined \cmd{try} command.

\DocCommand{HLet}\marg{Redefined command name}\marg{new command}

Variant of \cmd{let} that saves the original command under \cmd{\o:<name>:} name.
It can detect use of the redefined command inside picture. In such case, it will use
the original command to produce correct visual result in the picture.

\begin{texsource}
\NewConfigure{note}{3}
\def\:tempa#1{\a:note note:\b:note~#1\c:note}
\HLet\note\:tempa
\Configure{note}{*}{*}{*}
\note{hello}
% produces: "* note:* hello*
\end{texsource}

\DocCommand{HRestore}\marg{command name}

Restore command redefined using \cmd{HLet} to it's original content.

\DocCommand{pend:def<X>}\marg{redefined command}\marg{code to be inserted at the begin}

\DocCommand{append:def<X>}\marg{redefined command}\marg{code to be inserted at the end}

These two commands inserts code before and after a redefined command. There are several
versions of these commands, depending on the number of parameters that the redefined 
command expects. Number of parameters as roman number replaces the \verb|<X>| placeholder. 

Up to three parameters are supported.


\begin{texsource}
\newcommand\bar{xxx}
\pend:def\bar{*}
\append:def\bar{*}
\bar
% produces: "*xxx*"
\newcommand\foo[2]{#1, #2}
\pend:defII\foo{*}
\append:defII\foo{*}
\foo{a}{b}
% produces "*a, b*"

\end{texsource}

\DocCommand{:CheckOption}\marg{option name}
\DocCommand{if:Option}

Support for custom options. The \cmd{:CheckOption} checks if the given option
is active, and \cmd{if:Option} conditional then run true or false branch.

\begin{texsource}
\:CheckOption{info}\if:Option
... \else ...
\fi
\end{texsource}
      
\section{Two types of .4ht files}

% text from the old documentation:
% https://tug.org/tex4ht/doc/mn11.html#QQ1-11-66

The compilation starts by opening tex4ht.sty and loading a fraction of its code.
The main purpose of this phase is to request the loading of the system at a
later time (for instance, upon reaching \texcommand{\begin{document}}). The motivation for
the late loading is to allow TeX4ht to collect as much information as possible
about the environment requested by the source file, and help the system reshape
that environment with minimal interference from elsewhere.

The system uses two kinds of (4ht) configuration files. The files of the first
kind mainly seed hooks into the macros loaded by the source file (for instance,
\file{latex.4ht}, \file{fontmath.4ht}, and \file{article.4ht}).
The files of the second kind mainly
attach meaning to the hooks (for instance, \file{html4.4ht}, \file{unicode.4ht}, and
\file{mathml.4ht}).

Different source files may request the loading of different style files and in
different orders. The hook seeding files are loaded in response to the loading
of the style files, and in a compatible order. Since the different style files
may redefine the syntax and semantics of macros, \texfourht\ follows a similar route
of defining and redefining the hooks and their meanings.


% For instance, the mzlatex command
% refers to the mozilla option name of tex4ht.4ht, and the oolatex command refers
% to the ooffice option name. 

\subsection{Custom output formats}

The meaning attaching files are normally requested through option names
introduced in the \file{tex4ht.4ht} system file. It defines options for all output formats
supported by \texfourht. For instance, \option{html5}, \option{ooffice} for the ODT output,
\option{tei}, and so on. 

These options are passed to \texfourht\ by \makefourht according to the \texttt{--format} 
command line parameter, but you can pass them also yourself. 

The user may add option names, and redefine old ones, within a new file named \file{tex4ht.usr}.

A new tex4ht.usr file should group references to \file{*.4ht} configuration files
under arbitrarily chosen option names. For that purpose, \cmd{Configure} commands
similar to those provided in \file{tex4ht.4ht} should be employed. 
These are particularly useful if you use custom packages that are not included in TeX distributions
and thus aren't supported by \texfourht.

You can place your custom \file{.4ht} files or \file{tex4ht.usr} in your local TEXMF tree, for instance
in \shellcmd{~/texmf/tex/latex/my4htfiles}.

\subsubsection{Example}

Let's say that you have a custom package \file{mypackage.sty}:

\begin{texsource}
\newcommand\mycommand[1]{Hello #1}
\endinput
\end{texsource}

This can be configured using the following configuration file, \file{mypackage.4ht}:

\begin{texsource}
\NewConfigure{mycommand}{2}
\pend:defI\mycommand{\a:mycommand}
\append:defI\mycommand{\b:mycommand}
\Hinput{mypackage}
\endinput
\end{texsource}

Important command in this listing is \texcommand{\Hinput{mypackage}}. The \cmd{Hinput} expects
package name as it's argument. It registers it for the latter processing in the output format files.

Here is a custom output format file \file{sample.4ht}:

\begin{texsource}
\exit:ifnot{mypackage} 
 
%%%%%%%%%%%%%%%%%%%%%%%%%%%%%%%%%%%%%%%%%%%%%%%%%%%%%%%%%%%%%%%% 
\ConfigureHinput{mypackage} 
%%%%%%%%%%%%%%%%%%%%%%%%%%%%%%%%%%%%%%%%%%%%%%%%%%%%%%%%%%%%%%%% 
\Configure{mycommand}{\HCode{<span class="mycommand">}}{\HCode{</span>}}

%%%%%%%%%%%%%%%%%%%%%%%%%%%%%%%%%%%%%%%%%%%%%%%%%%%%%%%%%%%%%%%% 
\endinput\empty\empty\empty\empty\empty\empty
%%%%%%%%%%%%%%%%%%%%%%%%%%%%%%%%%%%%%%%%%%%%%%%%%%%%%%%%%%%%%%%% 

\endinput 
\end{texsource}

The \cmd{exit:ifnot} command takes comma separated list of packages supported by the
output format file. This stops it's loading if the currently processed package doesn't 
have configurations in the file. 

The configuration for the package is placed between \cmd{ConfigureHinput} and
\texcommand{\endinput\empty\empty\empty\empty\empty\empty}. 

To request the custom output format file, we need to add it to \file{tex4ht.usr}. Here
is an example that adds a new option \option{myhtml5}. It is based on the code for 
the \option{html5} option from \file{tex4ht.4ht}:

\begin{texsource}
\Configure{myhtml5}{%
   \:CheckOption{info}\if:Option
               \Hinclude[*]{infoht4.4ht}\fi
   \:CheckOption{info}\if:Option
               \Hinclude[*]{infomml.4ht}\fi
   \Hinclude[*]{html4.4ht}%
   \Hinclude[*]{unicode.4ht}%
   \:CheckOption{mathml}\if:Option%
   \else\:CheckOption{mathml-}\fi%
   \if:Option%
      \Hinclude[*]{mathml.4ht}%
      \Hinclude[*]{html-mml.4ht}%
   \else
      \Hinclude[*]{html4-math.4ht}%
   \fi
   \:CheckOption{svg}%
             \if:Option \else\:CheckOption{svg-}\fi
             \if:Option \else\:CheckOption{svg-obj}\fi
             \if:Option \else\:CheckOption{svg-inline}\fi
             \if:Option \Hinclude[*]{svg-option.4ht}%
                        \:CheckOption{info}\if:Option \Hinclude[*]{infosvg.4ht}\fi
             \fi
   \Hinclude[*]{html5.4ht}%
   \Hinclude[*]{sample.4ht}
}
\end{texsource}

It uses the \cmd{:CheckOption} commands to detect additional options, which results 
in conditional loading of various output format files using the \cmd{Hinclude} command. 
Our custom output file \file{sample.4ht} is placed at the end.

You can then require the custom output format using this command:

\begin{shellcommand}
$ make4ht filename.tex "myhtml5"
\end{shellcommand}





% \subsection{Inserting configurable hooks for packages}



% \subsection{Configure the hooks in output format configuration files}

\section{\texfourht\ literate sources}

To add a proper support for a new package, it is necessary to edit the 
\texfourht\ literate sources. All distributed \texfourht\ files, including
\file{tex4ht.sty} and all \file{.4ht} files, are generated from these literate
programming files. It is also the reason why the generated files don't contain
much comments, these are in the sources. If you want to understand
how \texfourht\ works, it is necessary to read them.

The source files are available in the \href{https://puszcza.gnu.org.ua/projects/tex4ht/}{\texfourht\ source repository}.
You can retrieve them using a SVN client. 

\begin{shellcommand}
$ svn checkout https://svn.gnu.org.ua/sources/tex4ht/
$ cd tex4ht/trunk/lit/
\end{shellcommand}


The configurable hooks for all packages are contained by the \file{tex4ht-4ht.tex} file.
Configurations of these hooks is placed in the output format configuration files.
The most common output format is \HTML, which can be configured in \file{tex4ht-html4.tex}, or 
\file{tex4ht-html5.tex} if \HTMLV\ features are used. You can also update sources for other output
formats, for example \file{tex4th-ooffice.tex} for the ODT format, or \file{tex4ht-tei.tex} for TEI.
The sources of the \file{tex4ht.sty} package are available in \file{tex4ht-sty.tex}.

To compile all literate sources, run the \shellcmd{make} command. You will need basic UNIX utilities 
for this to succeed, as well as \shellcmd{m4} and \shellcmd{javac}. You can also compile particular source
files. Most of them can be compiled using \LaTeX, but some of them, for example \file{tex4ht-4ht.tex}, needs
to be compiled using \shellcmd{etex}.

\subsection{How to add support for a package to the \texfourht\ literate sources}

Given following package \file{sample.sty}:

\begin{texsource}
\ProvidesPackage{sample}
\newcommand\hello{hello}
\endinput
\end{texsource}

This simple package defines command \texcommand{\hello}, which simply prints the word \enquote{hello} when used in a document.

Let's say that we want to insert some \HTML\ tags before and after the text content printed by the command.

Basic template for \file{tex4ht-4ht.tex}:

% examples/basicpackage/sample.4ht
\begin{texsource}
\<sample.4ht\><<<
% sample.4ht (|version), generated from |jobname.tex
% Copyright 2017 TeX Users Group
|<TeX4ht license text|>
\NewConfigure{hello}{2}
\pend:def\hello{\a:hello}
\append:def\hello{\b:hello}
\Hinput{sample}
\endinput
>>> \AddFile{9}{sample}
\end{texsource}

Configuration for each package must follow this basic template. The \ProTeX\ system is used as system for literate programming.

The \verb|\<name\><<<code>>>| block defines new macro which can be then called using \texcommand{|<name|>}. The license text
is included in this way in the example. The instruction to generate the \file{.4ht} file is given in the 
command \texcommand{\AddFile{9}{sample}} after the block definition. The first argument to \cmd{AddFile} is an arbitrary number.


Each package configuration  must include \texcommand{\Hinput{packagename}}, in order to load the configurations for the package.

The command \texcommand{\NewConfigure{hello}{2}} declares new configuration \texttt{hello}, with two configurable hooks. 
These hooks are named  \texcommand{\a:hello} and \texcommand{\b:hello}. The hooks must be inserted into the 
\texcommand{\hello}, which can be easily done using the \texcommand{\pend:def} and \texcommand{\append:def} commands. These
commands can insert code  at the beginning, respective at the end of the redefined command.

The package name must be also included in the \file{mktex4ht-cnf.tex} file. This file is used in the generation of the 

\begin{texsource}
\AddFile{9}{sample}
\end{texsource}

You can place configuration for \HTML\ to the \file{tex4ht-html4.tex} file:

% examples/basicpackage/config.cfg
\begin{texsource}
\<configure html4 sample\><<<
\Configure{hello}{\HCode{<span class="hello">}}{\HCode{</span>}}
\Css{.hello{color:red;}}
>>>
\end{texsource}

The \texcommand{\<configure html4 packagename\>} block will produce code that 
detects use of the package \file{packagename}. It then loads configurations
for the package.


The \file{.4ht} files can be generated simply using the \shellcmd{make} command.

The following sample \TeX\ file:

% examples/basicpackage/hello.tex
\begin{texsource}
\documentclass{article}
\usepackage{sample}
\begin{document}
  \hello\ world.
\end{document}
\end{texsource}

Produces a following \HTML\ code:

\begin{htmlsource}
<!--l. 4--><p class="noindent" >
<span class="hello">hello</span> world. 
</p> 
\end{htmlsource}




\section{ProTeX}


The literate programming system used in the previous section is called ProTeX. We should discuss some main ideas behind this system.

% copied from
% https://www.slac.stanford.edu/comp/unix/package/tex/tex4ht/mn2.html - it
% seems like an older version of documentation which contains some information later ommited

Literate programming is a discipline that promotes the writing of programs the
way one explains them to human beings. ProTeX is a literate programming system
fully implemented in terms of TeX, and it is compatible with LaTeX and other
TeX-base systems. TeX4ht, and ProTeX itself, are examples of applications
written in ProTeX.


\begin{texsource}
\input ProTex.sty
\AlProTex{extension,<<<>>>,list,title,escape-character}
\<title\><<<
code fragment
>>>  
|<title|>
\OutputCode\<...\> 
\end{texsource}

Some explanation:

\begin{texsource}
\input ProTex.sty
\AlProTex{extension,<<<>>>,list,title,escape-character}
\end{texsource}

The escape-character stands for `, @, |, or ?. If omitted, it stands for \verb'|'. 

\begin{texsource}
\<title\><<<
code fragment
>>>

\end{texsource}

This structure provides names to code fragments (the fragments should not be too large in size).


\begin{texsource}
 |<title|>
 \end{texsource}

 This command acts as a place holder for the code segment associated to the title (\texttt{|} stands for the escape character). 

\begin{texsource}
   \OutputCode\<...\>
 \end{texsource}

This command creates a file for the code whose root node is specified.





\chapter{Glossary}
\chapter{Bibliography}
\chapter{Index}
\chapter{Changes}

\begin{description}
  \item[2022/03/25] -- added documentation for the \cmd{Configure} command, examples for CSS.
  \item[2022/03/24] -- added Rmarkdown example.
  \item[2022/03/23] -- updated FAQ, added options section to tutorial.
  \item[2022/03/22] -- added supported output formats, updated \fourhtsty\ options instructions, continuing in tutorial rewrite.
  \item[2022/03/21] -- added link to example of \href{https://tex.stackexchange.com/a/637910/2891}{MathML code with \LaTeX\ annotations}.
  \item[2022/03/18] -- added \texcommand{\namerefpage} command. It prints crosslink title and in the PDF mode also page number.
  \item[2022/03/17] -- updated tutorial to use custom code environments, removed some obsolete information.
  \item[2022/01/26] -- added description for options that break HTML pages to subpages based on sectioning level.
  \item[2022/01/12] -- updated information about alternative text in Graphics HowTo.
  \item[2021/10/25] -- added example how to handle custom environments in MathJax output -- \nameref{sec:mathjax_env}.
  \item[2021/06/29] -- updated introduction to \nameref{sec:private-configuration}.
  \item[2021/06/13] -- added information  about \nameref{faq:compilation_errors} (\ref{faq:compilation_errors}) and \nameref{faq:dom_processing} (\ref{faq:dom_processing}).
  \item[2021/06/13] -- started tracking of changes.
\end{description}


\end{document}
