\documentclass{book}
\usepackage{url}
\usepackage{xcolor}
\usepackage{array}

% \usepackage[T1]{fontenc}
\usepackage{upquote}
\usepackage{textcomp}
\usepackage{tex4ht-styles}



\usepackage{glossaries}
\title{TeX4ht Documentation}

\author{by TeX4ht Project}

\begin{document}

\maketitle

\tableofcontents

\chapter{Introduction}

\chapter{Basic tutorial}
\chapter{Basic usage}
\section{Calling commands}
\section{Output Formats}
\section{Command Line Options}

% the list of options has been copied from the CVR's blog
% http://cvr.cc/?p=504



Following is an incomplete list of options that can be passed to the \fourhtsty\ package.
These options can be used to modify the compilation process, for example to
select a \term{SVG} format for generated images, to request math environments
to be convert as images or to split sections as separate HTML pages. 

The options may be defined in the \fourhtfile\ files and may depend on the
output format, so it is not feasible to provide their full list. Most of the
following options work only in the \HTML\ output.

There are several ways how to pass the options to \texfourht. The
non-recommended way is to pass them as options to \fourhtsty\ using
\texcommand{\usepackage} command in the \TeX\ file. 

Better ways don't require modifications of the \TeX\ files.
It is possible to pass the options to the calling script, as an command line
argument next to the filename:
% it is run from the command line. These can also be provided as options when \texfourht
% package is loaded in a \LaTeX\ document with the default usepackage command or to
% the \verb|\Preamble| command in the custom config file

\begin{shellcommand}
make4ht filename.tex "fn-in"
\end{shellcommand}

For more information about calling scripts see the section \ref{sec:calling-commands}.

It is also possible to pass options in the \texcommand{\Preamble} command in a \cfgfile.

\begin{texsource}
\Preamble{fn-in}
...
\begin{document}
\EndPreamble
\end{texsource}



\section{List of options}

% \begin{tabular}{>{\ttfamily}p{8em} l} 
%   -css & to ignore CSS code, use command line option -css. \\
%   -xtpipes & to avoid xtpipes post-processing the output. This might be useful for docbook XML output.\\
%   0 & pagination shall be obtained through the option 0 or 1, at locations marked with PageBreak.\\
%   1, 2, 3, 4, 5, 6, 7& for automatic sectioning pagination (to break at various section levels), use the appropriate command line option 1, 2, 3, 4, 5, 6, 7.
\begingroup
\catcode`\#=11 \catcode`\^=11 \catcode`\_=11


\begin{description}

\item[-css] to ignore \css\ code, use command line option \verb=-css=.

\item[-xtpipes] to avoid \verb=xtpipes= post-processing the
  output. This might be useful for docbook \xml\ output.

  % \item[/bib]
  % \item[/obeylines]
  % \item[0.0]

\item[0] pagination shall be obtained through the option \verb=0= or
  \verb=1=, at locations marked with \verb=\PageBreak=.

\item[1, 2, 3, 4, 5, 6, 7] for automatic sectioning pagination (to
  break at various section levels), use the appropriate command line
  option \verb=1, 2, 3, 4,= \verb=5, 6, 7=.

\item[DOCTYPE] to request a \verb=DOCTYPE= declaration, use the
  command line option \verb=DOCTYPE=.

\item[Gin-dim] for key dimensions of the graphic, try this option.

\item[Gin-dim+] for key dimensions when the bounding box is not
  available.

\item[NoFonts] to ignore \css\ font decoration.

\item[PMath] Option to choose positioned math. Example: 
  \verb=\def\({\PMath$}=;\allowbreak \verb=\def\){$\EndPMath}=;
  \verb=\def\[{\PMath$$}=; \verb=\def\]{$$\EndPMath}=.

\item[RL2LR] to reverse the direction of RL sentences.

%\item[ShowFont]

\item[TocLink] option to request links from the tables of contents.
  
\item[\textasciicircum 13] option for active superscript character.

\item[\_13] option for active subscript character.

\item[accent-] This option is available only together with
  \option{new-accents}. It produces pictures for some math accents.

%\item[base]

\item[bib-] for degraded bibliography friendlier for conversion to
  \verb=.doc=.

\item[bibtex2] Option \verb=bibtex2= requires compilation of
  \verb=\jobname j.aux= with bibtex.

%\item[broken-index]

\item[charset] for alternate character set, use the command line
  option \verb+charset="..."+ (e.g., \verb+charset="utf8"+).

%\item[core]

\item[css-in] the inline \css\ code will be extracted from the input of
  the previous compilation, so an extra compilaion might be needed for
  this option to make it effective.

\item[css2] for \css2 code.

% \item[css]
% \item[debug-]
% \item[debug]
% \item[draw]
% \item[dtd]

\item[early\textasciicircum] for default catcode of superscript in the
  \verb=\Preamble=.

\item[early\_] for default catcode of subscript in the
  \verb=\Preamble=.

%\item[edit]

\item[endnotes] for end notes instead of footnotes, use this option.

%\item[enum]

\item[enumerate+] for enumerated list elements that keep the list couter value. This
  will use the description list like \verb=<dt>...</dt>= for the list
  counter.

\item[enumerate-] for enumerated list element's \verb=<li>='s with
  value attributes, use this command line option. This will be an
  ordered list with the value of list counter provided as an attribute
  namely, \verb=value= of the \verb=<li>= element.

%\item[family]
\item[fancylogo] try to visually emulate \verb|\TeX| and \verb|\LaTeX| logos.

\item[fn-in] for inline footnotes use this option.

\item[fn-out] for offline footnotes.

\item[fonts] for tracing \latex\ font commands, use this command line
  option.

\item[fonts+] for marking of the base font, use this option.

\item[font] for adjusted font size, use the command line option
  \verb+font=...+ (e.g., font=-2).

\item[frames-] for frames support. \verb=frames= is also valid option
  for frames support.

\item[frames-fn] for content, \chfont{TOC}\ and footnotes in
  three frames.

\item[frames] for \chfont{TOC}\ and content in two frames.

%\item[fussy]

\item[gif] for bitmaps of pictures in \verb=.gif= format, use this
  option.

\item[graphics-] if the included graphics are of degraded quality, try
  the command line options \verb=graphics-num= or \verb=graphics-=.
  The \verb=num= should provide the density of pixels in the bitmaps
  (e.g., 110).

%\item[graphics-dim]

\item[hidden-ref] option to hide clickable index and bibliography
  references.

% \item[hooks++]
% \item[hooks+]
% \item[hooks]
% \item[hshow]
% \item[htm3]
% \item[htm4]
% \item[htm5]
% \item[htm]

\item[html+] for stricter \HTML\ code.

%\item[html]

\item[imgdir] for addressing images in a subdirectory, use the option
  \verb=\imgdir:.../=.

\item[image-maps] for \verb=image-maps= support.

\item[index] for \emph{n}-column index, use the command line option,
  \verb+index=n+ (e.g., index=2).

\item[info-oo] for extra tracing information while generating open
  office output.

\item[info] for extra information in the \verb=\jobname.log= file.

\item[java] for \verb=java=support.

\item[javahelp] for \verb=JavaHelp= output format, use this command
  line option.

\item[javascript] for \verb=javascript= support.

\item[jh-] for sources failing to produce \xml\ versions of \HTML, try
  this command line option.

%\item[jh1.0]

\item[jpg] for bitmaps of pictures in \verb=.jpg= format, use this
  option.

\item[li-] for enumerated list elements li's with value attributes.

\item[math-] option to use when sources fail to produce clean math
  code.

\item[mathjax] use \term{MathJax} for the math rendering.
%\item[mathaccent-]

\item[mathltx-] option to use when sources fail to produce clean
  \verb=mathltx= code.

\item[mathml-] option to use when sources fail to produce clean
  \mathml code.

\item[mathplayer] for \mathml\ on Internet Explorer + MathPlayer.

\item[minitoc\textless] for mini tocs immediately after the header use the
  command line option, \verb=minitoc<=.

\item[mouseover] for pop ups on mouse over.

\item[new-accents] alternative configurations for accented characters. 

\item[next] for linear cross-links of pages, use this option.

\item[nikud] for Hebrew vowels, use the command line option,
  \verb=nikud=.

\item[no-DOCTYPE] to remove \texttt{DOCTYPE}\space declaration from
  the output.

\item[no-VERSION] to remove \verb+<?xml version="..."?>+ processing
  instruction from the output.

\item[NoFonts] disable ht-fonts processing in the document.

% \item[no-align]
% \item[no-array]
% \item[no-bib]
% \item[no-cases]
\item[no-halign] suppress \texcommand{\halign} redefinition. It doesn't work with the \texcommand{tabular} environment.
% \item[no-matrix]
% \item[no-pmatrix]

\item[no\textasciicircum] for non-active \verb=^= (superscript), use the option
  \verb=no^=.

\item[no\_] for non-active \verb=_= (subscript command), use the
  command line option, \verb=no_=.

\item[no\_\textasciicircum] for both non-active superscript and subscript, use the
  option \verb=no_^=.

\item[nolayers] to remove overlays of slides, use this option.

\item[nominitoc] this will eliminate mini tables of contents from the
  output.

\item[notoc*] for tocs without \verb=*= entries, use this option. The
  \verb=notoc*= option is applicable only to pages that are
  automatically decomposed into separate web pages along section
  divides. It shall be used when \verb=\addcontentsline= instructions
  are present in the sources.

\item[obj-toc] for frames-like object based table of contents, use the
  command line option \verb=obj-toc=.

%\item[old-longtable]

\item[p-width] for width specifications of tabular \verb=p= entries,
  use this option.

\item[p-indent] for indented paragraphs, without blank spaces.

\item[pic-RL] for pictorial RL.

\item[pic-align] for pictorial align environment.

\item[pic-array] for pictorial array.

\item[pic-cases] for pictorial cases environment.

\item[pic-eqalign] for pictorial equalign environment.

\item[pic-eqnarray] for pictorial eqnarray.

\item[pic-equation] for pictorial equations.

\item[pic-fbox] for pictorial or bitmapped fbox'es.

\item[pic-framebox] for bitmap fameboxes.

\item[pic-longtable] for bitmapped longtable.

\item[pic-m+] for pictorial \verb=$...$= and \verb=$$...$$=
  environments with \latex\ alt, use the command line option
  \verb=pic-m+= (not safe).

\item[pic-m] for pictorial \verb=$...$= environments, use the command
  line option \verb=pic-m= (not recommended).

\item[pic-matrix] for pictorial matrix.

% \item[pic-tabbing']

% \item[pic-tabbing]

% \item[pic-table]

\item[pic-tabular] use this option for pictorial tabular.

\item[plain-] for scaled down implimentation.

% \item[pmathml-css]

% \item[pmathml]

% \item[postscript]

\item[prog-ref] for pointers to code files from root fragments, use
  the command line option \verb=prof-ref=. This is for debugging.

\item[refcaption] for links into captions, instead of flat heads, use
  this option.

\item[rl2lr] to reverse the direction of Hebrew words, use this
  option.

\item[sec-filename] for file names derived from section titles, use
  the command line option \verb=sec-filename=.

\item[sections+] for back links to table of contents, use this option.

% \item[sections-]
% \item[settabs-]
% \item[stackrel-]

\item[svg-] for external \svg\ files, try this option.

\item[svg-obj] same as above.

\item[svg] for dvi pictures in \verb=svg= format.

\item[svg-inline] same as the \option{svg}, but the \svg\ files are included in the document body.

\item[tab-eq] for tab-based layout of equation environment, use this
  option.

%\item[th4]

\item[trace-onmo] for mouseover tracing of compilation, use the
  command line option, \verb=trace-onmo=.

% \item[uni-emacspeak]
% \item[uni-html4]
% \item[uniaccents]
% \item[unicode]

% \item[url-]

\item[url-enc] for \chfont{URL}\space encoding within href, use this
  option.  \verb=\Configure{url-encoder}= can be used to fine tune
  encoding.

\item[url-il2-pl] for il2-pl \chfont{URL} encoding.

\item[ver] for vertically stacked frames. Effective when \verb=frames=
  option is requested.

% \item[verify+]
% \item[verify]

\item[xht] for file name extension, \verb=.xht=, use this command line
  option.

\item[xhtml] for \xml\ code, use the command line option, \verb=xml= or
  \verb=xhtml=.

\item[xml] See previous entry.

% \item[xmldtd]

\end{description}
\endgroup


\chapter{Configurations}
\section{tex4ht commands}
\subsection{Low-level \texfourht\ Commands}

\DocCommand{Configure}\marg{name}\marg{arg 1}\ldots\marg{arg n}

The \cmd{Configure} command declares code that is inserted into hooks related to the \textit{name} configuration.
It is possible to define up to nine hooks, so number of arguments is variable.


\DocCommand{ConfigureEnv}\marg{\textless environment name\textgreater}\marg{before env}\marg{after env}
\marg{before-list}\marg{after-list}


\DocCommand{HCode}\marg{output format markup} is a basic command for insertion
of the output format markup, as it's content is not escaped for the \textless{}
and \textgreater.

This command allows only for the expansion of macros, before sending its content to the output. 
The instruction \texcommand{\Hnewline} may be introduced there for requesting line breaks, and the command \texcommand{\#} may be used for the sharp symbol ‘\#’.

\begin{texsource}
First line\HCode{<br />}
second line 

You don't want to include tags directly into the document '<br>'. 
\end{texsource}

\DocCommand{Tg}\verb|<markup>| is a variation of the \cmd{HCode} command that don't require braces, and it does some additional processing.

\DocCommand{ifOption}\marg{name of the option to be checked}\marg{true part}\marg{false part}

This command can be used in the private configuration files to test if a custom option was used

\subsection{Hyperlinks}

\DocCommand{Link}\texttt{[target-file arguments]}\marg{target-loc}\marg{cur-loc} text inside link\cmd{EndLink}

This command requests an anchor that links to \verb|target-file#target-loc|, and marks the current location with the name \verb|‘cur-loc’|.

The component in square brackets \texttt{‘[...]’} is optional when it is empty, 
and the target file need not be mentioned if it is created from the current source file.


\DocCommand{LinkCommand} creates a \cmd{Link}\textit{-like} command. It has variable number for parameters:

\begin{enumerate}
  \item tag name
  \item href-like attribute
  \item name-like attribute
  \item insertion
  \item /, if empty element
  \item replacement for \texcommand{#} character  (ignored if absent)
\end{enumerate}

Example:

\begin{texsource}
\LinkCommand\JSLink{a,\noexpand\jsref,name}
\def\jsref="#1"{href="javascript:window.open('#1')"}

% example of use of the defined command:
\JSLink{a}{}xx\EndJSLink % link to a destination
\Link{}{a}\EndLink       % set link destination (or by \JSLink{}{a}\EndJSLink)
\end{texsource}


\subsection{Paragraph Handling}
\label{sec:paragraph_handling}

Paragraph handling is one of the more complicated areas in \texfourht.
You must handle insertion of tags for paragraph opening and closing,
to prevent wrong nesting of XML tags. Mismatch of tags leads to issues with 
LuaXML post-processing of the generated files, preventing many fixes 
that are necessary for correct conversion.

\DocCommand{HtmlParOff} turns off insertion of paragraph tags in the following text.

\DocCommand{HtmlParOn} enables insertion of paragraphs tags.

\DocCommand{IgnorePar} asks to ignore the next paragraph.
\DocCommand{ShowPar} asks to take into account the following paragraphs.

\DocCommand{IgnoreIndent}  asks to ignore indentation in the next paragraph.
\DocCommand{ShowIndent}    asks to check indentation in the following paragraphs.

\DocCommand{SaveEndP}  saves the content of \cmd{EndP}, and sets it to empty content.
\DocCommand{RecallEndP} resets the content of \cmd{EndP}.

\DocCommand{SaveHtmlPar} saves current configuration for paragraphs. It can be
useful before local declaration of \cmd{Configure}\marg{HtmlPar}.
\DocCommand{RecallHtmlPar} resets configuration for paragraphs to the value saved by
\cmd{SaveHtmlPar}.


The following example adds \htmlcommand{<div>...</div>} tags around contents of the document body.
The \texcommand{\ifvmode\IgnorePar\fi} commands will prevent insertion of the \htmlcommand{<p>} tag 
before \htmlcommand{<div>} if we are in \TeX's vertical mode. The \texcommand{\EndP} closes currently
opened paragraph, if it is opened. The \texcommand{\par\ShowPar} commands start new paragraph
after the inserted \htmlcommand{<div>} tag. It is necessary to explicitly start paragraphs sometimes.

\begin{texsource}
\Configure{@BODY}
{\ifvmode\IgnorePar\fi\EndP
 \HCode{<div>}\par\ShowPar}
\Configure{@/BODY}
{\ifvmode\IgnorePar\fi\EndP
 \HCode{</div>}}
\end{texsource}


\DocConfigure{HtmlPar} {content at the start non-indented paragraphs} 
   {content at the start indented paragraphs}
   {insertion into \cmd{EndP}, at the start of non-indented paragraphs}
   {insertion into \cmd{EndP}, at the start of indented paragraphs} \EndDoc

Example:

\begin{texsource}
\Configure{HtmlPar}
{\EndP\HCode{<p class="indent">}}
{\EndP\HCode{<p class="noindent">}}
{\HCode{</p>}}
{\HCode{</p>}}
\end{texsource}

% https://tex.stackexchange.com/a/66172/2891


\subsection{Logical Document Structure Commands}
I've created an alternative commands to \cmd{HCode} or \cmd{Tg}. 
The idea is to define semantic names for logical blocks of the document, such as titles, authors,
sections etc. HTML elements and attributes can be assigned to these
logical blocks. There are commands for inline and block level elements,
eliminating need for constructs like \texcommand{\ifvmode\IgnorePar\fi\EndP}
etc.

\DocCommand{NewLogicalBlock}\marg{name} create a new logical block.
\DocCommand{SetBlockProperty}\marg{name}\marg{attribute name}\marg{value} set block attribute .
\DocCommand{SetTag}\marg{name}{tag name} assign element name to the logical block
\DocCommand{BlockElementStart}\marg{name}\marg{additional attributes} start block level element.
\DocCommand{BlockElementEnd}\marg{name} close block level element.
\DocCommand{InlineElementStart}\marg{name}\marg{additional attributes} start inline level element.
\DocCommand{InlineElementEnd}\marg{name} close inline level element.


The default tag name for declared logical blocks is \htmlcommand{<span>}. You
need to use the \cmd{SetTag} command only when you want to use a different
element.

The attributes can be set either using \cmd{SetBlockProperty}, or in the second
argument to  \cmd{BlockElementStart} and \cmd{InlineElementStart} commands. First method should be
used for static attributes that don't change, for instance \textit{class}. The second method
is preferred for dynamic attributes, such as \textit{id}, which should be different for 
every element.

The main idea behind this mechanism is to allow easy work with new HTML5
elements and attributes for WAI-ARIA or Schema.org properties. I hope that
this should allow us to make somehow more clear configurations for basic
document building blocks.

Example:


\begin{texsource}
\NewLogicalBlock{textit}
\SetBlockProperty{textit}{class}{textit}

\NewLogicalBlock{maketitle}
\SetTag{maketitle}{header}

\NewLogicalBlock{titlehead}
\SetTag{titlehead}{h1}
\SetBlockProperty{titlehead}{class}{titleHead}

% configure \textit using inline level elements
\Configure{textit}
{\NoFonts\InlineElementStart{textit}{}}
{\InlineElementEnd{textit}\EndNoFonts}

% configure \maketitle using block level elements
\Configure{maketitle}{%
{\Configure{maketitle}{}{}{}{}%
\Tag{TITLE+}{\@title}}
\BlockElementStart{maketitle}{}}
{\BlockElementEnd{maketitle}}
{\NoFonts\BlockElementStart{titlehead}{}}
{\BlockElementEnd{titlehead}\EndNoFonts}
\end{texsource}







\section{Configuration files}
\section{Private Configuration Files}\label{sec:private-configuration}

The leading entry, in the first list of options of the \shellcmd{htlatex}-like
commands, can equal \option{html} or \option{xhtml}. If this is not the case,
the entry is assumed to be the name of a configuration file. The extension
‘cfg’ is assumed for names of configuration files that are listed without their
extension.

A configuration file should take the following form for LaTeX files.

\begin{texsource}
...early definitions...
\Preamble{options}
...definitions...
\begin{document}
...insertions into the header of the html file...
\EndPreamble
\end{texsource}

It is up to the user to decide the distribution of entries between the \texcommand{\Preamble} and the htlatex-like commands.

Example: The command \shellcmd{htlatex myfile "mycfg,2"} requests the
compilation of a file named \file{myfile.tex}, in the presence of a
configuration file named \file{mycfg.cfg}. The configuration file might have the
following content.

\begin{texsource}
\Preamble{html} 
\begin{document} 
  \Css{body { color : red; }} 
\EndPreamble 
\end{texsource}

Notes

\begin{itemize}
  \item Notice that for a LaTeX file the \texcommand{\begin{document}}
    instruction should be present both in the configuration file and the source
    file.

  \item Instructions defined within a source file may be redefined in a
    configuration file. Such a feature enables to keep source files intact for
    compilation to different formats by different tools.
\end{itemize}

For instance, a definition of the form \texcommand{\renewcommand\mycommand{...}} within a
configuration file provided for the following LaTeX source.

\begin{texsource}
\documentclass{...} 
\newcommand\mycommand{...} 
\begin{document} 
Use \mycommand{...} 
\end{document} 
\end{texsource}

\subsection{Configuration file management}

It is possible to reuse common \texfourht\ configurations used in several
configuration files.  They can be inserted in a custom LaTeX package, but there
is one important thing to be aware of. The configuration hooks are inserted to
the patched commands when the compilation reaches the  
\texcommand{\begin{document}} command, so configurations for these hooks
declared before the hook definition have no effect. It is necessary to include
them in the \texcommand{\AtBeginDocument} command.

Sample package, \file{commonconfigurations.sty}:

\begin{texsource}
\ProvidesPackage{commonconfigurations}
\AtBeginDocument{%
\Configure{@HEAD}
{\HCode{<meta name="test" content="test"/>\Hnewline}}
}
\endinput
\end{texsource}

It can be requested in a configuration file using \texcommand{\RequirePackage} command.

\begin{texsource}
\Preamble{xhtml}
\RequirePackage{commonconfigurations}
\begin{document}
\EndPreamble
\end{texsource}



\section{Styling the Document}

\section{Use Webfonts}
\section{Use JavaScript}
\section{Document Navigation}
\section{Tables of Contents}

\section{Sections}
\section{Lists}
\section{Tables}

\section{Fonts}
\subsection{Basic font commands}

Information about the \option{fonts} option and \term{MathML} issues. Example configuration:
\url{https://tex.stackexchange.com/a/416613/2891}
\section{Colors}

\section{Graphics}
\section{TikZ }

Animations using Animate package: \url{https://tex.stackexchange.com/a/404600/2891}
\section{Pstricks}

\section{Math}
\section{MathML}
\section{MathJax}

\section{Bibliographies}
\section{Indexing}

\chapter{Make4ht Build Files}
\section{Calling commands}
\section{Filters}

Some samples:

\begin{itemize}
  \item Render math by Mathjax during tex4ht compilation \url{https://tex.stackexchange.com/a/402159/2891}
\end{itemize}
\section{Image conversion}

\chapter{For developers}

\section{Writing basic support for a new package}
\begin{itemize}
  \item \url{https://tex.stackexchange.com/a/402283/2891}
\end{itemize}

\section{Two types of .4ht files}

\subsection{Inserting configurable hooks for packages}

\subsection{Configure the hooks in output format configuration files}

\section{How to add support for a package to the \texfourht\ literate sources}

To add a proper support for a new package, it is necessary to edit the 
\texfourht\ literate sources. The configurable hooks need to be placed in the \file{tex4ht-4ht.tex},
the configuration of these hooks must be added to the output format configuration files.
The most common output format is \HTML, which can be configured in \file{tex4ht-html4.tex}, or 
\file{tex4ht-html5.tex} if \HTMLV\ features are used. It is also necessary to update the
\file{mktex4ht-cnf.tex}.

\subsection{Example}

Given following package \file{sample.sty}:

\begin{texsource}
\ProvidesPackage{sample}
\newcommand\hello{world}
\endinput
\end{texsource}

This simple package defines command \texcommand{\hello}, which simply prints the word \textit{hello} when used in a document.

Let's say that we want to insert some \HTML\ tags before and after the text content printed by the command.

Basic template for \file{tex4ht-4ht.tex}

% examples/basicpackage/sample.4ht
\begin{texsource}
\<sample.4ht\><<<
% sample.4ht (|version), generated from |jobname.tex
% Copyright 2017 TeX Users Group
|<TeX4ht license text|>
\NewConfigure{hello}{2}
\pend:def\hello{\a:hello}
\append:def\hello{\b:hello}
\Hinput{sample}
>>> \AddFile{9}{sample}
\end{texsource}

Configuration for each package must follow this basic template. The \ProTeX\ system is used as system for literate programming.

The \texcommand{\<name\><<<code>>>} block defines new macro which can be then called using \texcommand{|<name|>}. The license text
is included in the example this way. In order to generate the \file{sample.4ht} file, we need to use \texttt{sample.4ht} as a name
in the code block and command \texcommand{\AddFile{9}{sample}} after the block definition\footnote{I have no idea what the number
in the first parameter means.}.

Each package configuration  must include \texcommand{\Hinput{packagename}}, in order to load the configurations for the package.

The command \texcommand{\NewConfigure{hello}{2}} declares new configuration \texttt{hello}, with two configurable hooks. 
These hooks are named  \texcommand{\a:hello} and \texcommand{\b:hello}. The hooks must be inserted into the 
\texcommand{\hello}, which can be easily done using the \texcommand{\pend:def} and \texcommand{\append:def} commands. These
commands can insert code  at the beginning, respective at the end of the redefined command.

The configuration for \HTML\ must be placed in the \file{tex4ht-html4.tex} file:


% examples/basicpackage/config.cfg
\begin{texsource}
\<configure html4 sample\><<<
\Configure{hello}{\HCode{<span class="hello">}}{\HCode{</span>}}
\Css{.hello{color:red;}}
>>>
\end{texsource}

The configuration for a package must be placed in \texcommand{\<configure html4 packagename\>} block.
% ToDo: write more info


The package name must be also included in \file{mktex4ht-cnf.tex}:

\begin{texsource}
\AddFile{9}{sample}
\end{texsource}

The \file{.4ht} files can be generated simply using 

\begin{shellcommand}
make
\end{shellcommand}

command.

The following sample \TeX\ file:


% examples/basicpackage/hello.tex
\begin{texsource}
\documentclass{article}
\usepackage{sample}
\begin{document}
  \hello\ world.
\end{document}
\end{texsource}

Produces a following \HTML\ code:

\begin{htmlsource}
<!--l. 4--><p class="noindent" >
<span class="hello">world</span> world. 
</p> 
\end{htmlsource}
\chapter{Glossary}
\chapter{Bibliography}
\chapter{Index}

\end{document}
