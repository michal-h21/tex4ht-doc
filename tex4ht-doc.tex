\documentclass{book}
\usepackage{url}
\usepackage{xcolor}
\usepackage{array}

% \usepackage[T1]{fontenc}
\usepackage{upquote}
\usepackage{textcomp}
\usepackage{hyperref}
\usepackage{csquotes}
\usepackage{tex4ht-styles}



\usepackage{glossaries}
\title{TeX4ht Documentation}

\author{by TeX4ht Project}

\begin{document}

\maketitle

% Don't introduce table of contents in the HTML mode, as it introduces another page
\ifdefined\HCode\else\tableofcontents\fi


\chapter{Introduction}

\section{Acknowledgements}

This work was supported with a financial support from \href{https://cstug.cz/}{CSTUG}.

\chapter{Basic Tutorial}
\chapter{Calling Commands}
\label{sec:calling-commands}
\chapter{Output Formats}
\chapter{\texfourht\ Options}

% the list of options has been copied from the CVR's blog
% http://cvr.cc/?p=504



Following is an incomplete list of options that can be passed to the \fourhtsty\ package.
These options can be used to modify the compilation process, for example to
select a \term{SVG} format for generated images, to request math environments
to be convert as images or to split sections as separate HTML pages. 

The options may be defined in the \fourhtfile\ files and may depend on the
output format, so it is not feasible to provide their full list. Most of the
following options work only in the \HTML\ output.

There are several ways how to pass the options to \texfourht. The
non-recommended way is to pass them as options to \fourhtsty\ using
\texcommand{\usepackage} command in the \TeX\ file. 

Better ways don't require modifications of the \TeX\ files.
It is possible to pass the options to the calling script, as an command line
argument next to the filename:
% it is run from the command line. These can also be provided as options when \texfourht
% package is loaded in a \LaTeX\ document with the default usepackage command or to
% the \verb|\Preamble| command in the custom config file

\begin{shellcommand}
make4ht filename.tex "fn-in"
\end{shellcommand}

For more information about calling scripts see the section \ref{sec:calling-commands}.

It is also possible to pass options in the \texcommand{\Preamble} command in a \cfgfile.

\begin{texsource}
\Preamble{fn-in}
...
\begin{document}
\EndPreamble
\end{texsource}



\section{List of options}

% \begin{tabular}{>{\ttfamily}p{8em} l} 
%   -css & to ignore CSS code, use command line option -css. \\
%   -xtpipes & to avoid xtpipes post-processing the output. This might be useful for docbook XML output.\\
%   0 & pagination shall be obtained through the option 0 or 1, at locations marked with PageBreak.\\
%   1, 2, 3, 4, 5, 6, 7& for automatic sectioning pagination (to break at various section levels), use the appropriate command line option 1, 2, 3, 4, 5, 6, 7.
\begingroup
\catcode`\#=11 \catcode`\^=11 \catcode`\_=11


\begin{description}

\item[-css] to ignore \css\ code, use command line option \verb=-css=.

\item[-xtpipes] to avoid \verb=xtpipes= post-processing the
  output. This might be useful for docbook \xml\ output.

  % \item[/bib]
  % \item[/obeylines]
  % \item[0.0]

\item[0] pagination shall be obtained through the option \verb=0= or
  \verb=1=, at locations marked with \verb=\PageBreak=.

\item[1, 2, 3, 4, 5, 6, 7] for automatic sectioning pagination (to
  break at various section levels), use the appropriate command line
  option \verb=1, 2, 3, 4,= \verb=5, 6, 7=.

\item[DOCTYPE] to request a \verb=DOCTYPE= declaration, use the
  command line option \verb=DOCTYPE=.

\item[Gin-dim] for key dimensions of the graphic, try this option.

\item[Gin-dim+] for key dimensions when the bounding box is not
  available.

\item[NoFonts] to ignore \css\ font decoration.

\item[PMath] Option to choose positioned math. Example: 
  \verb=\def\({\PMath$}=;\allowbreak \verb=\def\){$\EndPMath}=;
  \verb=\def\[{\PMath$$}=; \verb=\def\]{$$\EndPMath}=.

\item[RL2LR] to reverse the direction of RL sentences.

%\item[ShowFont]

\item[TocLink] option to request links from the tables of contents.
  
\item[\textasciicircum 13] option for active superscript character.

\item[\_13] option for active subscript character.

\item[accent-] This option is available only together with
  \option{new-accents}. It produces pictures for some math accents.

%\item[base]

\item[bib-] for degraded bibliography friendlier for conversion to
  \verb=.doc=.

\item[bibtex2] Option \verb=bibtex2= requires compilation of
  \verb=\jobname j.aux= with bibtex.

%\item[broken-index]

\item[charset] for alternate character set, use the command line
  option \verb+charset="..."+ (e.g., \verb+charset="utf8"+).

%\item[core]

\item[css-in] the inline \css\ code will be extracted from the input of
  the previous compilation, so an extra compilaion might be needed for
  this option to make it effective.

\item[css2] for \css2 code.

% \item[css]
% \item[debug-]
% \item[debug]
% \item[draw]
% \item[dtd]

\item[early\textasciicircum] for default catcode of superscript in the
  \verb=\Preamble=.

\item[early\_] for default catcode of subscript in the
  \verb=\Preamble=.

%\item[edit]

\item[endnotes] for end notes instead of footnotes, use this option.

%\item[enum]

\item[enumerate+] for enumerated list elements that keep the list couter value. This
  will use the description list like \verb=<dt>...</dt>= for the list
  counter.

\item[enumerate-] for enumerated list element's \verb=<li>='s with
  value attributes, use this command line option. This will be an
  ordered list with the value of list counter provided as an attribute
  namely, \verb=value= of the \verb=<li>= element.

%\item[family]
\item[fancylogo] try to visually emulate \verb|\TeX| and \verb|\LaTeX| logos.

\item[fn-in] for inline footnotes use this option.

\item[fn-out] for offline footnotes.

\item[fonts] for tracing \latex\ font commands, use this command line
  option.

\item[fonts+] for marking of the base font, use this option.

\item[font] for adjusted font size, use the command line option
  \verb+font=...+ (e.g., font=-2).

\item[frames-] for frames support. \verb=frames= is also valid option
  for frames support.

\item[frames-fn] for content, \chfont{TOC}\ and footnotes in
  three frames.

\item[frames] for \chfont{TOC}\ and content in two frames.

%\item[fussy]

\item[gif] for bitmaps of pictures in \verb=.gif= format, use this
  option.

\item[graphics-] if the included graphics are of degraded quality, try
  the command line options \verb=graphics-num= or \verb=graphics-=.
  The \verb=num= should provide the density of pixels in the bitmaps
  (e.g., 110).

%\item[graphics-dim]

\item[hidden-ref] option to hide clickable index and bibliography
  references.

% \item[hooks++]
% \item[hooks+]
% \item[hooks]
% \item[hshow]
% \item[htm3]
% \item[htm4]
% \item[htm5]
% \item[htm]

\item[html+] for stricter \HTML\ code.

%\item[html]

\item[imgdir] for addressing images in a subdirectory, use the option
  \verb=\imgdir:.../=.

\item[image-maps] for \verb=image-maps= support.

\item[index] for \emph{n}-column index, use the command line option,
  \verb+index=n+ (e.g., index=2).

\item[info-oo] for extra tracing information while generating open
  office output.

\item[info] for extra information in the \verb=\jobname.log= file.

\item[java] for \verb=java=support.

\item[javahelp] for \verb=JavaHelp= output format, use this command
  line option.

\item[javascript] for \verb=javascript= support.

\item[jh-] for sources failing to produce \xml\ versions of \HTML, try
  this command line option.

%\item[jh1.0]

\item[jpg] for bitmaps of pictures in \verb=.jpg= format, use this
  option.

\item[li-] for enumerated list elements li's with value attributes.

\item[math-] option to use when sources fail to produce clean math
  code.

\item[mathjax] use \term{MathJax} for the math rendering.
%\item[mathaccent-]

\item[mathltx-] option to use when sources fail to produce clean
  \verb=mathltx= code.

\item[mathml-] option to use when sources fail to produce clean
  \mathml code.

\item[mathplayer] for \mathml\ on Internet Explorer + MathPlayer.

\item[minitoc\textless] for mini tocs immediately after the header use the
  command line option, \verb=minitoc<=.

\item[mouseover] for pop ups on mouse over.

\item[new-accents] alternative configurations for accented characters. 

\item[next] for linear cross-links of pages, use this option.

\item[nikud] for Hebrew vowels, use the command line option,
  \verb=nikud=.

\item[no-DOCTYPE] to remove \texttt{DOCTYPE}\space declaration from
  the output.

\item[no-VERSION] to remove \verb+<?xml version="..."?>+ processing
  instruction from the output.

\item[NoFonts] disable ht-fonts processing in the document.

% \item[no-align]
% \item[no-array]
% \item[no-bib]
% \item[no-cases]
\item[no-halign] suppress \texcommand{\halign} redefinition. It doesn't work with the \texcommand{tabular} environment.
% \item[no-matrix]
% \item[no-pmatrix]

\item[no\textasciicircum] for non-active \verb=^= (superscript), use the option
  \verb=no^=.

\item[no\_] for non-active \verb=_= (subscript command), use the
  command line option, \verb=no_=.

\item[no\_\textasciicircum] for both non-active superscript and subscript, use the
  option \verb=no_^=.

\item[nolayers] to remove overlays of slides, use this option.

\item[nominitoc] this will eliminate mini tables of contents from the
  output.

\item[notoc*] for tocs without \verb=*= entries, use this option. The
  \verb=notoc*= option is applicable only to pages that are
  automatically decomposed into separate web pages along section
  divides. It shall be used when \verb=\addcontentsline= instructions
  are present in the sources.

\item[obj-toc] for frames-like object based table of contents, use the
  command line option \verb=obj-toc=.

%\item[old-longtable]

\item[p-width] for width specifications of tabular \verb=p= entries,
  use this option.

\item[p-indent] for indented paragraphs, without blank spaces.

\item[pic-RL] for pictorial RL.

\item[pic-align] for pictorial align environment.

\item[pic-array] for pictorial array.

\item[pic-cases] for pictorial cases environment.

\item[pic-eqalign] for pictorial equalign environment.

\item[pic-eqnarray] for pictorial eqnarray.

\item[pic-equation] for pictorial equations.

\item[pic-fbox] for pictorial or bitmapped fbox'es.

\item[pic-framebox] for bitmap fameboxes.

\item[pic-longtable] for bitmapped longtable.

\item[pic-m+] for pictorial \verb=$...$= and \verb=$$...$$=
  environments with \latex\ alt, use the command line option
  \verb=pic-m+= (not safe).

\item[pic-m] for pictorial \verb=$...$= environments, use the command
  line option \verb=pic-m= (not recommended).

\item[pic-matrix] for pictorial matrix.

% \item[pic-tabbing']

% \item[pic-tabbing]

% \item[pic-table]

\item[pic-tabular] use this option for pictorial tabular.

\item[plain-] for scaled down implimentation.

% \item[pmathml-css]

% \item[pmathml]

% \item[postscript]

\item[prog-ref] for pointers to code files from root fragments, use
  the command line option \verb=prof-ref=. This is for debugging.

\item[refcaption] for links into captions, instead of flat heads, use
  this option.

\item[rl2lr] to reverse the direction of Hebrew words, use this
  option.

\item[sec-filename] for file names derived from section titles, use
  the command line option \verb=sec-filename=.

\item[sections+] for back links to table of contents, use this option.

% \item[sections-]
% \item[settabs-]
% \item[stackrel-]

\item[svg-] for external \svg\ files, try this option.

\item[svg-obj] same as above.

\item[svg] for dvi pictures in \verb=svg= format.

\item[svg-inline] same as the \option{svg}, but the \svg\ files are included in the document body.

\item[tab-eq] for tab-based layout of equation environment, use this
  option.

%\item[th4]

\item[trace-onmo] for mouseover tracing of compilation, use the
  command line option, \verb=trace-onmo=.

% \item[uni-emacspeak]
% \item[uni-html4]
% \item[uniaccents]
% \item[unicode]

% \item[url-]

\item[url-enc] for \chfont{URL}\space encoding within href, use this
  option.  \verb=\Configure{url-encoder}= can be used to fine tune
  encoding.

\item[url-il2-pl] for il2-pl \chfont{URL} encoding.

\item[ver] for vertically stacked frames. Effective when \verb=frames=
  option is requested.

% \item[verify+]
% \item[verify]

\item[xht] for file name extension, \verb=.xht=, use this command line
  option.

\item[xhtml] for \xml\ code, use the command line option, \verb=xml= or
  \verb=xhtml=.

\item[xml] See previous entry.

% \item[xmldtd]

\end{description}
\endgroup


\chapter{Configurations}
\section{Configuration Files}
\section{Private Configuration Files}\label{sec:private-configuration}

The leading entry, in the first list of options of the \shellcmd{htlatex}-like
commands, can equal \option{html} or \option{xhtml}. If this is not the case,
the entry is assumed to be the name of a configuration file. The extension
‘cfg’ is assumed for names of configuration files that are listed without their
extension.

A configuration file should take the following form for LaTeX files.

\begin{texsource}
...early definitions...
\Preamble{options}
...definitions...
\begin{document}
...insertions into the header of the html file...
\EndPreamble
\end{texsource}

It is up to the user to decide the distribution of entries between the \texcommand{\Preamble} and the htlatex-like commands.

Example: The command \shellcmd{htlatex myfile "mycfg,2"} requests the
compilation of a file named \file{myfile.tex}, in the presence of a
configuration file named \file{mycfg.cfg}. The configuration file might have the
following content.

\begin{texsource}
\Preamble{html} 
\begin{document} 
  \Css{body { color : red; }} 
\EndPreamble 
\end{texsource}

Notes

\begin{itemize}
  \item Notice that for a LaTeX file the \texcommand{\begin{document}}
    instruction should be present both in the configuration file and the source
    file.

  \item Instructions defined within a source file may be redefined in a
    configuration file. Such a feature enables to keep source files intact for
    compilation to different formats by different tools.
\end{itemize}

For instance, a definition of the form \texcommand{\renewcommand\mycommand{...}} within a
configuration file provided for the following LaTeX source.

\begin{texsource}
\documentclass{...} 
\newcommand\mycommand{...} 
\begin{document} 
Use \mycommand{...} 
\end{document} 
\end{texsource}

\subsection{Configuration file management}

It is possible to reuse common \texfourht\ configurations used in several
configuration files.  They can be inserted in a custom LaTeX package, but there
is one important thing to be aware of. The configuration hooks are inserted to
the patched commands when the compilation reaches the  
\texcommand{\begin{document}} command, so configurations for these hooks
declared before the hook definition have no effect. It is necessary to include
them in the \texcommand{\AtBeginDocument} command.

Sample package, \file{commonconfigurations.sty}:

\begin{texsource}
\ProvidesPackage{commonconfigurations}
\AtBeginDocument{%
\Configure{@HEAD}
{\HCode{<meta name="test" content="test"/>\Hnewline}}
}
\endinput
\end{texsource}

It can be requested in a configuration file using \texcommand{\RequirePackage} command.

\begin{texsource}
\Preamble{xhtml}
\RequirePackage{commonconfigurations}
\begin{document}
\EndPreamble
\end{texsource}



\section{tex4ht Commands}
\subsection{Low-level \texfourht\ Commands}

\DocCommand{Configure}\marg{name}\marg{arg 1}\ldots\marg{arg n}

The \cmd{Configure} command declares code that is inserted into hooks related to the \textit{name} configuration.
It is possible to define up to nine hooks, so number of arguments is variable.


\DocCommand{ConfigureEnv}\marg{\textless environment name\textgreater}\marg{before env}\marg{after env}
\marg{before-list}\marg{after-list}


\DocCommand{HCode}\marg{output format markup} is a basic command for insertion
of the output format markup, as it's content is not escaped for the \textless{}
and \textgreater.

This command allows only for the expansion of macros, before sending its content to the output. 
The instruction \texcommand{\Hnewline} may be introduced there for requesting line breaks, and the command \texcommand{\#} may be used for the sharp symbol ‘\#’.

\begin{texsource}
First line\HCode{<br />}
second line 

You don't want to include tags directly into the document '<br>'. 
\end{texsource}

\DocCommand{Tg}\verb|<markup>| is a variation of the \cmd{HCode} command that don't require braces, and it does some additional processing.

\DocCommand{ifOption}\marg{name of the option to be checked}\marg{true part}\marg{false part}

This command can be used in the private configuration files to test if a custom option was used

\subsection{Hyperlinks}

\DocCommand{Link}\texttt{[target-file arguments]}\marg{target-loc}\marg{cur-loc} text inside link\cmd{EndLink}

This command requests an anchor that links to \verb|target-file#target-loc|, and marks the current location with the name \verb|‘cur-loc’|.

The component in square brackets \texttt{‘[...]’} is optional when it is empty, 
and the target file need not be mentioned if it is created from the current source file.


\DocCommand{LinkCommand} creates a \cmd{Link}\textit{-like} command. It has variable number for parameters:

\begin{enumerate}
  \item tag name
  \item href-like attribute
  \item name-like attribute
  \item insertion
  \item /, if empty element
  \item replacement for \texcommand{#} character  (ignored if absent)
\end{enumerate}

Example:

\begin{texsource}
\LinkCommand\JSLink{a,\noexpand\jsref,name}
\def\jsref="#1"{href="javascript:window.open('#1')"}

% example of use of the defined command:
\JSLink{a}{}xx\EndJSLink % link to a destination
\Link{}{a}\EndLink       % set link destination (or by \JSLink{}{a}\EndJSLink)
\end{texsource}


\subsection{Paragraph Handling}
\label{sec:paragraph_handling}

Paragraph handling is one of the more complicated areas in \texfourht.
You must handle insertion of tags for paragraph opening and closing,
to prevent wrong nesting of XML tags. Mismatch of tags leads to issues with 
LuaXML post-processing of the generated files, preventing many fixes 
that are necessary for correct conversion.

\DocCommand{HtmlParOff} turns off insertion of paragraph tags in the following text.

\DocCommand{HtmlParOn} enables insertion of paragraphs tags.

\DocCommand{IgnorePar} asks to ignore the next paragraph.
\DocCommand{ShowPar} asks to take into account the following paragraphs.

\DocCommand{IgnoreIndent}  asks to ignore indentation in the next paragraph.
\DocCommand{ShowIndent}    asks to check indentation in the following paragraphs.

\DocCommand{SaveEndP}  saves the content of \cmd{EndP}, and sets it to empty content.
\DocCommand{RecallEndP} resets the content of \cmd{EndP}.

\DocCommand{SaveHtmlPar} saves current configuration for paragraphs. It can be
useful before local declaration of \cmd{Configure}\marg{HtmlPar}.
\DocCommand{RecallHtmlPar} resets configuration for paragraphs to the value saved by
\cmd{SaveHtmlPar}.


The following example adds \htmlcommand{<div>...</div>} tags around contents of the document body.
The \texcommand{\ifvmode\IgnorePar\fi} commands will prevent insertion of the \htmlcommand{<p>} tag 
before \htmlcommand{<div>} if we are in \TeX's vertical mode. The \texcommand{\EndP} closes currently
opened paragraph, if it is opened. The \texcommand{\par\ShowPar} commands start new paragraph
after the inserted \htmlcommand{<div>} tag. It is necessary to explicitly start paragraphs sometimes.

\begin{texsource}
\Configure{@BODY}
{\ifvmode\IgnorePar\fi\EndP
 \HCode{<div>}\par\ShowPar}
\Configure{@/BODY}
{\ifvmode\IgnorePar\fi\EndP
 \HCode{</div>}}
\end{texsource}


\DocConfigure{HtmlPar} {content at the start non-indented paragraphs} 
   {content at the start indented paragraphs}
   {insertion into \cmd{EndP}, at the start of non-indented paragraphs}
   {insertion into \cmd{EndP}, at the start of indented paragraphs} \EndDoc

Example:

\begin{texsource}
\Configure{HtmlPar}
{\EndP\HCode{<p class="indent">}}
{\EndP\HCode{<p class="noindent">}}
{\HCode{</p>}}
{\HCode{</p>}}
\end{texsource}

% https://tex.stackexchange.com/a/66172/2891


\subsection{Logical Document Structure Commands}
I've created an alternative commands to \cmd{HCode} or \cmd{Tg}. 
The idea is to define semantic names for logical blocks of the document, such as titles, authors,
sections etc. HTML elements and attributes can be assigned to these
logical blocks. There are commands for inline and block level elements,
eliminating need for constructs like \texcommand{\ifvmode\IgnorePar\fi\EndP}
etc.

\DocCommand{NewLogicalBlock}\marg{name} create a new logical block.
\DocCommand{SetBlockProperty}\marg{name}\marg{attribute name}\marg{value} set block attribute .
\DocCommand{SetTag}\marg{name}{tag name} assign element name to the logical block
\DocCommand{BlockElementStart}\marg{name}\marg{additional attributes} start block level element.
\DocCommand{BlockElementEnd}\marg{name} close block level element.
\DocCommand{InlineElementStart}\marg{name}\marg{additional attributes} start inline level element.
\DocCommand{InlineElementEnd}\marg{name} close inline level element.


The default tag name for declared logical blocks is \htmlcommand{<span>}. You
need to use the \cmd{SetTag} command only when you want to use a different
element.

The attributes can be set either using \cmd{SetBlockProperty}, or in the second
argument to  \cmd{BlockElementStart} and \cmd{InlineElementStart} commands. First method should be
used for static attributes that don't change, for instance \textit{class}. The second method
is preferred for dynamic attributes, such as \textit{id}, which should be different for 
every element.

The main idea behind this mechanism is to allow easy work with new HTML5
elements and attributes for WAI-ARIA or Schema.org properties. I hope that
this should allow us to make somehow more clear configurations for basic
document building blocks.

Example:


\begin{texsource}
\NewLogicalBlock{textit}
\SetBlockProperty{textit}{class}{textit}

\NewLogicalBlock{maketitle}
\SetTag{maketitle}{header}

\NewLogicalBlock{titlehead}
\SetTag{titlehead}{h1}
\SetBlockProperty{titlehead}{class}{titleHead}

% configure \textit using inline level elements
\Configure{textit}
{\NoFonts\InlineElementStart{textit}{}}
{\InlineElementEnd{textit}\EndNoFonts}

% configure \maketitle using block level elements
\Configure{maketitle}{%
{\Configure{maketitle}{}{}{}{}%
\Tag{TITLE+}{\@title}}
\BlockElementStart{maketitle}{}}
{\BlockElementEnd{maketitle}}
{\NoFonts\BlockElementStart{titlehead}{}}
{\BlockElementEnd{titlehead}\EndNoFonts}
\end{texsource}







\section{Styling the Document}

\section{Use Webfonts}
\section{Webfonts}
\label{sec:webfonts}


The declared font family is not used automatically, it is necessary to select
it using the \term{font-family} Css property.

The default font family name which should be used in the Css
\term{font-family} command for a declared font is \term{rmfamily}. 
It use the Latin Modern font installed on the viewer's system. 

\DocConfigure {FontFamily} {cssfamilyname} {LocalFontName}\EndDoc

Change default CSS font family name. Example:

\begin{texsource}
\Configure{FontFamily}{rmfamily}{Latin Modern}
\end{texsource}

The font shapes can be configure using \cmd{Configure}\marg{NormalFont}, 
\cmd{Configure}\marg{ItalicFont}, \cmd{Configure}\marg{BoldItalicFont} and
\cmd{Configure}\marg{BoldFont}. The argument should be font file in the format
supported by browsers, such as \textit{woff} or \textit{otf}.

Full example of font CSS configurations:

\begin{texsource}
\Configure{NormalFont}{normal-font-file.otf}
\Configure{BoldFont}{bold-font-file.otf}
\Configure{BoldItalicFont}{bold-italic-font-file.otf}
\Configure{ItalicFont}{italic-font-file.otf}
% Add another font family
\Configure{FontFamily}{hello}{Linux Libertine O}
\Configure{NormalFont}{hello-font-file.otf}
\Css{body{
  font-family: rmfamily, "AnotherFontFamilyName", serif;
}}
\Css{span.hello{font-family: hello, sans-serif;}}
\end{texsource}

\section{Use JavaScript}
\section{Document Navigation}

\subsection{Cross-links}

The following configurations modify behaviour of cross-links between pages in a multi page document.

\DocConfigure{crosslinks} {left-delimiter} {right-delimiter} {next} {prev} {prev-tail} {front} {tail} {up}\EndDoc

This command configures the appearance of the cross-links between hypertext pages obtained for sectioning commands.

\begin{texsource}
 \Configure{crosslinks}
   {}{}{$\scriptstyle\Rightarrow$}
   {$\scriptstyle\Leftarrow$}
   {}{}{}{$\scriptstyle\Uparrow$}
\end{texsource}

\DocConfigure{crosslinks*} {1--7 arguments}\EndDoc

  Links to be included and their order. Available
  options: next, prev, prevtail, tail, front, up.
  The last argument must be empty.

  Default:

\begin{texsource}
\Configure{crosslinks*}{next}
   {prev}{prevtail}
   {tail}{front}
   {up}{}
\end{texsource}

\DocConfigure{crosslinks+} {before-top-links} {after-top-links} {before-bottom-links} {after-bottob-links}\EndDoc

The top cross links are omitted, if both \verb|#1| and \verb|#2| are empty.
The bottom cross links are omitted, if both \verb|#3| and \verb|#4| are empty.

\DocConfigure{next} {the anchor of the next button of the front page}\EndDoc

Default: The value provided in \texcommand{\Configure{crosslinks}}

\DocConfigure{next+}{before} {after}\EndDoc

\begin{description}
  \item[\#1]  before the next button of the front page, when the `next'
       option is active.
  \item[\#2]  after the button
\end{description}

    Default: The values provided in \texcommand{\Configure{crosslinks}}

\begin{texsource}
\Configure{crosslinks:next}..................1
\Configure{crosslinks:prev}..................1
\Configure{crosslinks:prevtail}..............1
\Configure{crosslinks:tail}..................1
\Configure{crosslinks:front}.................1
\Configure{crosslinks:up}....................1
\end{texsource}

  \verb|#1| local configurations for the delimiters and hooks

\DocConfigure{crosslinks-}{before} {after}\EndDoc

Asks to show linkless buttons with the following insertions.

The default values are used, if both \verb|#1| and \verb|#2| are empty

   Examples:

\begin{texsource}
\Configure{crosslinks-}{}{}

\Configure{crosslinks-}
    {\HCode{<span class="hidden">}[}
    {]\HCode{</span>} }
\Css{span.hidden {visibility:hidden;}}
\end{texsource}

\section{Tables of Contents}

\section{Sections}
\section{Lists}
\section{Tables}

\section{Fonts}
\subsection{Basic font commands}

Information about the \option{fonts} option and \term{MathML} issues. 
Example configuration:
\url{https://tex.stackexchange.com/a/416613/2891}
\section{Colors}

\section{Graphics and Pictures}

\subsection{Low level features}

\DocConfigure{Picture} {Extension name pictures generated by DVI conversion, stored in \cmd{PictExt}}\EndDoc

Default: 

\begin{texsource}
\Configure{Picture}{.png}
\end{texsource}

  The extension names of bitmap files of glyphs of htf fonts may be
  determined within a g-entry in the environment file \texttt{tex4ht.env}, or a
  g-flag of the \shellcmd{tex4ht} utility.

\DocConfigure{Picture-alt} {alt value for \cmd{Picture+}\marg{...}  and \cmd{Picture*}\marg{...}}\EndDoc


\DocConfigure{Picture+} {before the dvi picture code} {after the dvi picture code}\EndDoc
\DocConfigure{Picture*} {before the dvi picture code} {after the dvi picture code}\EndDoc

  Typically, the plus `+' variant is introduced as an inline
  contribution into paragraphs, and the star `*' variant as an
  independent block between paragraphs.

\DocConfigure{PictureAlt} {definitions before alt} {definitions after alt}\EndDoc
\DocConfigure{PictureAlt*+} {definitions before alt} {definitions after alt}\EndDoc
\DocConfigure{PictureAlt*+[]} {definitions before alt} {definitions after alt}\EndDoc

Apply to \cmd{Picture}, \cmd{Picture*+}, and \cmd{Picture*+[...]}


\DocConfigure{IMG}
{before file name}
{between file name and alt}
{close alt for  \cmd{Picture} without * or +}
{close alt for  \cmd{Picture} with * and +}
{right delimiter}\EndDoc

  Example:

\begin{texsource}
\Configure{IMG}
  {\ht:special{t4ht=<img src="}}
  {\ht:special{t4ht=" alt="}}
  {" }
  {\ht:special{t4ht=" }}
  {\ht:special{t4ht=/>}}
\end{texsource}

\DocCommand{NextPictureFile}\marg{filename} 

   Requests a file name for the next created picture.

\cmd{PictureFile}

   Records the filename of the most recent created picture.

\subsection{Configurations for the \package{Graphics} package bundle}

\DocConfigure{graphics}{before graphics}{after graphics}\EndDoc


Examples:

\begin{texsource}
\Configure{graphics}
   {\Picture+[PIC]{ class="graphics"}}
   {\EndPicture }
\end{texsource}


\DocConfigure{graphics*}
{extension name}
{insertion}\EndDoc


Allows to configure \texfourht{} for graphics files named in
the \cmd{includegraphics} macro, based on the type of the files.


An empty insertion for the second argument cancels previous requests for the
specified extension.

You can utilise the macros that contain information about the image, for example
\cmd{Gin@base} (file name), \cmd{Gin@ext} (extension), \cmd{Gin@req@width} (requested image width), \cmd{Gin@req@height} (requested image height),
    \noBoundingBox (defined iff bounding box is unknown)

    Example:

\begin{texsource}
\Configure{graphics*}
{jpg}
{\Picture[pict]{\csname Gin@base\endcsname.jpg}}

\Configure{graphics*}
{wmf}
{\Needs{"convert \csname Gin@base\endcsname.wmf
 \csname Gin@base\endcsname.gif"}%
 \Picture[pict]{\csname Gin@base\endcsname.gif
 width="\expandafter\the\csname Gin@req@width\endcsname"
 height="\expandafter\the\csname Gin@req@height\endcsname"}%
}

\Configure{graphics*}
{eps}
{\openin15=\csname Gin@base\endcsname\PictExt\relax
 \ifeof15 % test if the converted file already exists
 \Needs{"convert \csname Gin@base\endcsname.eps
 \csname Gin@base\endcsname\PictExt"}%
 \fi
 \closein15
 \Picture[pict]{\csname Gin@base\endcsname\PictExt}%
}
\end{texsource}

\subsection{PDF support}
\DocConfigure{PdfConvert}{}{}\EndDoc
\DocConfigure{Ghostscript} {name of the executable for GhostScript}\EndDoc

\subsection{TikZ }

Animations using Animate package: \url{https://tex.stackexchange.com/a/404600/2891}

Issues with drivers: \url{https://tex.stackexchange.com/a/471460/2891}.
\subsection{Pstricks}

\section{TikZ }

Animations using Animate package: \url{https://tex.stackexchange.com/a/404600/2891}
\section{Pstricks}

\section{Math}
\subsection{Default math handling}
\subsection{MathML}
\subsection{MathJax}

The \term{MathJax} processing can be required using \option{mathjax} option.
The address of \term{MathJax} script with its configuration string can be
specified in \configuration{MathjaxSource}. The default value of this configuration is:

\begin{texsource}
\Configure{MathjaxSource}
{https://cdnjs.cloudflare.com/ajax/libs/mathjax/2.7.5/latest.js?config=TeX-AMS-MML_HTMLorMML}
\end{texsource}

By default, inline and display math, as well as math environments, are kept as
raw LaTeX code in the \HTML\ output. \term{MathJax} then transforms this code
and displays it. In the \term{MathML} mode, \term{MathJax} is used only for the
rendering in browsers without \term{MathML} support. 

The additional configuration for \term{MathJax} can be provided in special
script environment in the \HTML\ page header. The following example provides
support for some custom \LaTeX\ macros.

\begin{texsource}
\Configure{@HEAD}{\HCode{\detokenize{%
<script type="text/x-mathjax-config">                                           
  MathJax.Hub.Config({
    TeX: {           
      Macros: {     
        sc : "\\small\\rm",
        sl: "\\it", 
      }        
    },        
    }                   
  );                  
</script>   
}}}
\end{texsource}

The \texcommand{\detokenize} macro is used to avoid issues with backslash
characters used in the macro definitions. Backslashes must be doubled in the
JavaScript strings.


\section{Bibliographies}
\section{Indexing}

\chapter{Make4ht Build Files}
\section{Commands execution}
\section{Filters}

Some samples:

\begin{itemize}
  \item Render math by Mathjax during tex4ht compilation \url{https://tex.stackexchange.com/a/402159/2891}
\end{itemize}
\section{Image conversion}

\chapter{For developers}
% \section{Introduction}

This chapter deals with \texfourht\ development. It starts with a basic
tutorial for a new package support, shows commands useful in the process,
different types of \texfourht\ configuration files, and the syntax and structure of 
literate source files.

\section{Tutorial: Basic Support For a New Package}

In this tutorial, we will try to show how to provide \texfourht\ support for a
simple \LaTeX\ package. 

% from https://tex.stackexchange.com/a/402283/2891
\texfourht\ tries to load a special \file{.4ht} file for each package loaded
by \LaTeX. This special file can contain modifications to commands provided by the package, like 
redefinitions of macros that cause clashes between the package and \texfourht, and most importantly
they insert special macros, called hooks, that are then used to include the output format tags.

Let's say that you have a custom package, called \file{mynote.sty}

\begin{texsource}
\newcommand\notetitle{Note:~}
\newcommand\note[1]{\textbf{\notetitle}#1}
\newcommand\highlight[1]{\textbf{#1}}
\endinput
\end{texsource}

It defines two user commands, \cmd{note} and \cmd{highlight}. 
They can be used in the following way:


\begin{texsource}
\documentclass{article}
\usepackage{mynote}
\begin{document}
\note{This is a note}

Try to highlight \highlight{something}.
\end{document}
\end{texsource}

\texfourht\ produces usable output for both of these commands out of the box, 
thanks to the support for \TeX\ fonts. But you may want to use custom HTML 
tags instead. To achieve that, you need to insert special commands, called hooks 
in \texfourht, to package commands. These hooks can be then configured to
insert tags in the output format.

To introduce hooks, you need to create a hook seeding configuration file for the package,
called \file{<name>.4ht}. For example, to seed hooks for the \file{mynote.sty} package, create file
\file{mynote.4ht}:

\begin{texsource}
\NewConfigure{note}{3}

% Use \HLet when you want to completely redefine a command
\def\:tempa#1{\a:note\notetitle\b:note~#1\c:note}
\HLet\note\:tempa

\NewConfigure{highlight}{2}
\pend:defI\highlight{\a:highlight}
\append:defI\highlight{\b:highlight}

\Hinput{mynote}
\endinput
\end{texsource}

There is several things to note. First is that the \verb|:| character 
can be included as a part of a command name in \file{.4ht} files. It is similar
to use of the \verb|@| character in \LaTeX\ packages. It allows us to 
create command names that don't clash with other command names.

The hooks are created using the \cmd{NewConfigure} command. They can be
later filled with the \cmd{Configure} command. To have an effect, hooks
must be inserted to the existing commands. There are two ways how to do that.
For simpler commands, where we want to insert tags only before and after 
the contents produced by the patched command, we can use the \cmd{pend:def<X>} and 
\cmd{append:def<X>} commands, where the \verb|<X>| is a roman number of parameters
that the patched command expects. In this example, it expects one parameter, 
so we can use the \cmd{pend:defI} command. For commands without parameters, use 
\cmd{pend:def}.

Of course, you can also insert hooks using other mechanisms, for example using
\LaTeX's hook system:

\begin{texsource}
\AddToHook{cmd/highlight/before}{\a:highlight}
\AddToHook{cmd/highlight/after}{\b:highlight}
\end{texsource}

The second way for hook insertion, useful for commands where we want to insert
tags also inside it's contents, is to use the \cmd{HLet} command. It is a
variant of the \cmd{let} command.  In contrast to \cmd{let}, it saves the
original command as \cmd{o:<command name>:}.  Commands redefined by \cmd{HLet}
also support the \cmd{Picture} command, where the original version of the
command will be used. This way, pictures will produce the same result as they
would produce in the PDF mode.

In our example, we redefined the \cmd{note} command to use a hook between note title
and note text. This enables us to style both the title and the text differently.


The configuration file for our hooks could look like this:

\begin{texsource}
\Preamble{xhtml}
\Configure{note}
{\ifvmode\IgnorePar\fi\EndP\HCode{<div class="note"><span class="notetitle">}}
{\HCode{</span><span class="notebody">}}
{\HCode{</span></div>}}
\Css{.notetitle{font-weight: bold;}}

\Configure{highlight}{\HCode{<span class="highlight">}\NoFonts}{\EndNoFonts\HCode{</span>}}
\Css{.highlight{font-weight:bold;}}
\begin{document}
\EndPreamble
\end{texsource}

As the \cmd{note} command should be used on it's own paragraph, we need to 
fix paragraph closing. See the \namerefpage{sec:paragraph_handling} section for
more information about this issue. More details about configuration files and configurations are
in section \namerefpage{sec:private-configuration}.

The HTML code produced by our configuration looks like this:

\begin{htmlsource}
<div class='note'><span class='notetitle'>Note: </span><span class='notebody'> This is a note</span></div>
<!--  l. 6  --><p class='indent'>   Try to highlight <span class='highlight'>something</span>.
</p>
\end{htmlsource}


\section{Commands Usable in the \file{.4ht} files}

\DocCommand{NewConfigure}\marg{name}\marg{number of defined hooks}

This command defines macros with an alphabetic prefix in the form of 
\cmd{a:name} \ldots \cmd{i:name}, depending on the number of defined hooks.
The maximum number is 9.

\begin{texsource}
\NewConfigure{try}{2}
\def\try#1{\a:try#1\b:try}
\Configure{try}{* }{}  
\try{ho} 
% produces "* ho"
\end{texsource}

\DocCommand{NewConfigure}\marg{name}\oarg{number or parameters}\marg{code}

Variant of \cmd{NewConfigure} that doesn't define hooks with 
alphabetic prefixes, but it passes argumens of \cmd{Configure}
as \TeX\ arguments. See this exampe:

\begin{texsource}
\NewConfigure{try}[2]{\def\hookI{#1}\def\hookII{#2}}
\def\try#1{\hookI#1\hookII}
\Configure{try}{* }{}  
\try{ho} 
% produces "* ho"
\end{texsource}

When you use \texcommand{\Configure{try}}, it defines \cmd{hookI} and \cmd{hookII}
commands. They can be then used in the redefined \cmd{try} command.

\DocCommand{HLet}\marg{Redefined command name}\marg{new command}

Variant of \cmd{let} that saves the original command under \cmd{\o:<name>:} name.
It can detect use of the redefined command inside picture. In such case, it will use
the original command to produce correct visual result in the picture.

\begin{texsource}
\NewConfigure{note}{3}
\def\:tempa#1{\a:note note:\b:note~#1\c:note}
\HLet\note\:tempa
\Configure{note}{*}{*}{*}
\note{hello}
% produces: "* note:* hello*
\end{texsource}

\DocCommand{HRestore}\marg{command name}

Restore command redefined using \cmd{HLet} to it's original content.

\DocCommand{pend:def<X>}\marg{redefined command}\marg{code to be inserted at the begin}

\DocCommand{append:def<X>}\marg{redefined command}\marg{code to be inserted at the end}

These two commands inserts code before and after a redefined command. There are several
versions of these commands, depending on the number of parameters that the redefined 
command expects. Number of parameters as roman number replaces the \verb|<X>| placeholder. 

Up to three parameters are supported.


\begin{texsource}
\newcommand\bar{xxx}
\pend:def\bar{*}
\append:def\bar{*}
\newcommand\foo[2]{#1, #2}
\pend:defII\foo{*}
\append:defII\foo{*}
\end{texsource}


\section{Two types of .4ht files}

% text from the old documentation:
% https://tug.org/tex4ht/doc/mn11.html#QQ1-11-66

The compilation starts by opening tex4ht.sty and loading a fraction of its code.
The main purpose of this phase is to request the loading of the system at a
later time (for instance, upon reaching \texcommand{\begin{document}}). The motivation for
the late loading is to allow TeX4ht to collect as much information as possible
about the environment requested by the source file, and help the system reshape
that environment with minimal interference from elsewhere.

The system uses two kinds of (4ht) configuration files. The files of the first
kind mainly seed hooks into the macros loaded by the source file (for instance,
\file{latex.4ht}, \file{fontmath.4ht}, and \file{article.4ht}).
The files of the second kind mainly
attach meaning to the hooks (for instance, \file{html4.4ht}, \file{unicode.4ht}, and
\file{mathml.4ht}).

Different source files may request the loading of different style files and in
different orders. The hook seeding files are loaded in response to the loading
of the style files, and in a compatible order. Since the different style files
may redefine the syntax and semantics of macros, \texfourht\ follows a similar route
of defining and redefining the hooks and their meanings.

The meaning attaching files are normally requested through option names
introduced in the \file{tex4ht.4ht} system file. It defines options for all output formats
supported by \texfourht. For instance, \option{html5}, \option{ooffice} for the ODT output,
\option{tei}, and so on.

% For instance, the mzlatex command
% refers to the mozilla option name of tex4ht.4ht, and the oolatex command refers
% to the ooffice option name. 

The user may add option names, and redefine old
ones, within a new file named tex4ht.usr.

% \subsection{Inserting configurable hooks for packages}



% \subsection{Configure the hooks in output format configuration files}

\section{\texfourht\ literate sources}

To add a proper support for a new package, it is necessary to edit the 
\texfourht\ literate sources. 
The source files are available in the \href{https://puszcza.gnu.org.ua/projects/tex4ht/}{\texfourht\ source repository}.
You can retrieve them using a SVN client. 

\begin{shellcommand}
$ svn checkout https://svn.gnu.org.ua/sources/tex4ht/
$ cd tex4ht/trunk/lit/
\end{shellcommand}


The configurable hooks for all packages are contained by the \file{tex4ht-4ht.tex} file.
Configurations of these hooks is placed in the output format configuration files.
The most common output format is \HTML, which can be configured in \file{tex4ht-html4.tex}, or 
\file{tex4ht-html5.tex} if \HTMLV\ features are used. You can also update sources for other output
formats, for example \file{tex4th-ooffice.tex} for the ODT format, or \file{tex4ht-tei.tex} for TEI.
The sources of the \file{tex4ht.sty} package are available in \file{tex4ht-sty.tex}.

To compile all literate sources, run the \shellcmd{make} command. You will need basic UNIX utilities 
for this to succeed, as well as \shellcmd{m4} and \shellcmd{javac}. You can also compile particular source
files. Most of them can be compiled using \LaTeX, but some of them, for example \file{tex4ht-4ht.tex}, needs
to be compiled using \shellcmd{etex}.

\subsection{How to add support for a package to the \texfourht\ literate sources}

Given following package \file{sample.sty}:

\begin{texsource}
\ProvidesPackage{sample}
\newcommand\hello{hello}
\endinput
\end{texsource}

This simple package defines command \texcommand{\hello}, which simply prints the word \enquote{hello} when used in a document.

Let's say that we want to insert some \HTML\ tags before and after the text content printed by the command.

Basic template for \file{tex4ht-4ht.tex}:

% examples/basicpackage/sample.4ht
\begin{texsource}
\<sample.4ht\><<<
% sample.4ht (|version), generated from |jobname.tex
% Copyright 2017 TeX Users Group
|<TeX4ht license text|>
\NewConfigure{hello}{2}
\pend:def\hello{\a:hello}
\append:def\hello{\b:hello}
\Hinput{sample}
\endinput
>>> \AddFile{9}{sample}
\end{texsource}

Configuration for each package must follow this basic template. The \ProTeX\ system is used as system for literate programming.

The \verb|\<name\><<<code>>>| block defines new macro which can be then called using \texcommand{|<name|>}. The license text
is included in this way in the example. The instruction to generate the \file{.4ht} file is given in the 
command \texcommand{\AddFile{9}{sample}} after the block definition. The first argument to \cmd{AddFile} is an arbitrary number.


Each package configuration  must include \texcommand{\Hinput{packagename}}, in order to load the configurations for the package.

The command \texcommand{\NewConfigure{hello}{2}} declares new configuration \texttt{hello}, with two configurable hooks. 
These hooks are named  \texcommand{\a:hello} and \texcommand{\b:hello}. The hooks must be inserted into the 
\texcommand{\hello}, which can be easily done using the \texcommand{\pend:def} and \texcommand{\append:def} commands. These
commands can insert code  at the beginning, respective at the end of the redefined command.

The package name must be also included in the \file{mktex4ht-cnf.tex} file. This file is used in the generation of the 

\begin{texsource}
\AddFile{9}{sample}
\end{texsource}

You can place configuration for \HTML\ to the \file{tex4ht-html4.tex} file:

% examples/basicpackage/config.cfg
\begin{texsource}
\<configure html4 sample\><<<
\Configure{hello}{\HCode{<span class="hello">}}{\HCode{</span>}}
\Css{.hello{color:red;}}
>>>
\end{texsource}

The \texcommand{\<configure html4 packagename\>} block will produce code that 
detects use of the package \file{packagename}. It then loads configurations
for the package.


The \file{.4ht} files can be generated simply using the \shellcmd{make} command.

The following sample \TeX\ file:

% examples/basicpackage/hello.tex
\begin{texsource}
\documentclass{article}
\usepackage{sample}
\begin{document}
  \hello\ world.
\end{document}
\end{texsource}

Produces a following \HTML\ code:

\begin{htmlsource}
<!--l. 4--><p class="noindent" >
<span class="hello">hello</span> world. 
</p> 
\end{htmlsource}




\section{ProTeX}


The literate programming system used in the previous section is called ProTeX. We should discuss some main ideas behind this system.

% copied from
% https://www.slac.stanford.edu/comp/unix/package/tex/tex4ht/mn2.html - it
% seems like an older version of documentation which contains some information later ommited

Literate programming is a discipline that promotes the writing of programs the
way one explains them to human beings. ProTeX is a literate programming system
fully implemented in terms of TeX, and it is compatible with LaTeX and other
TeX-base systems. TeX4ht, and ProTeX itself, are examples of applications
written in ProTeX.


\begin{texsource}
\input ProTex.sty
\AlProTex{extension,<<<>>>,list,title,escape-character}
\<title\><<<
code fragment
>>>  
|<title|>
\OutputCode\<...\> 
\end{texsource}

Some explanation:

\begin{texsource}
\input ProTex.sty
\AlProTex{extension,<<<>>>,list,title,escape-character}
\end{texsource}

The escape-character stands for `, @, |, or ?. If omitted, it stands for \verb'|'. 

\begin{texsource}
\<title\><<<
code fragment
>>>

\end{texsource}

This structure provides names to code fragments (the fragments should not be too large in size).


\begin{texsource}
 |<title|>
 \end{texsource}

 This command acts as a place holder for the code segment associated to the title (\texttt{|} stands for the escape character). 

\begin{texsource}
   \OutputCode\<...\>
 \end{texsource}

This command creates a file for the code whose root node is specified.





\chapter{Glossary}
\chapter{Bibliography}
\chapter{Index}

\end{document}
